\documentclass[12pt,a4paper]{article}

\usepackage[utf8]{inputenc}
\usepackage[T2A]{fontenc}
\usepackage[russian]{babel}

% --- Шрифты и математика ---
\usepackage{amsmath,amssymb}
\usepackage{graphicx}
\usepackage{ragged2e}


\usepackage{amsmath,amssymb}

\usepackage{geometry}
\geometry{margin=1cm}
\linespread{1}
\usepackage{enumitem}

\begin{document}

\section{Карта областей и соответствие пунктам программы}

Сокращения областей:
\begin{itemize}
    \item МА --- математический анализ / вещественный анализ;
    \item ФА --- функциональный анализ / теория меры;
    \item ЛА --- линейная алгебра;
    \item АГ --- аналитическая геометрия;
    \item ОДУ --- обыкновенные дифференциальные уравнения;
    \item ЧПУ --- уравнения в частных производных;
    \item КА --- теория функций комплексного переменного (ТФКП);
    \item ВИ --- вариационное исчисление;
    \item ВМ --- вычислительная математика.
\end{itemize}

\subsection{МА: вещественный и математический анализ}

\begin{itemize}
    \item Пункты: 1--11, 31 (частично).
    \item пункт 1. Предел, непрерывность функции одной переменной, свойства непрерывной функции на отрезке. Понятие производной.
    \item пункт 2. Функции многих переменных, полный дифференциал, геометрический смысл, достаточные условия дифференцируемости, градиент.
    \item пункт 3. Первообразная и неопределённый интеграл. Интеграл Римана.
    \item пункт 4. Определённый интеграл. Интегрируемость непрерывной функции. Формула Ньютона--Лейбница.
    \item пункт 5. Числовой ряд и его сходимость. Необходимый признак сходимости. Признаки Даламбера и Коши. Интегральный признак.
    \item пункт 6. Абсолютная и условная сходимость ряда. Свойство абсолютно сходящихся рядов. Умножение рядов.
    \item пункт 7. Функциональный ряд. Равномерная сходимость. Признак Вейерштрасса. Свойства равномерно сходящихся рядов.
    \item пункт 8. Степенной ряд и его радиус сходимости. Почленное интегрирование и дифференцирование. Ряд Тейлора.
    \item пункт 9. Несобственные интегралы и их сходимость.
    \item пункт 10. Ряд Фурье. Тригонометрический ряд Фурье. Достаточные условия представимости функции рядом Фурье.
    \item пункт 11. Метрическое пространство, полнота, компактность. Теорема Больцано--Вейерштрасса. Принцип Коши.
    \item пункт 31 (частично). Ортогональные системы функций. Неравенство Бесселя. Равенство Парсеваля.
\end{itemize}

\subsubsection*{Минимальные знания}

\begin{itemize}
    \item Предел (по Коши и по Гейне), непрерывность, свойства непрерывных функций на отрезке (ограниченность, достижение максимумов/минимумов, теорема о промежуточном значении).
    \item Производная, связь с монотонностью, экстремумами, правила дифференцирования.
    \item Необходимые и достаточные условия экстремума одной и нескольких переменных.
    \item Определённый интеграл Римана, интегрируемость непрерывных и монотонных функций, формула Ньютона--Лейбница.
    \item Классические признаки сходимости числовых рядов (необходимый признак, Даламбер, Коши, интегральный).
    \item Понятия абсолютной и условной сходимости, перестановки членов ряда.
    \item Определение функционального и степенного ряда, радиус сходимости, базовые свойства.
    \item Определение равномерной сходимости, связь с непрерывностью, интегрированием и дифференцированием.
    \item Несобственные интегралы (по бесконечному промежутку и по особенностям подынтегральной функции).
    \item Базовые факты о рядах Фурье: формулы коэффициентов, ортогональность тригонометрических функций, формулировка теорем Фурье/Парсеваля.
    \item Метрические пространства: определение метрики, сходимость, замкнутые и открытые множества, компактность, полные пространства, критерий Коши.
\end{itemize}

\subsubsection*{Смежные темы}

\begin{itemize}
    \item Связь рядов Фурье с гильбертовыми пространствами $L^2$ (ФА).
    \item Связь равномерной сходимости с метриками на пространствах функций.
    \item Связь степенных рядов с ТФКП (расклады аналитических функций).
\end{itemize}

\subsubsection*{Основные источники}

\begin{itemize}
    \item Г.М.~Фихтенгольц, ``Курс дифференциального и интегрального исчисления'', т.~1--3.
    \item Л.Д.~Кудрявцев, ``Курс математического анализа'', т.~1--3.
    \item С.М.~Никольский, ``Курс математического анализа''.
    \item (Более современно) Т.~Тао, ``Введение в математический анализ'' (покрывает существенную часть пунктов 1--11).
\end{itemize}

\subsubsection*{Задачники}

\begin{itemize}
    \item Б.П.~Демидович, ``Сборник задач и упражнений по математическому анализу''.
    \item В.А.~Зорич, ``Математический анализ'' (упражнения).
\end{itemize}

\bigskip

\subsection{ФА/теория меры: интеграл Лебега и гильбертовы пространства}

\subsubsection*{Соответствующие пункты программы}

\begin{itemize}
    \item Пункты: 12--13.
    \item 12. Функции с ограниченным изменением. Мера в смысле Лебега. Теорема Д.Ф.~Егорова, $C$-свойства. Абсолютно непрерывные функции.
    \item 13. Суммируемые функции. Интеграл Лебега и его основные свойства. Гильбертово пространство. Пространства $L^2$ и $l^2$. Сходимость в среднем.
\end{itemize}

\subsubsection*{Минимальные знания}

\begin{itemize}
    \item Понимание конструкции меры Лебега на $\mathbb{R}$ (внешняя мера, измеримые множества, борелевская $\sigma$-алгебра).
    \item Определение измеримой функции.
    \item Определение интеграла Лебега для неотрицательных и для абсолютно интегрируемых функций.
    \item Основные теоремы: монотонная сходимость, теорема Фату, теорема Лебега о мажорируемой сходимости (формулировки).
    \item Понятие функций с ограниченной вариацией и абсолютно непрерывных функций (на уровне формулировок).
    \item Определение гильбертова пространства, примеры $L^2[a,b]$ и $l^2$.
    \item Понятие сходимости в среднем (сходимость в метрике $L^2$).
\end{itemize}

\subsubsection*{Смежные темы}

\begin{itemize}
    \item Связь интеграла Лебега с интегралом Римана: условия совпадения.
    \item Связь рядов Фурье с пространством $L^2$ и ортогональными системами.
    \item Основы теории вероятностей на базе меры Лебега (если спрашивают про тервер).
\end{itemize}

\subsubsection*{Основные источники}

\begin{itemize}
    \item А.Н.~Колмогоров, С.В.~Фомин, ``Элементы теории функций и функционального анализа''.
    \item А.Н.~Ширяев, ``Вероятность'', т.~1 (разделы о мере и интеграле Лебега).
    \item В.А.~Треногин, ``Функциональный анализ'' (более продвинутый уровень).
\end{itemize}

\subsubsection*{Задачники}

\begin{itemize}
    \item Упражнения в книге Колмогорова--Фомина.
    \item Специализированные задачники по теории меры (по необходимости).
\end{itemize}

\bigskip

\subsection{АГ: аналитическая геометрия}

\subsubsection*{Соответствующие пункты программы}

\begin{itemize}
    \item Пункты: 14--15.
    \item 14. Плоскости и прямые в пространстве. Различные виды уравнений. Взаимное расположение прямых и плоскостей. Метрические приложения.
    \item 15. Кривые и поверхности второго порядка. Канонические уравнения. Приведение к каноническому виду.
\end{itemize}

\subsubsection*{Минимальные знания}

\begin{itemize}
    \item Уравнение прямой и плоскости в $\mathbb{R}^3$ в различных формах (общий вид, нормальный вид).
    \item Взаимное расположение прямых и плоскостей, углы, расстояния.
    \item Канонические уравнения кривых второго порядка (окружность, эллипс, парабола, гипербола).
    \item Канонические уравнения поверхностей второго порядка (эллипсоид, одно- и двуполостный гиперболоид, параболоиды и т.д.).
\end{itemize}

\subsubsection*{Смежные темы}

\begin{itemize}
    \item Линейные преобразования и диагонализация квадратичных форм (связь с ЛА).
    \item Метрические свойства через скалярное произведение.
\end{itemize}

\subsubsection*{Основные источники}

\begin{itemize}
    \item М.М.~Постников, ``Аналитическая геометрия''.
    \item А.П.~Веселов, Е.В.~Троицкий, ``Лекции по аналитической геометрии''.
\end{itemize}

\subsubsection*{Задачники}

\begin{itemize}
    \item Упражнения в книгах Постникова и Веселова--Троицкого.
\end{itemize}

\bigskip

\subsection{ЛА: линейная алгебра}

\subsubsection*{Соответствующие пункты программы}

\begin{itemize}
    \item Пункты: 16--18.
    \item 16. Системы линейных алгебраических уравнений. Методы решения. Теорема о структуре общего решения однородной и неоднородной систем. Фундаментальная система решений.
    \item 17. Собственные векторы и собственные значения матриц. Характеристический многочлен. Теорема Гамильтона--Кэли.
    \item 18. Билинейные и квадратичные формы. Изменение матрицы при смене базиса. Канонический и нормальный вид. Закон инерции.
\end{itemize}

\subsubsection*{Минимальные знания}

\begin{itemize}
    \item Матрицы, операции, ранг, определитель.
    \item Решение СЛАУ (метод Гаусса, условия существования/единственности).
    \item Понятие ядра и образа линейного оператора.
    \item Собственные значения и собственные векторы, характеристический многочлен.
    \item Формулировка и смысл теоремы Гамильтона--Кэли.
    \item Определение билинейной и квадратичной формы, матрицы формы.
    \item Приведение квадратичной формы к каноническому виду, закон инерции.
\end{itemize}

\subsubsection*{Смежные темы}

\begin{itemize}
    \item Связь квадратичных форм с кривыми и поверхностями второго порядка (АГ).
    \item Линейные операторы в гильбертовых пространствах (ФА, при обсуждении $L^2$).
\end{itemize}

\subsubsection*{Основные источники}

\begin{itemize}
    \item И.М.~Гельфанд, ``Лекции по линейной алгебре''.
    \item А.И.~Кострикин, ``Введение в алгебру''.
    \item А.Г.~Курош, ``Курс высшей алгебры''.
\end{itemize}

\subsubsection*{Задачники}

\begin{itemize}
    \item Задачи в книгах Гельфанда и Кострикина.
\end{itemize}

\bigskip

\subsection{ОДУ: обыкновенные дифференциальные уравнения}

\subsubsection*{Соответствующие пункты программы}

\begin{itemize}
    \item Пункты: 19--21.
    \item 19. ОДУ первого порядка. Теорема Коши о существовании и единственности решения.
    \item 20. Линейное ОДУ второго порядка. Однородное уравнение. Линейная зависимость функций. Фундаментальная система решений.
    \item 21. Системы ОДУ. Определитель Вронского.
\end{itemize}

\subsubsection*{Минимальные знания}

\begin{itemize}
    \item Базовые методы решения ОДУ первого порядка (разделяющиеся, линейные, уравнения с разделяющимися переменными).
    \item Формулировка теоремы Пикара--Линдёлефа (Коши) о существовании и единственности.
    \item Решение линейных ОДУ второго порядка с постоянными коэффициентами.
    \item Понятие фундаментальной системы решений, общее решение.
    \item Определитель Вронского и критерий линейной независимости решений.
\end{itemize}

\subsubsection*{Смежные темы}

\begin{itemize}
    \item Связь с рядом Тейлора (построение решений в виде степенных рядов).
    \item Численные методы решения ОДУ (связь с разделом ВМ).
\end{itemize}

\subsubsection*{Основные источники}

\begin{itemize}
    \item Л.С.~Понтрягин, ``Обыкновенные дифференциальные уравнения''.
    \item В.В.~Степанов, ``Курс дифференциальных уравнений''.
\end{itemize}

\subsubsection*{Задачники}

\begin{itemize}
    \item Задачники по ОДУ (Демидович и др.).
\end{itemize}

\bigskip

\subsection{ЧПУ и пространства Соболева}

\subsubsection*{Соответствующие пункты программы}

\begin{itemize}
    \item Пункты: 22--24.
    \item 22. Уравнения с частными производными. Линейные уравнения второго порядка. Классификация.
    \item 23. Пространства Соболева. Теорема вложения.
    \item 24. Однозначная и фредгольмова разрешимость эллиптических задач. Задача на собственные функции и значения. Гладкость обобщённых решений.
\end{itemize}

\subsubsection*{Минимальные знания}

\begin{itemize}
    \item Базовая классификация PDE второго порядка (эллиптические, параболические, гиперболические) по знаку дискриминанта.
    \item Общее представление о пространстве Соболева $W^{k,p}$, особенно $H^1$, $H^2$.
    \item Идея слабых решений и фредгольмовой постановки задач.
    \item Формулировка теорем вложения (на уровне понимания).
\end{itemize}

\subsubsection*{Смежные темы}

\begin{itemize}
    \item Связь с функциональным анализом (операторы в гильбертовых пространствах).
    \item Связь с вариационным исчислением (задачи на минимум функционалов).
\end{itemize}

\subsubsection*{Основные источники}

\begin{itemize}
    \item В.А.~Треногин, ``Функциональный анализ''.
    \item Классические лекции по ЧПУ (Жидков, Самарский, Ладыженская и др.).
\end{itemize}

\bigskip

\subsection{КА: теория функций комплексного переменного (ТФКП)}

\subsubsection*{Соответствующие пункты программы}

\begin{itemize}
    \item Пункты: 25--27.
    \item 25. Функции комплексного переменного. Дифференцируемость. Условия Коши--Римана. Геометрический смысл аргумента и модуля производной.
    \item 26. Теорема Коши об интеграле по замкнутому контуру. Интеграл Коши.
    \item 27. Степенные ряды с комплексными членами. Ряд Лорана. Особые точки. Вычеты.
\end{itemize}

\subsubsection*{Минимум по ТФКП, который нужно знать}

\begin{itemize}
    \item Комплексная плоскость, аналитическая функция, голоморфность.
    \item Условия Коши--Римана (формулировка и интерпретация).
    \item Понятие конформного отображения и геометрический смысл производной комплексной функции (масштаб и поворот).
    \item Интеграл по контуру, формулировка теоремы Коши об интеграле по замкнутому контуру.
    \item Формула Коши для значения функции и её производных.
    \item Степенные ряды комплексных функций, диски сходимости.
    \item Ряд Лорана, выделение главной части.
    \item Классификация особых точек: устранимая, полюс, существенная.
    \item Определение вычета, базовые формулы для вычисления вычетов.
    \item Применение вычетов к вычислению определённых и несобственных интегралов (на уровне примеров).
\end{itemize}

\subsubsection*{Смежные темы}

\begin{itemize}
    \item Связь степенных рядов с вещественным анализом (разложения элементарных функций).
    \item Связь с PDE (уравнение Лапласа и гармонические функции).
\end{itemize}

\subsubsection*{Основные источники}

\begin{itemize}
    \item Б.В.~Шабат, ``Введение в комплексный анализ'', ч.~1.
    \item И.Н.~Привалов, ``Введение в теорию функций комплексного переменного''.
\end{itemize}

\subsubsection*{Задачники}

\begin{itemize}
    \item Упражнения в книге Шабата.
\end{itemize}

\bigskip

\subsection{ВИ: вариационное исчисление и механика}

\subsubsection*{Соответствующие пункты программы}

\begin{itemize}
    \item Пункты: 28--30.
    \item 28. Простейшая проблема вариационного исчисления. Уравнение Эйлера--Лагранжа. Геодезические линии.
    \item 29. Задача о движении механической системы при наличии связей. Классификация связей и перемещений.
    \item 30. Принцип наименьшего действия.
\end{itemize}

\subsubsection*{Минимальные знания}

\begin{itemize}
    \item Постановка классической задачи вариационного исчисления для функционалов вида
    \[
        J[y] = \int\limits_a^b L(x, y, y') \, dx,
    \]
    вывод уравнения Эйлера--Лагранжа.
    \item Примеры геодезических линий (прямая на плоскости, дуга большой окружности на сфере).
    \item Общее представление о связях в механике и принципе наименьшего действия.
\end{itemize}

\subsubsection*{Смежные темы}

\begin{itemize}
    \item Связь с пространствами Соболева и слабыми решениями.
    \item Связь с ОДУ (уравнения Эйлера--Лагранжа как система ОДУ).
\end{itemize}

\subsubsection*{Основные источники}

\begin{itemize}
    \item И.М.~Гельфанд, С.В.~Фомин, ``Вариационное исчисление''.
    \item В.М.~Алексеев, В.М.~Тихомиров, С.В.~Фомин, ``Оптимальное управление'' (для расширения кругозора).
\end{itemize}

\bigskip

\subsection{ВМ: вычислительная математика и ортогональные системы}

\subsubsection*{Соответствующие пункты программы}

\begin{itemize}
    \item Пункты: 31--33.
    \item 31. Ортогональные системы функций. Метод ортогонализации Шмидта. Неравенство Бесселя. Равенство Парсеваля.
    \item 32. Интерполяционные формулы Ньютона и Лагранжа. Многочлены Чебышева, их свойства.
    \item 33. Численное решение задачи Коши для ОДУ: метод Эйлера, методы второго порядка, метод Рунге--Кутты.
\end{itemize}

\subsubsection*{Минимальные знания}

\begin{itemize}
    \item Определение ортогональной системы функций в $L^2$, смысл ортогонализации по Граму--Шмидту.
    \item Формулировки неравенства Бесселя и равенства Парсеваля.
    \item Интерполяционные многочлены Ньютона и Лагранжа, смысл узлов интерполяции.
    \item Базовые свойства многочленов Чебышева (ортогональность, минимаксное свойство).
    \item Схема метода Эйлера, базовых методов второго порядка и метода Рунге--Кутты для решения задачи Коши.
\end{itemize}

\subsubsection*{Смежные темы}

\begin{itemize}
    \item Связь ортогональных систем с рядами Фурье и пространствами $L^2$.
    \item Связь численных методов ОДУ с теоремой существования и единственности (точность и устойчивость).
\end{itemize}

\subsubsection*{Основные источники}

\begin{itemize}
    \item Н.С.~Бахвалов, Н.П.~Жидков, ``Численные методы''.
    \item А.А.~Самарский, ``Численные методы'' (и другие его книги по ЧПУ).
\end{itemize}

\subsubsection*{Задачники}

\begin{itemize}
    \item Задачи в книгах Бахвалова и Жидкова.
\end{itemize}

\bigskip

\section{Рекомендуемые источники в разрезе разделов}

\begin{itemize}
    \item МА (1--11): Фихтенгольц, Кудрявцев, Никольский, Тао.
    \item Теория меры и ФА (12--13): Колмогоров--Фомин, Ширяев (т.~1).
    \item АГ (14--15): Постников, Веселов--Троицкий.
    \item ЛА (16--18): Гельфанд, Кострикин, Курош.
    \item ОДУ (19--21): Понтрягин, Степанов.
    \item ЧПУ и Соболев (22--24): Треногин, классические конспекты по ЧПУ.
    \item КА / ТФКП (25--27): Шабат, Привалов.
    \item ВИ (28--30): Гельфанд--Фомин, Алексеев--Тихомиров--Фомин.
    \item ВМ (31--33): Бахвалов--Жидков, Самарский.
\end{itemize}

\section{Как брать информацию для задач и как лучше решать}

\subsection*{Источник задач}

\begin{itemize}
    \item Для математического анализа: Демидович, задачи у Фихтенгольца и Зорича.
    \item Для линейной алгебры и аналитической геометрии: задачи у Гельфанда, Кострикина, Постникова.
    \item Для ОДУ: задачники Понтрягина, Демидовича.
    \item Для комплексного анализа: упражнения у Шабата.
    \item Для теории меры и ФА: упражнения у Колмогорова--Фомина.
    \item Для вариационного исчисления: задачи у Гельфанда--Фомина.
    \item Для вычислительных методов: задачи у Бахвалова--Жидкова.
\end{itemize}

\subsection*{Рекомендуемая методика решения}

\begin{enumerate}
    \item Для каждого пункта программы (1--33) составить краткий конспект:
    \begin{itemize}
        \item определения;
        \item 2--3 ключевые теоремы;
        \item 1--2 простых примера;
        \item 1 контрпример (если уместно).
    \end{itemize}
    \item По каждому разделу прорешать 10--20 типовых задач:
    \begin{itemize}
        \item на сходимость рядов;
        \item на пределы и непрерывность;
        \item на простые ОДУ;
        \item на вычеты и контурные интегралы;
        \item на собственные значения и квадратичные формы и т.п.
    \end{itemize}
    \item После решения задач обязательно проговаривать устно:
    \begin{itemize}
        \item формулировку задачи;
        \item метод решения;
        \item ключевую идею, которая используется (например, какой признак сходимости применён).
    \end{itemize}
    \item Смоделировать формат экзамена:
    \begin{itemize}
        \item выбрать 2 пункта из списка 1--33;
        \item за 10--15 минут подготовить план устного ответа с определениями, теоремами и одним примером;
        \item проговорить ответ вслух, как на экзамене.
    \end{itemize}
\end{enumerate}

Такой подход позволяет минимально, но системно закрыть все разделы программы и выйти на уровень, достаточный для набора $\ge 30$ баллов на вступительном экзамене.

\end{document}