\begin{center}
\specialsection{Введение}
\end{center}

{Развитие финансовых рынков и повышение волатильности макроэкономической среды усиливают роль управления рисками портфельного инвестирования. Наиболее значимыми источниками неопределённости остаются динамика процентных ставок и структурные изменения в денежно-кредитной политике, влияние которых особенно заметно на российском рынке в 2014–2025 гг. В эти годы ключевая ставка Банка России изменялась скачкообразно, что приводило к резким переоценкам долговых инструментов и усиливало необходимость количественной оценки процентного риска.}

{Ключевое место в российской финансовой системе занимают инструменты долгового рынка — государственные и корпоративные облигации. Они характеризуются высокой ликвидностью, регулярностью денежных потоков и значительной чувствительностью к изменениям процентных ставок. В отличие от акций, стоимость облигаций определяется дисконтированием будущих платежей, вследствие чего процентный риск становится фундаментальным фактором их динамики. Соответственно, в портфельных конструкциях процентный риск выступает доминирующим источником неопределённости.}

{Несмотря на наличие развитых аналитических подходов, практическая задача количественной оценки процентного риска облигационных портфелей остаётся сложной. Классические методы дюрации и выпуклости обеспечивают лишь локальные приближения и не отражают нелинейный характер реакции стоимости облигаций на крупные изменения ставок. Стохастические модели процентных ставок допускают формальное описание динамики кривой доходности, однако требуют строгих предположений и недостаточно гибки в условиях резких процентных шоков, типичных для российского рынка.}

{Современные методы оценки процентного риска опираются на моделирование временных рядов доходностей и волатильности, а также на факторные модели структуры процентных ставок. Эконометрические модели позволяют описывать поведение доходностей облигаций, учитывать кластеризацию волатильности и формировать вероятностные прогнозы, необходимые для вычисления Value at Risk и Conditional Value at Risk. Их сочетание с параметрическими моделями кривой доходности и факторизацией динамики ставок методом главных компонент формирует комплексный прикладной инструментарий для анализа процентного риска.}

{Объектом исследования является портфель облигаций, включающий государственные и корпоративные долговые инструменты.}

{Предмет исследования — процентный риск портфельного инвестирования и его количественная оценка на основе математических и статистических моделей.}

{Цель исследования заключается в разработке методологии оценки и математического моделирования процентного риска облигационного портфеля, опирающейся на анализ временных рядов доходностей, факторную структуру кривой доходности и риск-метрики VaR и CVaR.}

Для достижения цели работы решаются следующие задачи:
\vspace{-3.5mm}
\begin{enumerate}
\setlength{\itemsep}{1pt}
\setlength{\parskip}{0pt}
\setlength{\parsep}{0pt}
\item проанализировать природу процентного риска и его влияние на стоимость облигаций;
\item исследовать методы анализа временных рядов доходностей;
\item рассмотреть структуру процентных ставок и модели построения кривой доходности;
\item выполнить факторизацию динамики ставок методом главных компонент;
\item построить модели доходностей и оценить вероятностные распределения будущих изменений ставок;
\item рассчитать VaR и CVaR различными методами (исторический, параметрический, GARCH-VaR, Monte-Carlo);
\item провести стресс-тестирование процентного риска, включая факторные и исторические сценарии;
\item сравнить риск-профили портфелей государственных и корпоративных облигаций.
\end{enumerate}
\vspace{-3.5mm}

{Структура работы соответствует поставленным задачам. В первой главе рассматриваются теоретические основы риска портфельного инвестирования, природа процентного риска, свойства облигаций и методы анализа временных рядов доходностей. Во второй главе разрабатывается методология количественной оценки процентного риска, включая построение кривой доходности, факторизацию ставок и моделирование динамики доходностей. В третьей главе проводится эмпирическое исследование: формирование портфелей, оценка параметров моделей, расчёт VaR и CVaR и анализ стресс-сценариев.}
\clearpage