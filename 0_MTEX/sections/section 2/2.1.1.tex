\subsubsection{Источник данных и выбор облигаций}

Для проведения исследования используются данные по государственным и корпоративным облигациям, обращающимся на российском долговом рынке. Основным источником информации являются котировки Московской биржи, предоставляющие ежедневные значения цен, доходностей к погашению, параметров купонов и оборотов торгов. В качестве дополнительного источника информации использовались данные информационного агентства Cbonds, позволяющие уточнять характеристики выпусков и параметры денежных потоков.

Период анализа охватывает 2016–2025 гг. и включает несколько фаз рыночного цикла: период повышенной ключевой ставки 2016–2017 гг., фазу снижения ставок 2019–2020 гг., период повышенной волатильности 2022 г., а также восстановление долгового рынка в 2023–2025 гг. Такой диапазон обеспечивает репрезентативную выборку процентных шоков различной природы, что важно для корректной оценки процентного риска.

Для формирования выборки облигаций применяются следующие критерии:
\vspace{-3.5mm}
\begin{enumerate}
    \setlength{\itemsep}{1pt}
    \setlength{\parskip}{0pt}
    \setlength{\parsep}{0pt}
    \item эмитент — государственные облигации федерального займа (ОФЗ) и корпоративные облигации первого эшелона;
    \item валюта номинации — рубль;
    \item наличие регулярных купонных выплат;
    \item достаточная ликвидность (наличие ежедневных котировок);
    \item отсутствие технических дефолтов и длительных остановок торгов.
\end{enumerate}
\vspace{-3.5mm}
На основе указанных критериев формируются три портфеля, различающиеся уровнем процентного и кредитного риска:
\vspace{-3.5mm}
\begin{enumerate}
    \setlength{\itemsep}{1pt}
    \setlength{\parskip}{0pt}
    \setlength{\parsep}{0pt}
    \item портфель государственных облигаций (ОФЗ), характеризующийся минимальным кредитным риском и высокой чувствительностью к динамике ключевой ставки;
    \item портфель корпоративных облигаций крупных эмитентов, отражающий сочетание процентного и кредитного риска и позволяющий анализировать влияние кредитного спреда;
    \item смешанный портфель, включающий как государственные, так и корпоративные выпуски и представляющий типичную структуру портфеля умеренного риска.
\end{enumerate}
\vspace{-3.5mm}

Выбор нескольких выпусков различной дюрации внутри каждого портфеля обеспечивает корректное измерение процентного риска: длинные облигации отражают чувствительность к изменениям долгосрочных ставок, тогда как короткие — реакцию на решения центрального банка по денежно-кредитной политике.

Для всех отобранных облигаций формируются временные ряды цен, доходностей к погашению и логарифмических доходностей. Эти данные используются на последующих этапах анализа для построения моделей структуры процентных ставок, оценки параметров волатильности, расчёта риск-метрик и проведения стресс-тестирования.

