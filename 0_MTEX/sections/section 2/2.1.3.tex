\subsubsection{Проверка и обработка временных рядов}

Перед применением эконометрических моделей и методов оценки риска требуется выполнить предварительную обработку временных рядов доходностей и доходностей к погашению. Данные долгового рынка могут содержать пропуски, нерегулярные наблюдения, а также разрывы, связанные с переходом через купонные даты и изменением ликвидности отдельных выпусков.

Первым шагом является синхронизация рядов по датам торгов. Облигации с низкой ликвидностью могут иметь неполные временные ряды, поэтому такие выпуски либо исключаются из анализа, либо их данные заполняются методом прямой форвардной интерполяции при условии, что пропуски краткосрочные и не влияют на динамику доходностей.

Вторым этапом выполняется идентификация и обработка выбросов. Резкие скачки цен, не подтверждённые объёмами торгов или рыночными событиями, классифицируются как технические выбросы и подлежат корректировке. Для их выявления применяется правило трёх стандартных отклонений или анализ межквартильного размаха.

Третьим этапом является корректировка данных после купонных выплат. В день отсечки рыночная цена облигации снижается на величину накопленного купонного дохода, что приводит к искусственному отрицательному скачку цен. Для предотвращения искажения доходностей такие переходы корректируются путём добавления накопленного купона к цене предыдущего дня.

После очистки выполняется проверка стационарности временных рядов. Для модели ARIMA требуется, чтобы логарифмические доходности имели стабильное распределение во времени. Проверка осуществляется с помощью тестов:
\vspace{-3.5mm}
\begin{enumerate}
    \setlength{\itemsep}{1pt}
    \setlength{\parskip}{0pt}
    \setlength{\parsep}{0pt}
    \item Расширенного теста Дики–Фуллера (ADF);
    \item Теста Квятковского–Филлипса–Шмидта–Шина (KPSS).
\end{enumerate}
\vspace{-3.5mm}

В случае наличия тренда или изменяющейся дисперсии применяются преобразования дифференцирования или удаления deterministic trend. Итогом обработки является набор корректных стационарных временных рядов, пригодных для дальнейшего анализа волатильности, построения моделей ARIMA и GARCH, а также для расчёта риск-метрик VaR и ES на основе прогнозных распределений доходностей.