\subsubsection{Преобразование цен в доходности}

Для моделирования процентного риска необходимо перейти от исходных данных о ценах облигаций к рядам доходностей. Цена отдельного выпуска отражает совокупное влияние нескольких факторов — текущего уровня процентных ставок, кредитного качества эмитента, ликвидности и ожиданий рынка. Однако для количественного анализа волатильности, построения эконометрических моделей и расчёта риск-метрик предпочтительно использовать именно доходности, так как они обладают более стабильными свойствами и допускают корректное статистическое моделирование.

В исследовании используются два вида доходностей:
\vspace{-3.5mm}
\begin{itemize}
    \setlength{\itemsep}{1pt}
    \setlength{\parskip}{0pt}
    \setlength{\parsep}{0pt}
    \item доходность к погашению (yield to maturity, YTM), характеризующая внутреннюю ставку дисконтирования будущих денежных потоков облигации;
    \item логарифмические доходности, рассчитываемые на основе динамики рыночных цен.
\end{itemize}\vspace{-3.5mm}

Доходность к погашению применяется для анализа структуры процентных ставок и построения кривых доходности, поскольку она обеспечивает сопоставимость различных выпусков по единой шкале ставок. Для каждого дня наблюдения YTM рассчитывается как решение уравнения стоимости облигации:
$$
P_t = \sum_{i=1}^{N} \frac{C_i}{(1+y_t)^{\tau_i}} + \frac{N}{(1+y_t)^{T-t}},
$$
где $P_t$ — рыночная цена, $C_i$ — значения купонных выплат, $\tau_i$ — интервалы до купонных дат, $N$ — номинал, $T$ — дата погашения.

Для анализа волатильности и моделирования временных рядов используются логарифмические доходности:
$$
r_t = \ln\left( \frac{P_t}{P_{t-1}} \right).
$$
Логдоходности обладают удобными статистическими свойствами: они аддитивны по времени и ближе к нормальному распределению на коротких горизонтах. Кроме того, использование логарифмов снижает влияние редких, но сильных колебаний цен, что положительно сказывается на устойчивости моделей ARIMA и GARCH.

Перед расчётом доходностей выполняется предварительная очистка данных: исключаются дни отсутствия торгов, корректируются выбросы, связанные с техническими сбоями, а также устраняются разрывы, возникающие при переходе облигаций через купонные даты. Это обеспечивает корректность будущего анализа и стабильность временных рядов.