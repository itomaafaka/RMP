\section{ Методология количественной оценки процентного риска облигационного портфеля}

% 2.1. Данные и предварительная подготовка
\subsection{Данные и предварительная подготовка}
%     2.1.1. Источник данных и выбор облигаций
\subsubsection{Источник данных и выбор облигаций}

Для проведения исследования используются данные по государственным и корпоративным облигациям, обращающимся на российском долговом рынке. Основным источником информации являются котировки Московской биржи, предоставляющие ежедневные значения цен, доходностей к погашению, параметров купонов и оборотов торгов. В качестве дополнительного источника информации использовались данные информационного агентства Cbonds, позволяющие уточнять характеристики выпусков и параметры денежных потоков.

Период анализа охватывает 2016–2025 гг. и включает несколько фаз рыночного цикла: период повышенной ключевой ставки 2016–2017 гг., фазу снижения ставок 2019–2020 гг., период повышенной волатильности 2022 г., а также восстановление долгового рынка в 2023–2025 гг. Такой диапазон обеспечивает репрезентативную выборку процентных шоков различной природы, что важно для корректной оценки процентного риска.

Для формирования выборки облигаций применяются следующие критерии:
\vspace{-3.5mm}
\begin{enumerate}
    \setlength{\itemsep}{1pt}
    \setlength{\parskip}{0pt}
    \setlength{\parsep}{0pt}
    \item эмитент — государственные облигации федерального займа (ОФЗ) и корпоративные облигации первого эшелона;
    \item валюта номинации — рубль;
    \item наличие регулярных купонных выплат;
    \item достаточная ликвидность (наличие ежедневных котировок);
    \item отсутствие технических дефолтов и длительных остановок торгов.
\end{enumerate}
\vspace{-3.5mm}
На основе указанных критериев формируются три портфеля, различающиеся уровнем процентного и кредитного риска:
\vspace{-3.5mm}
\begin{enumerate}
    \setlength{\itemsep}{1pt}
    \setlength{\parskip}{0pt}
    \setlength{\parsep}{0pt}
    \item портфель государственных облигаций (ОФЗ), характеризующийся минимальным кредитным риском и высокой чувствительностью к динамике ключевой ставки;
    \item портфель корпоративных облигаций крупных эмитентов, отражающий сочетание процентного и кредитного риска и позволяющий анализировать влияние кредитного спреда;
    \item смешанный портфель, включающий как государственные, так и корпоративные выпуски и представляющий типичную структуру портфеля умеренного риска.
\end{enumerate}
\vspace{-3.5mm}

Выбор нескольких выпусков различной дюрации внутри каждого портфеля обеспечивает корректное измерение процентного риска: длинные облигации отражают чувствительность к изменениям долгосрочных ставок, тогда как короткие — реакцию на решения центрального банка по денежно-кредитной политике.

Для всех отобранных облигаций формируются временные ряды цен, доходностей к погашению и логарифмических доходностей. Эти данные используются на последующих этапах анализа для построения моделей структуры процентных ставок, оценки параметров волатильности, расчёта риск-метрик и проведения стресс-тестирования.


%     2.1.2. Преобразование цен в доходности
\subsubsection{Преобразование цен в доходности}

Для моделирования процентного риска необходимо перейти от исходных данных о ценах облигаций к рядам доходностей. Цена отдельного выпуска отражает совокупное влияние нескольких факторов — текущего уровня процентных ставок, кредитного качества эмитента, ликвидности и ожиданий рынка. Однако для количественного анализа волатильности, построения эконометрических моделей и расчёта риск-метрик предпочтительно использовать именно доходности, так как они обладают более стабильными свойствами и допускают корректное статистическое моделирование.

В исследовании используются два вида доходностей:
\vspace{-3.5mm}
\begin{itemize}
    \setlength{\itemsep}{1pt}
    \setlength{\parskip}{0pt}
    \setlength{\parsep}{0pt}
    \item доходность к погашению (yield to maturity, YTM), характеризующая внутреннюю ставку дисконтирования будущих денежных потоков облигации;
    \item логарифмические доходности, рассчитываемые на основе динамики рыночных цен.
\end{itemize}\vspace{-3.5mm}

Доходность к погашению применяется для анализа структуры процентных ставок и построения кривых доходности, поскольку она обеспечивает сопоставимость различных выпусков по единой шкале ставок. Для каждого дня наблюдения YTM рассчитывается как решение уравнения стоимости облигации:
$$
P_t = \sum_{i=1}^{N} \frac{C_i}{(1+y_t)^{\tau_i}} + \frac{N}{(1+y_t)^{T-t}},
$$
где $P_t$ — рыночная цена, $C_i$ — значения купонных выплат, $\tau_i$ — интервалы до купонных дат, $N$ — номинал, $T$ — дата погашения.

Для анализа волатильности и моделирования временных рядов используются логарифмические доходности:
$$
r_t = \ln\left( \frac{P_t}{P_{t-1}} \right).
$$
Логдоходности обладают удобными статистическими свойствами: они аддитивны по времени и ближе к нормальному распределению на коротких горизонтах. Кроме того, использование логарифмов снижает влияние редких, но сильных колебаний цен, что положительно сказывается на устойчивости моделей ARIMA и GARCH.

Перед расчётом доходностей выполняется предварительная очистка данных: исключаются дни отсутствия торгов, корректируются выбросы, связанные с техническими сбоями, а также устраняются разрывы, возникающие при переходе облигаций через купонные даты. Это обеспечивает корректность будущего анализа и стабильность временных рядов.
%     2.1.3. Проверка и обработка временных рядов
\subsubsection{Проверка и обработка временных рядов}

Перед применением эконометрических моделей и методов оценки риска требуется выполнить предварительную обработку временных рядов доходностей и доходностей к погашению. Данные долгового рынка могут содержать пропуски, нерегулярные наблюдения, а также разрывы, связанные с переходом через купонные даты и изменением ликвидности отдельных выпусков.

Первым шагом является синхронизация рядов по датам торгов. Облигации с низкой ликвидностью могут иметь неполные временные ряды, поэтому такие выпуски либо исключаются из анализа, либо их данные заполняются методом прямой форвардной интерполяции при условии, что пропуски краткосрочные и не влияют на динамику доходностей.

Вторым этапом выполняется идентификация и обработка выбросов. Резкие скачки цен, не подтверждённые объёмами торгов или рыночными событиями, классифицируются как технические выбросы и подлежат корректировке. Для их выявления применяется правило трёх стандартных отклонений или анализ межквартильного размаха.

Третьим этапом является корректировка данных после купонных выплат. В день отсечки рыночная цена облигации снижается на величину накопленного купонного дохода, что приводит к искусственному отрицательному скачку цен. Для предотвращения искажения доходностей такие переходы корректируются путём добавления накопленного купона к цене предыдущего дня.

После очистки выполняется проверка стационарности временных рядов. Для модели ARIMA требуется, чтобы логарифмические доходности имели стабильное распределение во времени. Проверка осуществляется с помощью тестов:
\vspace{-3.5mm}
\begin{enumerate}
    \setlength{\itemsep}{1pt}
    \setlength{\parskip}{0pt}
    \setlength{\parsep}{0pt}
    \item Расширенного теста Дики–Фуллера (ADF);
    \item Теста Квятковского–Филлипса–Шмидта–Шина (KPSS).
\end{enumerate}
\vspace{-3.5mm}

В случае наличия тренда или изменяющейся дисперсии применяются преобразования дифференцирования или удаления deterministic trend. Итогом обработки является набор корректных стационарных временных рядов, пригодных для дальнейшего анализа волатильности, построения моделей ARIMA и GARCH, а также для расчёта риск-метрик VaR и ES на основе прогнозных распределений доходностей.

% 2.2. Моделирование структуры процентных ставок
\subsection{Моделирование структуры процентных ставок}
%     2.2.1. Параметрические модели кривой доходности (Nelson–Siegel, NSS)
\input{sections/section 2/2.2.1.tex}
%     2.2.2. Факторная декомпозиция ставок методом PCA
\input{sections/section 2/2.2.2.tex}
%     2.2.3. Интерпретация факторов (уровень, наклон, кривизна)
\input{sections/section 2/2.2.3.tex}

% % 2.3. Эконометрическое моделирование доходностей
% \subsection{Эконометрическое моделирование доходностей}
% %     2.3.1. Стационарность и анализ автокорреляций
% \input{sections/section 2/2.3.1.tex}
% %     2.3.2. Модели ARIMA и прогнозирование доходностей
% \input{sections/section 2/2.3.2.tex}
% %     2.3.3. Модели волатильности: GARCH и EGARCH
% \input{sections/section 2/2.3.3.tex}
% %     2.3.4. Прогноз распределения доходностей
% \input{sections/section 2/2.3.4.tex}

% 2.4. Оценка риск-метрик портфеля
%     2.4.1. Исторический VaR
%     2.4.2. Параметрический VaR (нормальный и t-распределение)
%     2.4.3. GARCH-VaR
%     2.4.4. Monte-Carlo VaR
%     2.4.5. Расчёт. CVaR и Expected Shortfall

% 2.5. Стресс-тестирование процентного риска
%     2.5.1. Факторные шоки кривой доходности (уровень/наклон/кривизна)
%     2.5.2. Сценарии из исторических кризисов (2014, 2020, 2022)
%     2.5.3. Дюрация/выпуклость как локальная аппроксимация

% 2.6. Сравнение портфелей и агрегированная оценка риска
%     2.6.1. Портфель из ОФЗ
%     2.6.2. Портфель корпоративных облигаций
%     2.6.3. Смешанный портфель
%     2.6.4. Сравнение VaR, CVaR, стресс-метрик
%     2.6.5. Выводы и интерпретации