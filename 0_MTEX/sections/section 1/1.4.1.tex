\subsubsection{Модифицированная дюрация и дюрация Маколея}

Дюрация в классическом понимании была введена Ф.~Маколеем как средневзвешенный срок получения денежных потоков. Если поток платежей облигации равен $\{C_t\}_{t=1}^T$, а $y$ обозначает доходность к погашению, то цена облигации выражается как
$$
P = \sum_{t=1}^{T} \frac{C_t}{(1+y)^t}.
$$
Тогда Дюрация Маколея определяется формулой
$$
D_M = \frac{1}{P} \sum_{t=1}^{T} t \cdot \frac{C_t}{(1+y)^t}.
$$
Она показывает момент времени, в среднем соответствующий получению стоимости денежного потока облигации. Однако в анализе процентного риска более удобной является модифицированная дюрация, определяемая как
$$
D = \frac{D_M}{1+y}.
$$
Модифицированная дюрация непосредственно характеризует относительное изменение цены облигации при малом изменении ставки:
$$
\frac{\Delta P}{P} \approx - D \cdot \Delta y.
$$
Знак «минус» отражает обратную зависимость между ценой и доходностью: при росте ставок цена облигации падает.

Дюрация зависит от структуры денежных потоков: чем длиннее срок до погашения и чем ниже купон, тем выше дюрация и чувствительность облигации к процентным изменениям. Поэтому дюрация является базовым показателем при сравнении процентного риска различных долговых инструментов.

