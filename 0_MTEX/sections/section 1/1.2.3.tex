\subsubsection{Особенности поведения доходностей различных облигаций}

Доходности облигаций различаются в зависимости от типа эмитента, срока до погашения и купонной структуры инструмента. Эти различия оказывают существенное влияние на характер процентного риска и определяют специфику моделирования. В рамках анализа портфельного риска важно учитывать неоднородность поведения доходностей государственных и корпоративных облигаций, а также инструментов с различной дюрацией.

Одним из ключевых факторов является срок до погашения. Короткие облигации демонстрируют меньшую чувствительность к изменениям процентных ставок, что отражается в более низкой волатильности доходностей. Длинные выпуски, напротив, обладают повышенной изменчивостью доходности вследствие большего влияния будущих процентных ставок на текущую цену. Это приводит к тому, что ряды доходностей длинных облигаций, как правило, характеризуются более выраженной нестабильностью и более сильной реакцией на макроэкономические изменения.

Корпоративные облигации отличаются от государственных не только кредитным риском, но и более высокой волатильностью доходностей. Наличие кредитного спрэда усиливает реакцию доходностей на рыночные и кредитные шоки, что приводит к увеличению вероятности экстремальных отклонений и к появлению асимметрии распределений. В сравнении с ОФЗ корпоративные облигации также могут демонстрировать более высокие автокорреляции доходностей, связанные с особенностями ликвидности и структуры рынка.

Дополнительным фактором, влияющим на динамику доходностей, является купонная структура. Высококупонные облигации, как правило, менее чувствительны к колебаниям процентных ставок, поскольку значительная часть их стоимости определяется ближайшими купонными выплатами. Напротив, низкокупонные и бессрочные инструменты сильнее реагируют на изменения ставок, что отражается в их более волатильных доходностях.

Таким образом, доходности различных типов облигаций обладают неодинаковыми временными и статистическими свойствами. Различия в волатильности, автокорреляциях, асимметрии распределений и реакции на процентные шоки имеют существенное значение для выбора методов моделирования и оценки процентного риска, что рассматривается в последующих разделах.
