\subsubsection{Основные статистические характеристики доходностей}

Статистические свойства доходностей облигаций играют ключевую роль в построении моделей процентного риска. Анализ доходностей позволяет выявить закономерности их поведения во времени, оценить устойчивость динамики и определить подходящие методы моделирования. 

Пусть $\{r_t\}_{t=1}^T$ обозначает последовательность доходностей за равные интервалы наблюдения. из базовых характеристик является математическое ожидание доходности, определяемое как:
$$
\mu = \mathbb{E}[r_t].
$$
В дискретном случае, когда доходность принимает значения $r_i$ с вероятностями $p_i$, математическое ожидание задаётся суммой
$$
\mathbb{E}[r_t] = \sum_{i=1}^k r_i p_i.
$$
Если распределение доходности обладает плотностью $p(r)$, то математическое ожидание определяется интегралом
$$
\mathbb{E}[r_t] = \int_{\Omega} rp(r)dr.
$$

Однако для анализа риска более важной величиной является изменчивость доходности, измеряемая дисперсией
$$
\sigma^2 = \mathrm{Var}(r_t) = \mathbb{E}\left[(r_t - \mu)^2\right],
$$
отражающей степень отклонения фактических значений доходности от среднего уровня. Стандартное отклонение \(\sigma\) традиционно используется в качестве показателя волатильности прибыли.

Зависимость текущих значений доходности от прошлых наблюдений оценивается через автокорреляционную функцию
\[
\rho(k) = \frac{\mathrm{Cov}(r_t, r_{t-k})}{\mathrm{Var}(r_t)},
\]
где \(k\) — лаг. Наличие автокорреляции сигнализирует о возможных временных зависимостях, что необходимо учитывать при построении моделей ARIMA и GARCH. У облигаций разных сроков обращения автокорреляционная структура может существенно различаться вследствие особенностей формирования цен и влияния динамики процентных ставок.

Кроме того, доходности облигаций нередко характеризуются отклонениями от нормального распределения. Такие особенности, как асимметрия распределения и повышенная вероятность экстремальных значений (тяжёлые хвосты), оцениваются через коэффициенты асимметрии и эксцесса:
\[
\mathrm{Skew}(r_t)=\frac{\mathbb{E}\left[(r_t-\mu)^3\right]}{\sigma^3}, \qquad
\mathrm{Kurt}(r_t)=\frac{\mathbb{E}\left[(r_t-\mu)^4\right]}{\sigma^4}.
\]
Положительная эксцессивность (\(\mathrm{Kurt}(r_t) > 3\)) указывает на наличие частых резких скачков, что характерно для доходностей долговых инструментов в периоды нестабильности процентных ставок.

Статистические свойства доходностей определяют выбор методов моделирования в последующих разделах. В частности, наличие автокорреляций и кластеризации волатильности требует применения моделей временных рядов (ARIMA, GARCH), а тяжёлые хвосты и асимметрия существенно влияют на формирование кривых распределений, используемых при расчёте мер риска, таких как Value at Risk и Conditional Value at Risk.