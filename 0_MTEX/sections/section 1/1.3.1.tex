\subsubsection{Понятие процентной ставки и её роль в оценке стоимости облигаций}

Процентная ставка является базовым параметром финансового рынка и отражает стоимость денег во времени. В наиболее общем виде ставка характеризует цену, по которой экономические агенты готовы обменивать текущие денежные средства на будущие. На финансовых рынках процентные ставки выполняют две ключевые функции: служат механизмом дисконтирования будущих денежных потоков и определяют предпочтения инвесторов при выборе между различными активами фиксированного дохода.

В практике анализа долговых инструментов особое значение имеет безрисковая процентная ставка, выступающая ориентиром для оценки стоимости денежных потоков, не несущих кредитного риска. В российских условиях таким ориентиром традиционно считается доходность государственных облигаций федерального займа (ОФЗ), обладающих минимальным кредитным риском и высокой ликвидностью. На международных рынках аналогичную роль выполняют казначейские облигации США и других развитых стран. Использование безрисковой ставки позволяет сопоставлять финансовые инструменты на единой базе и оценивать премии за риск.

Цена облигации определяется приведённой стоимостью будущих купонных выплат и номинала, суммированных с использованием соответствующей процентной ставки. Если $C_t$ обозначает купонный платёж в момент времени $t$, а $N$ — номинальную стоимость, то текущая цена облигации выражается через классическую формулу дисконтирования:
$$
P = \sum_{t=1}^{T} \frac{C_t}{(1+r)^t} + \frac{N}{(1+r)^T},
$$
где $r$ — ставка, отражающая требования инвестора к доходности данного инструмента. Увеличение $r$ приводит к снижению приведённой стоимости денежных потоков, а следовательно, и к падению рыночной цены облигации. Таким образом, связь между ставкой и ценой носит обратный характер, что лежит в основе процентного риска.

Поскольку рыночные процентные ставки изменяются во времени под воздействием денежно-кредитной политики, инфляционных ожиданий и состояния экономики, цены облигаций оказываются чувствительными к этим изменениям. Разные облигации по-разному реагируют на колебания процентных ставок, что определяется их сроком до погашения, купонной структурой и формой денежного потока. Именно через механизм дисконтирования процентная ставка становится ключевым фактором динамики стоимости долговых инструментов.

Таким образом, процентная ставка выступает как фундаментальная величина, определяющая стоимость облигации, её чувствительность к изменениям рыночной конъюнктуры и уровень процентного риска. В следующем разделе рассматривается структура кривой доходности, позволяющая анализировать поведение ставок различной срочности и их влияние на динамику долговых рынков.