\subsubsection{Понятие риска в портфельном инвестировании}

В современной финансовой теории риск рассматривается как ключевая характеристика инвестиционной деятельности и отражает степень неопределённости будущих результатов. Доходности финансовых инструментов и портфелей зависят от множества случайных факторов, поэтому их естественно описывать как случайные величины, заданные на некотором вероятностном пространстве. Такой подход последовательно проводится в стохастической финансовой математике, где прибыль и убыток по финансовым операциям рассматриваются как случайные величины с определёнными распределениями \cite{ShiryaevSFM}.

В классических курсах по теории вероятностей \cite{ShiryaevProb1}\cite{KoralovSinai} подчёркивается, что поведение случайной величины описывается не только её возможными значениями, но и вероятностями этих значений. Для количественного анализа используются числовые характеристики распределения, в том числе средний уровень и мера разброса. В финансовой интерпретации они позволяют описывать ожидаемый результат инвестирования и степень его изменчивости, то есть потенциальную нестабильность доходности.

В контексте портфельного инвестирования риск связан не только с поведением отдельного актива, но и со структурой зависимостей внутри портфеля. Вклад каждого инструмента в совокупную неопределённость определяется не только его собственной волатильностью, но и тем, насколько согласованно он изменяется с другими активами. В этом смысле фундаментальное значение приобретают ковариационная структура доходностей и эффекты диверсификации, подробно обсуждаемые в работах по теории портфеля, начиная с классической статьи Г.~Марковица, где риск формализуется через разброс доходности портфеля \cite{markowitz1952}.

При этом описывать риск только через средний уровень и разброс бывает недостаточно. Финансовые данные часто демонстрируют асимметрию распределений и наличие «тяжёлых хвостов», то есть повышенную вероятность экстремальных значений. В практическом риск-менеджменте особый интерес представляет именно вероятность и масштаб неблагоприятных исходов, связанных с существенными потерями. Это приводит к использованию риск-метрик, ориентированных на поведение распределения в его неблагоприятной области, таких как Value at Risk и Conditional Value at Risk, широко применяемых в современных стохастических моделях финансовых рисков \cite{ShiryaevSFM}.

В дальнейшем под риском будет пониматься вероятностная характеристика возможных неблагоприятных отклонений фактического результата от ожидаемого, который определяется формой распределений доходностей, зависимостями между активами и вероятностью реализации значимых потерь. Эти общие положения создают теоретическую основу для последующего анализа процентного риска, который является одним из ключевых источников неопределённости в облигационных портфелях.