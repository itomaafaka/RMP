\subsubsection{Практическая роль дюрации и выпуклости в измерении процентного риска}

С математической точки зрения дюрация и выпуклость являются первыми и вторыми производными цены облигации по процентной ставке:
\[
D = -\frac{1}{P}\frac{\partial P}{\partial y}, \qquad 
C = \frac{1}{P}\frac{\partial^2 P}{\partial y^2}.
\]
Эти выражения показывают, что локальная структура зависимости цены от ставки может быть аппроксимирована рядом Тейлора до второго порядка:
\[
\frac{\Delta P}{P} \approx 
- D \, \Delta y + \frac{1}{2} C \, (\Delta y)^2,
\]
что обосновывает использование дюрации и выпуклости для оценки процентного риска.

Использование дюрации и выпуклости позволяет оперативно оценивать чувствительность портфеля к изменениям ставок без необходимости пересчёта стоимости каждого инструмента по полной модели. Эти показатели широко применяются в:
\vspace{-3.5mm}
\begin{enumerate}
    \setlength{\itemsep}{1pt}
    \setlength{\parskip}{0pt}
    \setlength{\parsep}{0pt}
    \item Управлении процентным риском банков и инвестиционных фондов;
    \item Стресс-тестировании портфелей облигаций;
    \item Сравнении чувствительности государственных и корпоративных выпусков;
    \item Анализе стратегий хеджирования процентного риска;
    \item Оценке влияния шоков уровня, наклона и кривизны на стоимость активов.
\end{enumerate}
\vspace{-3.5mm}

В совокупности они дают приближённое квадратичное представление зависимости цены от ставки и образуют основу для локального анализа процентного риска, используемого как в академических моделях, так и в регуляторных методологиях. Оба являются важными показателями, которые в дальнейшем будут использоваться для количественной оценки риска в портфельном инвестировании.