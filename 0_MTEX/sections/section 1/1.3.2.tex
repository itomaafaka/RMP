\subsubsection{Структура кривой доходности: уровень, наклон и кривизна}

Кривая доходности представляет собой зависимость доходности безрисковых облигаций от срока до погашения. Она отражает, под какую ставку инвестор может разместить капитал на различные временные горизонты, и служит центральным инструментом анализа процентного риска. Форма кривой доходности агрегирует ожидания рынка относительно будущей динамики процентных ставок, инфляции и макроэкономической активности, а её изменения оказывают непосредственное влияние на цены облигаций различных сроков.

В аналитической практике принято выделять три ключевых компонента, описывающих динамику кривой доходности: уровень, наклон и кривизну. Эти компоненты позволяют компактно представить структуру процентных ставок и понять, какие типы шоков доминируют на рынке.

Уровень (level) характеризует общее положение кривой доходности. Рост уровня означает одновременное увеличение доходностей по всем срокам, что приводит к снижению цен облигаций независимо от их дюрации. Шоки уровня являются наиболее распространённым типом изменений, отражая, как правило, пересмотр базовой денежно-кредитной политики центрального банка или изменение инфляционных ожиданий.

Наклон (slope) отражает разницу между доходностями краткосрочных и долгосрочных облигаций. Рост наклона означает увеличение разницы между длинными и короткими ставками, тогда как падение наклона указывает на их сближение. Изменение наклона влияет на распределение процентного риска по срокам обращения: краткосрочные инструменты реагируют в первую очередь на политику центрального банка, тогда как долгосрочные — на ожидания экономического роста и инфляции.

Кривизна (curvature) описывает выпуклость кривой доходности и отражает различия между динамикой среднесрочных ставок и поведением коротких и длинных. Изменение кривизны приводит к локальным деформациям кривой, в результате которых среднесрочные доходности могут двигаться иначе, чем остальные сегменты. Такие изменения особенно важны при анализе портфелей, включающих облигации со средними сроками до погашения.

В совокупности эти три компонента позволяют компактно описывать поведение процентных ставок и анализировать влияние различных факторов на структуру кривой доходности. Именно такая декомпозиция лежит в основе большинства факторных моделей, применяемых в анализе процентного риска. В следующем разделе рассматривается факторный подход с использованием метода главных компонент, позволяющий количественно оценить вклад каждого компонента в динамику кривой доходности.