\subsubsection{Формальные определения доходности облигации}

Доходность облигации является базовой характеристикой её эффективности и лежит в основе количественной оценки процентного риска. 

В финансовой литературе доходность трактуется как относительное изменение стоимости инвестиции, включающее учёт купонных выплат \cite{Hull2022}\cite{Jorion2006}. Корректное определение доходности позволяет сравнивать инструменты различной купонной структуры, сроков погашения и ценовой динамики.

Наиболее распространённым показателем является простая доходность за период $[t, t+1]$. Пусть $P_t$ и $P_{t+1}$ обозначают чистые цены облигации в моменты $t$ и $t+1$, а $C_{t+1}$ — купонный платёж, приходящийся на данный интервал. Тогда простая доходность определяется как
$$
R_{t+1}=\frac{P_{t+1}+C_{t+1}-P_t}{P_t}.
$$
Этот показатель отражает фактический финансовый результат инвестора за период наблюдения и широко используется при расчётах валовой доходности.

В стохастическом и эконометрическом анализе нередко применяется логарифмическая доходность, задаваемая выражением
$$
r_{t+1}=\ln\left(\frac{P_{t+1}+C_{t+1}}{P_t}\right).
$$
Логарифмическая форма обладает важным свойством аддитивности по времени и упрощает работу с распределениями и моделями временных рядов \cite{ShiryaevSFM}. При умеренных ценовых колебаниях логарифмическая и простая доходности близки по величине.

Для анализа зрелости и сравнительной оценки долговых инструментов используется также показатель доходности к погашению (yield to maturity, YTM). Он определяется как внутренняя ставка доходности, при которой приведённая стоимость будущих денежных потоков равна текущей рыночной цене облигации:
$$
P_t = \sum_{i=1}^{n} \frac{C_i}{(1+y)^i} + \frac{N}{(1+y)^n}.
$$
Доходность к погашению служит важным элементом при построении кривой доходности и сравнении облигаций с различными сроками обращения.

Различные формулы доходности выполняют разные функции в анализе: простая доходность отражает фактическую динамику, логарифмическая — удобна для статистического моделирования, а YTM — характеризует структуру денежных потоков и применяется при анализе процентной кривой. Эти определения формируют основу для дальнейшего изучения статистических свойств доходностей, что рассматривается в следующем разделе.