\subsubsection{Выпуклость облигации}

При конечном изменении процентной ставки линейного приближения, основанного на дюрации, может оказаться недостаточно. Для учёта нелинейного характера зависимости цены от ставки используется показатель выпуклости:
$$
C = \frac{1}{P} \sum_{t=1}^{T} \frac{C_t \, t(t+1)}{(1+y)^{t+2}}.
$$
Полная формула изменения цены облигации с учётом выпуклости записывается как
$$
\frac{\Delta P}{P} \approx - D \cdot \Delta y + \frac{1}{2} C \cdot (\Delta y)^2.
$$
Выпуклость всегда положительна для стандартных облигаций с фиксированными купонами, что означает: фактическая цена при снижении ставок падает медленнее, а при росте ставок — быстрее, чем предсказывает дюрация. Это улучшает точность оценки процентного риска.

Облигации с высокой выпуклостью демонстрируют более устойчивое поведение при колебаниях ставок, что делает этот показатель особенно важным при стресс-тестировании и сравнении портфелей с разной структурой сроков до погашения.
