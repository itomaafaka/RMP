\subsubsection{Обоснование выбора цели и предмета исследования}

Выбор облигационного портфеля в качестве объекта исследования обусловлен как теоретическими, так и практическими соображениями. В современной финансовой системе долговые инструменты занимают центральное место: государственные облигации обеспечивают финансирование бюджетного дефицита, а корпоративные выпуски служат важным источником заимствований для компаний реального сектора. Для крупных институциональных инвесторов — банков, пенсионных фондов, страховых организаций, паевых и биржевых фондов — облигации формируют значительную долю портфелей, определяя общую доходность и уровень риск-профиля. Регулярность денежных потоков, наличие развитой инфраструктуры обращения и сравнительно высокая ликвидность делают долговые инструменты естественным объектом прикладного анализа риска \cite{ShiryaevSFM}.

С теоретической точки зрения облигации представляют особый класс финансовых активов, стоимость которых полностью определяется структурой будущих платежей и ставками дисконтирования. В моделях структуры процентных ставок именно динамика кривой доходности задаёт механизм переоценки долговых инструментов и формирует их чувствительность к изменениям процентных ставок \cite{Filipovic2009}. В отличие от акций, для которых существенную роль играют ожидания по прибыли, дивидендам и риск-премии, оценка облигаций теснее связана с межвременной стоимостью денег и поведением процентных ставок. Это делает облигации удобной площадкой для исследования именно процентного риска.

Практический интерес к процентному риску усиливается тем, что он носит преимущественно системный характер. Изменение процентных ставок затрагивает большинство долговых инструментов одновременно, поскольку в основе их оценки лежат одни и те же рыночные ориентиры — безрисковые ставки, доходности государственных бумаг, ставки денежного рынка. В терминологии теории вероятностей это означает, что доходности облигаций зависят от общих факторов, формирующих высокую степень совместимости их динамики \cite{KoralovSinai}. В результате даже хорошо диверсифицированный облигационный портфель остаётся уязвимым к шокам процентных ставок, а возможности снижения именно процентного риска за счёт простого расширения набора эмитентов ограничены.

Исторически процентный риск рассматривается как ключевой компонент риска облигационного портфеля и в рамках теории портфельного выбора. В классической работе Г.~Марковица риск портфеля связывается с разбросом его доходности, определяемым как свойствами отдельных инструментов, так и структурой зависимостей между ними \cite{markowitz1952}. Для долговых бумаг основным источником такого разброса выступает изменение рыночных процентных ставок, поскольку именно оно определяет переоценку текущей стоимости фиксированных будущих платежей. Тем самым процентный риск становится фундаментальным объектом анализа в задаче оптимизации и контроля риск-профиля долговых портфелей.

Отдельное значение имеет специфика российского рынка в рассматриваемый период. В 2014–2025 гг. денежно-кредитная политика Банка России сопровождалась несколькими эпизодами резкого изменения ключевой ставки, что приводило к существенным колебаниям доходностей государственных и корпоративных облигаций. Эти события показали, что именно процентный риск может выступать доминирующим фактором как для краткосрочных ценовых шоков, так и для среднесрочных изменений доходности портфелей долговых инструментов. Для участников рынка это усилило запрос на методологию количественной оценки процентного риска с использованием современных инструментов анализа.

Таким образом, выбор облигационного портфеля в качестве объекта исследования и процентного риска в качестве основного вида риска является естественным и обоснованным. Облигации обладают прозрачной структурой денежных потоков и прямой зависимостью стоимости от динамики процентных ставок, а процентный риск для них носит системный и зачастую определяющий характер. Это позволяет одновременно опираться на строгую теоретическую базу, развитую в стохастической финансовой математике и моделях структуры процентных ставок, и решать практическую задачу оценки риск-профиля реальных портфелей долговых инструментов.
