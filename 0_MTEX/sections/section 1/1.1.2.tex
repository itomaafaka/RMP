\subsubsection{Роль и особенности процентного риска}

Процентный риск занимает центральное место в структуре рисков облигационных портфелей, поскольку стоимость долговых инструментов напрямую зависит от уровня процентных ставок. В отличие от акций, динамика которых определяется совокупностью фундаментальных и рыночных факторов, цена облигации полностью определяется параметрами будущих денежных потоков и ставкой дисконтирования. Даже небольшие изменения структуры процентных ставок приводят к существенным переоценкам стоимости долговых инструментов, что отмечается как в практических финансовых исследованиях, так и в стохастических моделях оценки стоимости \cite{ShiryaevSFM}.

В академической литературе процентная ставка рассматривается как фундаментальная характеристика, определяющая временную стоимость денег и формирующая связь между текущей и будущей стоимостью финансовых активов. В курсах по стохастической финансовой математике и теории производных инструментов подчёркивается, что операции дисконтирования и начисления процента лежат в основе оценки стоимости потоков платежей и построения ценовых моделей долговых инструментов \cite{ShiryaevSFM,Hull2022}.

В монографиях, посвящённых структуре процентных ставок, процентный риск связывается с изменением кривой доходности — её уровня, наклона и кривизны \cite{Filipovic2009}. Эти компоненты оказывают различное влияние на стоимость облигаций разных сроков, что делает реакцию облигационного портфеля на изменение ставок неоднородной. Длинные бумаги обладают повышенной чувствительностью к изменению долгосрочной части кривой, тогда как краткосрочные инструменты реагируют преимущественно на изменения краткосрочных ставок.

Дополнительным аспектом процентного риска является его системный характер. Тогда как ценовые изменения отдельных облигаций могут частично компенсироваться внутри портфеля, изменения процентных ставок затрагивают всю совокупность инструментов одновременно. В работах по теории случайных процессов \cite{KoralovSinai} отмечается, что системные факторы формируют высокую степень зависимости между динамикой различных финансовых величин. Это приводит к тому, что возможности диверсификации процентного риска существенно ограничены по сравнению с другими видами рыночных рисков.

Для облигационных портфелей влияние процентного риска проявляется сразу в нескольких измерениях: изменении текущей рыночной стоимости, перераспределении доходности во времени и корректировке доходности к погашению. Поскольку структура долговых инструментов включает купонные выплаты и возврат номинала, изменение ставок приводит к изменению справедливой цены облигации даже при неизменности кредитного качества эмитента.

Таким образом, процентный риск представляет собой ключевой фактор неопределённости в облигационных портфелях. Он обусловлен стохастической природой процентных ставок, зависимостью стоимости долевых инструментов от параметров кривой доходности и системным характером влияния изменений ставок на набор финансовых активов. Эти особенности определяют необходимость комплексного анализа процентного риска и обосновывают его центральное место в структуре настоящего исследования.