\subsubsection{Факторная структура процентных ставок и метод главных компонент (PCA)}

Изменения кривой доходности во времени, как правило, имеют высокую степень совместимости между различными сроками до погашения. Эмпирические исследования показывают, что большая часть вариации процентных ставок может быть объяснена небольшим числом общих факторов, отражающих основные типы движений кривой. Такой подход получил широкое распространение благодаря классической работе Литтермана и Шейнкмана, где было установлено, что динамику ставок можно описать тремя доминирующими компонентами: уровнем, наклоном и кривизной.

Для формализации этого наблюдения используется метод главных компонент (Principal Component Analysis, PCA), позволяющий преобразовать коррелированные изменения доходностей в набор некоррелированных факторов. Исходными данными служат наблюдаемые изменения доходностей облигаций различных сроков. Если обозначить вектор изменений доходностей в момент времени $t$ как
$$
\Delta y_t = 
\begin{pmatrix}
\Delta y_t(\tau_1) \\
\Delta y_t(\tau_2) \\
\vdots \\
\Delta y_t(\tau_n)
\end{pmatrix},
$$
то PCA осуществляет разложение этого многомерного ряда на линейные комбинации ортогональных факторов:
$$
\Delta y_t = {W}{f}_t + \varepsilon_t,
$$

где ${W}$ - матрица factor loadings, ${f}_t$ - вектор главных компонент, а $\varepsilon_t$ - остаточный компонент.

Первый фактор, как правило, интерпретируется как уровень кривой доходности, так как его веса приблизительно постоянны вдоль всей кривой. Второй фактор отображает наклон, поскольку его коэффициенты обычно отличаются по знаку в коротком и длинном сегментах кривой. Третий фактор отражает кривизну, характеризуя движения среднесрочных ставок относительно коротких и длинных.

Использование PCA обладает двумя преимуществами. Во-первых, позволяет количественно оценивать вклад каждого фактора в общую динамику процентных ставок. Во-вторых, существенно упрощает последующее моделирование процентного риска, так как вместо многомерного процесса ставок анализируется небольшое число некоррелированных компонент. На практике первые три фактора объясняют до 95\% вариации доходностей в большинстве развитых и развивающихся рынков, включая рынок облигаций Российской Федерации.

Факторная структура процентных ставок, основанная на методе главных компонент, предоставляет компактный и информативный способ описания динамики кривой доходности. Это является фундаментом для дальнейшего анализа процентного риска портфеля и используется в прикладных моделях, рассматриваемых во второй главе.