\section{ Теоритическая часть}
% Риски портфельного инвестирования и обс1нование выбора
\subsection{Риски портфельного инвестирования и обоснование выбора}
% 1.1.1. Понятие риска в портфельном инвестировании
\subsubsection{Понятие риска в портфельном инвестировании}

В современной финансовой теории риск рассматривается как ключевая характеристика инвестиционной деятельности и отражает степень неопределённости будущих результатов. Доходности финансовых инструментов и портфелей зависят от множества случайных факторов, поэтому их естественно описывать как случайные величины, заданные на некотором вероятностном пространстве. Такой подход последовательно проводится в стохастической финансовой математике, где прибыль и убыток по финансовым операциям рассматриваются как случайные величины с определёнными распределениями \cite{ShiryaevSFM}.

В классических курсах по теории вероятностей \cite{ShiryaevProb1}\cite{KoralovSinai} подчёркивается, что поведение случайной величины описывается не только её возможными значениями, но и вероятностями этих значений. Для количественного анализа используются числовые характеристики распределения, в том числе средний уровень и мера разброса. В финансовой интерпретации они позволяют описывать ожидаемый результат инвестирования и степень его изменчивости, то есть потенциальную нестабильность доходности.

В контексте портфельного инвестирования риск связан не только с поведением отдельного актива, но и со структурой зависимостей внутри портфеля. Вклад каждого инструмента в совокупную неопределённость определяется не только его собственной волатильностью, но и тем, насколько согласованно он изменяется с другими активами. В этом смысле фундаментальное значение приобретают ковариационная структура доходностей и эффекты диверсификации, подробно обсуждаемые в работах по теории портфеля, начиная с классической статьи Г.~Марковица, где риск формализуется через разброс доходности портфеля \cite{markowitz1952}.

При этом описывать риск только через средний уровень и разброс бывает недостаточно. Финансовые данные часто демонстрируют асимметрию распределений и наличие «тяжёлых хвостов», то есть повышенную вероятность экстремальных значений. В практическом риск-менеджменте особый интерес представляет именно вероятность и масштаб неблагоприятных исходов, связанных с существенными потерями. Это приводит к использованию риск-метрик, ориентированных на поведение распределения в его неблагоприятной области, таких как Value at Risk и Conditional Value at Risk, широко применяемых в современных стохастических моделях финансовых рисков \cite{ShiryaevSFM}.

В дальнейшем под риском будет пониматься вероятностная характеристика возможных неблагоприятных отклонений фактического результата от ожидаемого, который определяется формой распределений доходностей, зависимостями между активами и вероятностью реализации значимых потерь. Эти общие положения создают теоретическую основу для последующего анализа процентного риска, который является одним из ключевых источников неопределённости в облигационных портфелях.
% 1.1.2. Особенности процентного риска и его роль в облигационных портфелях
\subsubsection{Роль и особенности процентного риска}

Процентный риск занимает центральное место в структуре рисков облигационных портфелей, поскольку стоимость долговых инструментов напрямую зависит от уровня процентных ставок. В отличие от акций, динамика которых определяется совокупностью фундаментальных и рыночных факторов, цена облигации полностью определяется параметрами будущих денежных потоков и ставкой дисконтирования. Даже небольшие изменения структуры процентных ставок приводят к существенным переоценкам стоимости долговых инструментов, что отмечается как в практических финансовых исследованиях, так и в стохастических моделях оценки стоимости \cite{ShiryaevSFM}.

В академической литературе процентная ставка рассматривается как фундаментальная характеристика, определяющая временную стоимость денег и формирующая связь между текущей и будущей стоимостью финансовых активов. В курсах по стохастической финансовой математике и теории производных инструментов подчёркивается, что операции дисконтирования и начисления процента лежат в основе оценки стоимости потоков платежей и построения ценовых моделей долговых инструментов \cite{ShiryaevSFM,Hull2022}.

В монографиях, посвящённых структуре процентных ставок, процентный риск связывается с изменением кривой доходности — её уровня, наклона и кривизны \cite{Filipovic2009}. Эти компоненты оказывают различное влияние на стоимость облигаций разных сроков, что делает реакцию облигационного портфеля на изменение ставок неоднородной. Длинные бумаги обладают повышенной чувствительностью к изменению долгосрочной части кривой, тогда как краткосрочные инструменты реагируют преимущественно на изменения краткосрочных ставок.

Дополнительным аспектом процентного риска является его системный характер. Тогда как ценовые изменения отдельных облигаций могут частично компенсироваться внутри портфеля, изменения процентных ставок затрагивают всю совокупность инструментов одновременно. В работах по теории случайных процессов \cite{KoralovSinai} отмечается, что системные факторы формируют высокую степень зависимости между динамикой различных финансовых величин. Это приводит к тому, что возможности диверсификации процентного риска существенно ограничены по сравнению с другими видами рыночных рисков.

Для облигационных портфелей влияние процентного риска проявляется сразу в нескольких измерениях: изменении текущей рыночной стоимости, перераспределении доходности во времени и корректировке доходности к погашению. Поскольку структура долговых инструментов включает купонные выплаты и возврат номинала, изменение ставок приводит к изменению справедливой цены облигации даже при неизменности кредитного качества эмитента.

Таким образом, процентный риск представляет собой ключевой фактор неопределённости в облигационных портфелях. Он обусловлен стохастической природой процентных ставок, зависимостью стоимости долевых инструментов от параметров кривой доходности и системным характером влияния изменений ставок на набор финансовых активов. Эти особенности определяют необходимость комплексного анализа процентного риска и обосновывают его центральное место в структуре настоящего исследования.
% 1.1.3. Обоснование выбора цели и предмета исследования
\subsubsection{Обоснование выбора цели и предмета исследования}

Выбор облигационного портфеля в качестве объекта исследования обусловлен как теоретическими, так и практическими соображениями. В современной финансовой системе долговые инструменты занимают центральное место: государственные облигации обеспечивают финансирование бюджетного дефицита, а корпоративные выпуски служат важным источником заимствований для компаний реального сектора. Для крупных институциональных инвесторов — банков, пенсионных фондов, страховых организаций, паевых и биржевых фондов — облигации формируют значительную долю портфелей, определяя общую доходность и уровень риск-профиля. Регулярность денежных потоков, наличие развитой инфраструктуры обращения и сравнительно высокая ликвидность делают долговые инструменты естественным объектом прикладного анализа риска \cite{ShiryaevSFM}.

С теоретической точки зрения облигации представляют особый класс финансовых активов, стоимость которых полностью определяется структурой будущих платежей и ставками дисконтирования. В моделях структуры процентных ставок именно динамика кривой доходности задаёт механизм переоценки долговых инструментов и формирует их чувствительность к изменениям процентных ставок \cite{Filipovic2009}. В отличие от акций, для которых существенную роль играют ожидания по прибыли, дивидендам и риск-премии, оценка облигаций теснее связана с межвременной стоимостью денег и поведением процентных ставок. Это делает облигации удобной площадкой для исследования именно процентного риска.

Практический интерес к процентному риску усиливается тем, что он носит преимущественно системный характер. Изменение процентных ставок затрагивает большинство долговых инструментов одновременно, поскольку в основе их оценки лежат одни и те же рыночные ориентиры — безрисковые ставки, доходности государственных бумаг, ставки денежного рынка. В терминологии теории вероятностей это означает, что доходности облигаций зависят от общих факторов, формирующих высокую степень совместимости их динамики \cite{KoralovSinai}. В результате даже хорошо диверсифицированный облигационный портфель остаётся уязвимым к шокам процентных ставок, а возможности снижения именно процентного риска за счёт простого расширения набора эмитентов ограничены.

Исторически процентный риск рассматривается как ключевой компонент риска облигационного портфеля и в рамках теории портфельного выбора. В классической работе Г.~Марковица риск портфеля связывается с разбросом его доходности, определяемым как свойствами отдельных инструментов, так и структурой зависимостей между ними \cite{markowitz1952}. Для долговых бумаг основным источником такого разброса выступает изменение рыночных процентных ставок, поскольку именно оно определяет переоценку текущей стоимости фиксированных будущих платежей. Тем самым процентный риск становится фундаментальным объектом анализа в задаче оптимизации и контроля риск-профиля долговых портфелей.

Отдельное значение имеет специфика российского рынка в рассматриваемый период. В 2014–2025 гг. денежно-кредитная политика Банка России сопровождалась несколькими эпизодами резкого изменения ключевой ставки, что приводило к существенным колебаниям доходностей государственных и корпоративных облигаций. Эти события показали, что именно процентный риск может выступать доминирующим фактором как для краткосрочных ценовых шоков, так и для среднесрочных изменений доходности портфелей долговых инструментов. Для участников рынка это усилило запрос на методологию количественной оценки процентного риска с использованием современных инструментов анализа.

Таким образом, выбор облигационного портфеля в качестве объекта исследования и процентного риска в качестве основного вида риска является естественным и обоснованным. Облигации обладают прозрачной структурой денежных потоков и прямой зависимостью стоимости от динамики процентных ставок, а процентный риск для них носит системный и зачастую определяющий характер. Это позволяет одновременно опираться на строгую теоретическую базу, развитую в стохастической финансовой математике и моделях структуры процентных ставок, и решать практическую задачу оценки риск-профиля реальных портфелей долговых инструментов.


% 1.2. Доходности облигаций и их статические свойства
\subsection{Доходности облигаций и их статические свойства}
% Формальные определения доходности облигации
\subsubsection{Формальные определения доходности облигации}

Доходность облигации является базовой характеристикой её эффективности и лежит в основе количественной оценки процентного риска. 

В финансовой литературе доходность трактуется как относительное изменение стоимости инвестиции, включающее учёт купонных выплат \cite{Hull2022}\cite{Jorion2006}. Корректное определение доходности позволяет сравнивать инструменты различной купонной структуры, сроков погашения и ценовой динамики.

Наиболее распространённым показателем является простая доходность за период $[t, t+1]$. Пусть $P_t$ и $P_{t+1}$ обозначают чистые цены облигации в моменты $t$ и $t+1$, а $C_{t+1}$ — купонный платёж, приходящийся на данный интервал. Тогда простая доходность определяется как
$$
R_{t+1}=\frac{P_{t+1}+C_{t+1}-P_t}{P_t}.
$$
Этот показатель отражает фактический финансовый результат инвестора за период наблюдения и широко используется при расчётах валовой доходности.

В стохастическом и эконометрическом анализе нередко применяется логарифмическая доходность, задаваемая выражением
$$
r_{t+1}=\ln\left(\frac{P_{t+1}+C_{t+1}}{P_t}\right).
$$
Логарифмическая форма обладает важным свойством аддитивности по времени и упрощает работу с распределениями и моделями временных рядов \cite{ShiryaevSFM}. При умеренных ценовых колебаниях логарифмическая и простая доходности близки по величине.

Для анализа зрелости и сравнительной оценки долговых инструментов используется также показатель доходности к погашению (yield to maturity, YTM). Он определяется как внутренняя ставка доходности, при которой приведённая стоимость будущих денежных потоков равна текущей рыночной цене облигации:
$$
P_t = \sum_{i=1}^{n} \frac{C_i}{(1+y)^i} + \frac{N}{(1+y)^n}.
$$
Доходность к погашению служит важным элементом при построении кривой доходности и сравнении облигаций с различными сроками обращения.

Различные формулы доходности выполняют разные функции в анализе: простая доходность отражает фактическую динамику, логарифмическая — удобна для статистического моделирования, а YTM — характеризует структуру денежных потоков и применяется при анализе процентной кривой. Эти определения формируют основу для дальнейшего изучения статистических свойств доходностей, что рассматривается в следующем разделе.
% Основные статистические характеристики доходностей
\subsubsection{Основные статистические характеристики доходностей}

Статистические свойства доходностей облигаций играют ключевую роль в построении моделей процентного риска. Анализ доходностей позволяет выявить закономерности их поведения во времени, оценить устойчивость динамики и определить подходящие методы моделирования. 

Пусть $\{r_t\}_{t=1}^T$ обозначает последовательность доходностей за равные интервалы наблюдения. из базовых характеристик является математическое ожидание доходности, определяемое как:
$$
\mu = \mathbb{E}[r_t].
$$
В дискретном случае, когда доходность принимает значения $r_i$ с вероятностями $p_i$, математическое ожидание задаётся суммой
$$
\mathbb{E}[r_t] = \sum_{i=1}^k r_i p_i.
$$
Если распределение доходности обладает плотностью $p(r)$, то математическое ожидание определяется интегралом
$$
\mathbb{E}[r_t] = \int_{\Omega} rp(r)dr.
$$

Однако для анализа риска более важной величиной является изменчивость доходности, измеряемая дисперсией
$$
\sigma^2 = \mathrm{Var}(r_t) = \mathbb{E}\left[(r_t - \mu)^2\right],
$$
отражающей степень отклонения фактических значений доходности от среднего уровня. Стандартное отклонение \(\sigma\) традиционно используется в качестве показателя волатильности прибыли.

Зависимость текущих значений доходности от прошлых наблюдений оценивается через автокорреляционную функцию
\[
\rho(k) = \frac{\mathrm{Cov}(r_t, r_{t-k})}{\mathrm{Var}(r_t)},
\]
где \(k\) — лаг. Наличие автокорреляции сигнализирует о возможных временных зависимостях, что необходимо учитывать при построении моделей ARIMA и GARCH. У облигаций разных сроков обращения автокорреляционная структура может существенно различаться вследствие особенностей формирования цен и влияния динамики процентных ставок.

Кроме того, доходности облигаций нередко характеризуются отклонениями от нормального распределения. Такие особенности, как асимметрия распределения и повышенная вероятность экстремальных значений (тяжёлые хвосты), оцениваются через коэффициенты асимметрии и эксцесса:
\[
\mathrm{Skew}(r_t)=\frac{\mathbb{E}\left[(r_t-\mu)^3\right]}{\sigma^3}, \qquad
\mathrm{Kurt}(r_t)=\frac{\mathbb{E}\left[(r_t-\mu)^4\right]}{\sigma^4}.
\]
Положительная эксцессивность (\(\mathrm{Kurt}(r_t) > 3\)) указывает на наличие частых резких скачков, что характерно для доходностей долговых инструментов в периоды нестабильности процентных ставок.

Статистические свойства доходностей определяют выбор методов моделирования в последующих разделах. В частности, наличие автокорреляций и кластеризации волатильности требует применения моделей временных рядов (ARIMA, GARCH), а тяжёлые хвосты и асимметрия существенно влияют на формирование кривых распределений, используемых при расчёте мер риска, таких как Value at Risk и Conditional Value at Risk.
% Особенности поведения доходностей различных облигаций
\subsubsection{Особенности поведения доходностей различных облигаций}

Доходности облигаций различаются в зависимости от типа эмитента, срока до погашения и купонной структуры инструмента. Эти различия оказывают существенное влияние на характер процентного риска и определяют специфику моделирования. В рамках анализа портфельного риска важно учитывать неоднородность поведения доходностей государственных и корпоративных облигаций, а также инструментов с различной дюрацией.

Одним из ключевых факторов является срок до погашения. Короткие облигации демонстрируют меньшую чувствительность к изменениям процентных ставок, что отражается в более низкой волатильности доходностей. Длинные выпуски, напротив, обладают повышенной изменчивостью доходности вследствие большего влияния будущих процентных ставок на текущую цену. Это приводит к тому, что ряды доходностей длинных облигаций, как правило, характеризуются более выраженной нестабильностью и более сильной реакцией на макроэкономические изменения.

Корпоративные облигации отличаются от государственных не только кредитным риском, но и более высокой волатильностью доходностей. Наличие кредитного спрэда усиливает реакцию доходностей на рыночные и кредитные шоки, что приводит к увеличению вероятности экстремальных отклонений и к появлению асимметрии распределений. В сравнении с ОФЗ корпоративные облигации также могут демонстрировать более высокие автокорреляции доходностей, связанные с особенностями ликвидности и структуры рынка.

Дополнительным фактором, влияющим на динамику доходностей, является купонная структура. Высококупонные облигации, как правило, менее чувствительны к колебаниям процентных ставок, поскольку значительная часть их стоимости определяется ближайшими купонными выплатами. Напротив, низкокупонные и бессрочные инструменты сильнее реагируют на изменения ставок, что отражается в их более волатильных доходностях.

Таким образом, доходности различных типов облигаций обладают неодинаковыми временными и статистическими свойствами. Различия в волатильности, автокорреляциях, асимметрии распределений и реакции на процентные шоки имеют существенное значение для выбора методов моделирования и оценки процентного риска, что рассматривается в последующих разделах.


% 1.3. Процентные ставки и кривая доходности
\subsection{Процентные ставки и кривая доходности}
% 1.3.1. Понятие процентной ставки и её роль в оценке стоимости облигаций
\subsubsection{Понятие процентной ставки и её роль в оценке стоимости облигаций}

Процентная ставка является базовым параметром финансового рынка и отражает стоимость денег во времени. В наиболее общем виде ставка характеризует цену, по которой экономические агенты готовы обменивать текущие денежные средства на будущие. На финансовых рынках процентные ставки выполняют две ключевые функции: служат механизмом дисконтирования будущих денежных потоков и определяют предпочтения инвесторов при выборе между различными активами фиксированного дохода.

В практике анализа долговых инструментов особое значение имеет безрисковая процентная ставка, выступающая ориентиром для оценки стоимости денежных потоков, не несущих кредитного риска. В российских условиях таким ориентиром традиционно считается доходность государственных облигаций федерального займа (ОФЗ), обладающих минимальным кредитным риском и высокой ликвидностью. На международных рынках аналогичную роль выполняют казначейские облигации США и других развитых стран. Использование безрисковой ставки позволяет сопоставлять финансовые инструменты на единой базе и оценивать премии за риск.

Цена облигации определяется приведённой стоимостью будущих купонных выплат и номинала, суммированных с использованием соответствующей процентной ставки. Если $C_t$ обозначает купонный платёж в момент времени $t$, а $N$ — номинальную стоимость, то текущая цена облигации выражается через классическую формулу дисконтирования:
$$
P = \sum_{t=1}^{T} \frac{C_t}{(1+r)^t} + \frac{N}{(1+r)^T},
$$
где $r$ — ставка, отражающая требования инвестора к доходности данного инструмента. Увеличение $r$ приводит к снижению приведённой стоимости денежных потоков, а следовательно, и к падению рыночной цены облигации. Таким образом, связь между ставкой и ценой носит обратный характер, что лежит в основе процентного риска.

Поскольку рыночные процентные ставки изменяются во времени под воздействием денежно-кредитной политики, инфляционных ожиданий и состояния экономики, цены облигаций оказываются чувствительными к этим изменениям. Разные облигации по-разному реагируют на колебания процентных ставок, что определяется их сроком до погашения, купонной структурой и формой денежного потока. Именно через механизм дисконтирования процентная ставка становится ключевым фактором динамики стоимости долговых инструментов.

Таким образом, процентная ставка выступает как фундаментальная величина, определяющая стоимость облигации, её чувствительность к изменениям рыночной конъюнктуры и уровень процентного риска. В следующем разделе рассматривается структура кривой доходности, позволяющая анализировать поведение ставок различной срочности и их влияние на динамику долговых рынков.
% 1.3.2. Структура кривой доходности
\subsubsection{Структура кривой доходности: уровень, наклон и кривизна}

Кривая доходности представляет собой зависимость доходности безрисковых облигаций от срока до погашения. Она отражает, под какую ставку инвестор может разместить капитал на различные временные горизонты, и служит центральным инструментом анализа процентного риска. Форма кривой доходности агрегирует ожидания рынка относительно будущей динамики процентных ставок, инфляции и макроэкономической активности, а её изменения оказывают непосредственное влияние на цены облигаций различных сроков.

В аналитической практике принято выделять три ключевых компонента, описывающих динамику кривой доходности: уровень, наклон и кривизну. Эти компоненты позволяют компактно представить структуру процентных ставок и понять, какие типы шоков доминируют на рынке.

Уровень (level) характеризует общее положение кривой доходности. Рост уровня означает одновременное увеличение доходностей по всем срокам, что приводит к снижению цен облигаций независимо от их дюрации. Шоки уровня являются наиболее распространённым типом изменений, отражая, как правило, пересмотр базовой денежно-кредитной политики центрального банка или изменение инфляционных ожиданий.

Наклон (slope) отражает разницу между доходностями краткосрочных и долгосрочных облигаций. Рост наклона означает увеличение разницы между длинными и короткими ставками, тогда как падение наклона указывает на их сближение. Изменение наклона влияет на распределение процентного риска по срокам обращения: краткосрочные инструменты реагируют в первую очередь на политику центрального банка, тогда как долгосрочные — на ожидания экономического роста и инфляции.

Кривизна (curvature) описывает выпуклость кривой доходности и отражает различия между динамикой среднесрочных ставок и поведением коротких и длинных. Изменение кривизны приводит к локальным деформациям кривой, в результате которых среднесрочные доходности могут двигаться иначе, чем остальные сегменты. Такие изменения особенно важны при анализе портфелей, включающих облигации со средними сроками до погашения.

В совокупности эти три компонента позволяют компактно описывать поведение процентных ставок и анализировать влияние различных факторов на структуру кривой доходности. Именно такая декомпозиция лежит в основе большинства факторных моделей, применяемых в анализе процентного риска. В следующем разделе рассматривается факторный подход с использованием метода главных компонент, позволяющий количественно оценить вклад каждого компонента в динамику кривой доходности.
% 1.3.3. Структура изменений ставки (PCA) и её значение для процентного риска
\subsubsection{Факторная структура процентных ставок и метод главных компонент (PCA)}

Изменения кривой доходности во времени, как правило, имеют высокую степень совместимости между различными сроками до погашения. Эмпирические исследования показывают, что большая часть вариации процентных ставок может быть объяснена небольшим числом общих факторов, отражающих основные типы движений кривой. Такой подход получил широкое распространение благодаря классической работе Литтермана и Шейнкмана, где было установлено, что динамику ставок можно описать тремя доминирующими компонентами: уровнем, наклоном и кривизной.

Для формализации этого наблюдения используется метод главных компонент (Principal Component Analysis, PCA), позволяющий преобразовать коррелированные изменения доходностей в набор некоррелированных факторов. Исходными данными служат наблюдаемые изменения доходностей облигаций различных сроков. Если обозначить вектор изменений доходностей в момент времени $t$ как
$$
\Delta y_t = 
\begin{pmatrix}
\Delta y_t(\tau_1) \\
\Delta y_t(\tau_2) \\
\vdots \\
\Delta y_t(\tau_n)
\end{pmatrix},
$$
то PCA осуществляет разложение этого многомерного ряда на линейные комбинации ортогональных факторов:
$$
\Delta y_t = {W}{f}_t + \varepsilon_t,
$$

где ${W}$ - матрица factor loadings, ${f}_t$ - вектор главных компонент, а $\varepsilon_t$ - остаточный компонент.

Первый фактор, как правило, интерпретируется как уровень кривой доходности, так как его веса приблизительно постоянны вдоль всей кривой. Второй фактор отображает наклон, поскольку его коэффициенты обычно отличаются по знаку в коротком и длинном сегментах кривой. Третий фактор отражает кривизну, характеризуя движения среднесрочных ставок относительно коротких и длинных.

Использование PCA обладает двумя преимуществами. Во-первых, позволяет количественно оценивать вклад каждого фактора в общую динамику процентных ставок. Во-вторых, существенно упрощает последующее моделирование процентного риска, так как вместо многомерного процесса ставок анализируется небольшое число некоррелированных компонент. На практике первые три фактора объясняют до 95\% вариации доходностей в большинстве развитых и развивающихся рынков, включая рынок облигаций Российской Федерации.

Факторная структура процентных ставок, основанная на методе главных компонент, предоставляет компактный и информативный способ описания динамики кривой доходности. Это является фундаментом для дальнейшего анализа процентного риска портфеля и используется в прикладных моделях, рассматриваемых во второй главе.    

% 1.4. Дюрация и выпуклость облигаций
\subsection{Дюрация и выпуклость облигаций}
% 1.4.1. Дюрация и её виды
\subsubsection{Модифицированная дюрация и дюрация Маколея}

Дюрация в классическом понимании была введена Ф.~Маколеем как средневзвешенный срок получения денежных потоков. Если поток платежей облигации равен $\{C_t\}_{t=1}^T$, а $y$ обозначает доходность к погашению, то цена облигации выражается как
$$
P = \sum_{t=1}^{T} \frac{C_t}{(1+y)^t}.
$$
Тогда Дюрация Маколея определяется формулой
$$
D_M = \frac{1}{P} \sum_{t=1}^{T} t \cdot \frac{C_t}{(1+y)^t}.
$$
Она показывает момент времени, в среднем соответствующий получению стоимости денежного потока облигации. Однако в анализе процентного риска более удобной является модифицированная дюрация, определяемая как
$$
D = \frac{D_M}{1+y}.
$$
Модифицированная дюрация непосредственно характеризует относительное изменение цены облигации при малом изменении ставки:
$$
\frac{\Delta P}{P} \approx - D \cdot \Delta y.
$$
Знак «минус» отражает обратную зависимость между ценой и доходностью: при росте ставок цена облигации падает.

Дюрация зависит от структуры денежных потоков: чем длиннее срок до погашения и чем ниже купон, тем выше дюрация и чувствительность облигации к процентным изменениям. Поэтому дюрация является базовым показателем при сравнении процентного риска различных долговых инструментов.

  
%
\subsubsection{Выпуклость облигации}

При конечном изменении процентной ставки линейного приближения, основанного на дюрации, может оказаться недостаточно. Для учёта нелинейного характера зависимости цены от ставки используется показатель выпуклости:
$$
C = \frac{1}{P} \sum_{t=1}^{T} \frac{C_t \, t(t+1)}{(1+y)^{t+2}}.
$$
Полная формула изменения цены облигации с учётом выпуклости записывается как
$$
\frac{\Delta P}{P} \approx - D \cdot \Delta y + \frac{1}{2} C \cdot (\Delta y)^2.
$$
Выпуклость всегда положительна для стандартных облигаций с фиксированными купонами, что означает: фактическая цена при снижении ставок падает медленнее, а при росте ставок — быстрее, чем предсказывает дюрация. Это улучшает точность оценки процентного риска.

Облигации с высокой выпуклостью демонстрируют более устойчивое поведение при колебаниях ставок, что делает этот показатель особенно важным при стресс-тестировании и сравнении портфелей с разной структурой сроков до погашения.
  
%
\subsubsection{Практическая роль дюрации и выпуклости в измерении процентного риска}

С математической точки зрения дюрация и выпуклость являются первыми и вторыми производными цены облигации по процентной ставке:
\[
D = -\frac{1}{P}\frac{\partial P}{\partial y}, \qquad 
C = \frac{1}{P}\frac{\partial^2 P}{\partial y^2}.
\]
Эти выражения показывают, что локальная структура зависимости цены от ставки может быть аппроксимирована рядом Тейлора до второго порядка:
\[
\frac{\Delta P}{P} \approx 
- D \, \Delta y + \frac{1}{2} C \, (\Delta y)^2,
\]
что обосновывает использование дюрации и выпуклости для оценки процентного риска.

Использование дюрации и выпуклости позволяет оперативно оценивать чувствительность портфеля к изменениям ставок без необходимости пересчёта стоимости каждого инструмента по полной модели. Эти показатели широко применяются в:
\vspace{-3.5mm}
\begin{enumerate}
    \setlength{\itemsep}{1pt}
    \setlength{\parskip}{0pt}
    \setlength{\parsep}{0pt}
    \item Управлении процентным риском банков и инвестиционных фондов;
    \item Стресс-тестировании портфелей облигаций;
    \item Сравнении чувствительности государственных и корпоративных выпусков;
    \item Анализе стратегий хеджирования процентного риска;
    \item Оценке влияния шоков уровня, наклона и кривизны на стоимость активов.
\end{enumerate}
\vspace{-3.5mm}

В совокупности они дают приближённое квадратичное представление зависимости цены от ставки и образуют основу для локального анализа процентного риска, используемого как в академических моделях, так и в регуляторных методологиях. Оба являются важными показателями, которые в дальнейшем будут использоваться для количественной оценки риска в портфельном инвестировании.  