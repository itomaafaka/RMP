\documentclass[aspectratio=169]{beamer}
% Прогресс-бар и нумерация
\usetheme[numbering=fraction,progressbar=frametitle]{metropolis}
\usepackage{graphicx}
\makeatletter
\setbeamertemplate{progress bar in frametitle}{%
  \begin{tikzpicture}[remember picture,overlay]
    % Размер и позиция «капсулы»
    \def\barW{\paperwidth}     % ширина
    \def\barH{0.15cm}          % высота полосы
    \def\barY{-0.32cm}         % смещение вниз внутри area заголовка
    % Прогресс (0..1)
    \pgfmathsetmacro{\p}{\insertframenumber/\inserttotalframenumber}

    % Фон (светлый), рамка и заполнение (оранжевое)
    \path[rounded corners=0.08cm, draw=ClaudeOrange, line width=1pt, fill=ClaudeSand]
      (0,\barY) rectangle (\barW,\barY+\barH);
    \path[rounded corners=0.08cm, fill=ClaudeOrange]
      (0,\barY) rectangle (\p*\barW,\barY+\barH);
  \end{tikzpicture}%
}
\makeatother
\metroset{block=fill}

% --- Язык ---
\usepackage[utf8]{inputenc}
\usepackage[T2A]{fontenc}
\usepackage[russian]{babel}

% --- Шрифты и математика ---
\usepackage[default]{opensans}
\renewcommand{\familydefault}{\sfdefault}
\usefonttheme{professionalfonts} 
\usepackage{amsmath,amssymb}
\usepackage{graphicx}
\usepackage{ragged2e}

% --- Цветовая схема ---
\definecolor{ClaudeOrange1}{HTML}{4967D6}
\definecolor{ClaudeOrange}{HTML}{E39A4C}
\definecolor{ClaudeSand}{HTML}{F6F3EE}
\definecolor{ClaudeDark}{HTML}{3C3328}

\setbeamercolor{background canvas}{bg=ClaudeSand}
\setbeamercolor{normal text}{fg=ClaudeDark}
\setbeamercolor{frametitle}{bg=ClaudeOrange1, fg=ClaudeSand}
\setbeamercolor{structure}{fg=ClaudeOrange}
\setbeamercolor{block title}{fg=white,bg=ClaudeOrange}
\setbeamercolor{block body}{bg=ClaudeSand,fg=ClaudeDark}

% --- tcolorbox для блоков со скруглениями ---
\usepackage{tcolorbox}
\tcbuselibrary{skins,breakable}
\colorlet{BlockBodyBG}{ClaudeSand}
\colorlet{BlockTitleFG}{white}
\colorlet{BlockFrame}{ClaudeOrange}

\setbeamertemplate{block begin}{%
  \begin{tcolorbox}[
      enhanced,
      breakable,
      colback=white,
      colframe=BlockFrame,
      coltitle=BlockTitleFG,
      arc=5pt,                    % радиус скругления (исправлено)
      boxrule=0.1pt,              % толщина рамки
      left=6pt,right=6pt,top=8pt,bottom=8pt, % внутренние отступы
      before skip=8pt,after skip=8pt,        % внешние отступы
      title=\insertblocktitle,
      fonttitle=\bfseries
  ]%
}
\setbeamertemplate{block end}{%
  \end{tcolorbox}%
}

% --- Метаданные презентации ---
\title{\huge{Хэджирование с помощью опционов}}
\author{Выгузов Алексей — ЭКРмд-01-24}
\begin{document}

\begin{frame}[plain]
  \titlepage
\end{frame}

\begin{frame}
    \centering\Large\textbf{0. Банковский счёт, экспонента и интеграл}        
\end{frame}

\begin{frame}{Что такое банковский счёт?}
    \begin{block}{Определение}
        \textbf{Банковский счёт} (безрисковый актив) — это детерминированный процесс $\{B(t)\}_{t\ge 0}$, удовлетворяющий уравнению
        $$
        \frac{dB(t)}{dt} = r(t)\,B(t), \quad r(t) \text{ — известная безрисковая ставка.}
        $$
    \end{block}
Начальное условие принято задавать как $B(0)=1$. Тогда решение имеет вид
\[
B(t) = \exp\!\left(\int_0^t r(s)\,ds\right).
\]
При постоянной ставке $r(t)\equiv r$ получаем классическую формулу:
\[
B(t)=e^{rt}.
\]  
\end{frame}

\begin{frame}{Банковский счёт}
    \textbf{Банковский счёт (безрисковый актив)} определяется как процесс $B(t)$, описывающий эволюцию капитала, размещённого под фиксированную и заранее известную процентную ставку $r$. Он обладает следующими свойствами:
\begin{itemize}
    \item \textbf{Детерминированность} - все значения заранее известны  
    \item \textbf{Отсутствие риска дефолта} - гарантированная выплата
    \item \textbf{Numeraire} — используется для оценки стоимости других активов
    
\end{itemize}

\end{frame}

\begin{frame}{Что означает эта формула}
Пусть промежуток времени \([0,T]\) разбит на равные отрезки длины \(\Delta t\):
\[
0,\ \Delta t,\ 2\Delta t,\ \dots,\ n\Delta t = T,\qquad 
\Delta t = \frac{T}{n}.
\]
Будем предполагать, что за один малый шаг времени капитал изменяется по правилу
\[
B(t+\Delta t) = B(t)\bigl(1+r\Delta t\bigr).
\]
После \(n\) шагов имеем
\[
B(T) = B(0)\,(1+r\Delta t)^n.
\]
\end{frame}


\begin{frame}{Что означает эта формула}
    Переходя к пределу при \(\Delta t \to 0\) (то есть при \(n\to\infty\)), получаем
\[
\lim_{\Delta t\to 0} (1+r\Delta t)^{T/\Delta t} = e^{rT}.
\]

Таким образом, в пределе дискретная динамика переходит в дифференциальное уравнение
\[
\frac{dB(t)}{dt}
= \lim_{\Delta t\to 0} 
  \frac{B(t+\Delta t)-B(t)}{\Delta t}
= r B(t).
\]
\end{frame}

\begin{frame}{Пример}
    Пусть на банковском счёте лежит сумма $B(0)=100{\,}000 \text{ рублей},$а годовая ставка равна ключевой ставке ЦБ РФ: $r = 16,5\% = 0.165.$\\
Рассчитаем рост капитала за один год ($T=1$) при разных схемах начисления.
\end{frame}

\begin{frame}{Решение}
1. Капитализация 1 раз в год (линейная модель):
\begin{align*}
B_{1}(1) &= 100{\,}000\left(1+\frac{0.16}{1}\right)^{1} = 100{\,}000 \cdot 1.16 = \bf{116{\,}000} \text{ рублей}.         
\end{align*}
2. Капитализация раз в месяц ($n=12$):
\begin{align*}
B_{12}(1) &= 100{\,}000\left(1+\frac{0.16}{12}\right)^{12} \approx 100{\,}000 \cdot 1.17180 \approx \bf{117{\,}180} \text{ рублей}.
\end{align*}
3. Непрерывное начисление ($n\to\infty$):
\[B_{\text{cont}}(1) = 100{\,}000\, e^{0.16} \approx 100{\,}000 \cdot 1.17351 \approx \bf{117{\,}351} \text{ рублей}.\]
\end{frame}

\begin{frame}{График}
    \begin{figure}[h]
    \centering
    \begin{center}
        \includegraphics[width=0.85\textwidth]{bank_exponent.png}
    \end{center}
\end{figure}
\end{frame}

\begin{frame}{Сравнение результатов}
\[
\overbrace{116{\,}000}^{{B_1(1)}}
\;<\;
\overbrace{117{\,}180}^{{B_{12}(1)}}
\;<\;
\overbrace{117{\,}351}^{{B_{\text{cont}}(1)}}
\]

\begin{itemize}
    \item При простой схеме процент не капитализируется — итог меньше всех.
    \item Чем чаще происходит капитализация, тем выше итоговая сумма.
    \item Непрерывный рост даёт теоретический максимум: $e^{r}$.
\end{itemize}
\centering\textbf{Рост ускоряется со временем — \textbf{экспоненциальный эффект}.}
\end{frame}

\begin{frame}
    \centering\Large\textbf{1. Случайность и риск}
\end{frame}


\begin{frame}{Цена завтра 1/3}
    \begin{block}{Пример}
        Представим, что у нас есть акция, которая сегодня стоит 100 рублей.\\
        Что будет с этой ценой завтра?   
    \end{block}
\end{frame}

\begin{frame}{Цена завтра 2/3}
    \begin{block}{Пример}
        Представим, что у нас есть акция, которая сегодня стоит 100 рублей.\\
        Что будет с этой ценой завтра?   
    \end{block}
    На самом деле вариантов бесконечно много: 99, 101, 98.5, 105 — любое число, и заранее мы его не знаем.\\
\end{frame}

\begin{frame}{Цена завтра 3/3}
    \begin{block}{Пример}
        Представим, что у нас есть акция, которая сегодня стоит 100 рублей.\\
        Что будет с этой ценой завтра?   
    \end{block}
    На самом деле вариантов бесконечно много: 99, 101, 98.5, 105 — любое число, и заранее мы его не знаем.\\
    \textbf{Цена актива} — это случайная величина, а будущее рынка — это множество возможных исходов, которыми мы должны уметь работать.
\end{frame}

\begin{frame}{Риск и принятие решений}

\begin{itemize}
    \item Финансовые решения принимаются \textbf{сегодня}, а результат проявляется \textbf{в будущем}.
    \item Завтрашняя цена неизвестна → каждое решение связано с неопределённостью.
    \item \textbf{Риск} возникает тогда, когда итог зависит от того, какой сценарий реализуется.
    \item Чтобы принимать осмысленные решения, нужно понимать:
    \begin{itemize}
        \item какие исходы возможны;
        \item каковы их вероятности;
        \item какие убытки или прибыли они несут.
    \end{itemize}
    \item Модели риска (в т.ч. биномиальная модель и дельта-хедж) — инструменты управления этой неопределённостью.
\end{itemize}

\end{frame}

\begin{frame}{}
    \centering\Huge\textbf{$(\Omega, \mathcal{F}, \mathbb{P})$}\\
    \centering\Large\textbf{2. Теория вероятностей как язык}
\end{frame}

\begin{frame}{Вероятностное пространство}
    \begin{block}{Определение}
        Вероятностное пространство — это тройка $(\Omega, \mathcal{F}, \mathbb{P})$, где
        \begin{enumerate}
            \item \(\Omega\) — множество всех возможных исходов (пространство элементарных событий);
            \item \(\mathcal{F}\) — \textbf{$\sigma$-алгебра} событий, то есть множество подмножеств \(\Omega\), для которых определены вероятности;
            \item \(\mathbb{P}\) — функция вероятности, каждому событию \(A \in \mathcal{F}\) ставит число \(\mathbb{P}(A)\) в интервале \([0,1]\) и удовлетворяет аксиомам вероятности.
        \end{enumerate}
    \end{block}
\end{frame}

\begin{frame}{Теория вероятностей как язык (формально)}

\textbf{Вероятностное пространство задаётся тремя объектами:}

\begin{enumerate}
    \item \textbf{Множество исходов} (возможные будущие состояния рынка):
    \[
    \Omega = \{\text{up},\, \text{down},\, \text{flat}\}.
    \]

    \item \textbf{Система событий} — все наблюдаемые комбинации исходов:
    \[
    \mathcal{F} = \sigma(\{\text{up}\},\{\text{down}\},\{\text{flat}\}).
    \]

    \item \textbf{Вероятностная мера}:
    \[
    P(\text{up})=0.3,\quad P(\text{down})=0.4,\quad P(\text{flat})=0.3.
    \]
\end{enumerate}

\centering
\textbf{Получаем вероятностное пространство } $(\Omega,\mathcal{F},P)$.
\end{frame}

\begin{frame}
    \centering\Large\textbf{3. Почему банкам нужно хеджирование}
\end{frame}

\begin{frame}{Почему банкам нужно хеджирование}

\begin{itemize}
    \item Банк подвержен рыночным рискам:
    \begin{itemize}
        \item процентным,
        \item валютным,
        \item фондовым.
    \end{itemize}
    \item Проданные клиентам опционы создают \textbf{нелинейный} P\&L.
    \item Малые изменения цены базового актива могут вызывать крупные колебания результата.
    \item Хеджирование — инструмент стабилизации P\&L в условиях рыночной неопределённости.
\end{itemize}

\end{frame}

\begin{frame}{Что такое хеджирование}

\begin{itemize}
    \item \textbf{Хеджирование} — снижение риска за счёт корректирующей позиции.
    \item Не цель заработка, а цель — \textbf{стабилизация P\&L}.
    \item Компенсация чувствительности портфеля к изменениям цены.
    \item Основная идея: нейтрализовать влияние случайных движений рынка.
\end{itemize}

\end{frame}

\begin{frame}
    \centering\Large\textbf{4. Опционы}
\end{frame}

\begin{frame}{Что такое опцион?}
    \begin{block}{Определение}
        Опцион — это финансовый контракт, который даёт его владельцу \textbf{право}, но не обязанность,совершить определённое действие в будущем.
        \begin{itemize}
            \item либо \textbf{купить} базовый актив по фиксированной цене,
            \item либо \textbf{продать} базовый актив по фиксированной цене.
        \end{itemize}
    \end{block}

    Это ключевая характеристика:
    \begin{itemize}
        \item Покупатель опциона принимает решение \emph{после того, как узнает рыночную цену}. \\
        \item Продавец опциона, наоборот, берёт на себя \textbf{обязательство} выполнить решение покупателя. 
    \end{itemize}
    \centering\textbf{Поэтому покупатель всегда платит премию, а продавец её получает.}
\end{frame}

\begin{frame}{Вспомогательные обозначения 1/3}
    \begin{itemize}
        \item \textbf{Страйк (K)} — это заранее установленная цена сделки, по которой действует право опциона
    \end{itemize}
\end{frame}


\begin{frame}{Вспомогательные обозначения 2/3}
    \begin{itemize}
        \item \textbf{Страйк (K)} — это заранее установленная цена сделки, по которой действует право опциона
        \item \textbf{Экспирация (T)} — момент времени, когда право может быть использовано (европейский стиль). В момент $T$ определяется конечная выплата опциона.
    \end{itemize}
\end{frame}


\begin{frame}{Вспомогательные обозначения 3/3}
    \begin{itemize}
        \item \textbf{Страйк (K)} — это заранее установленная цена сделки, по которой действует право опциона
        \item \textbf{Экспирация (T)} — момент времени, когда право может быть использовано (европейский стиль). В момент $T$ определяется конечная выплата опциона.
    \end{itemize}
    \begin{block}{Определение(Формальное)}
        \textbf{Опцион} — это право совершить операцию по цене $K$ в момент времени $T$, когда уже известно фактическое значение цены $S_T$.    
    \end{block}
\end{frame}


\begin{frame}{Виды опционов}
\textbf{Call-опцион} — право \emph{купить} актив по цене $K$.
\begin{itemize}
    \item выгоден, если $S_T > K$;
    \item не нужен, если $S_T \le K$.
\end{itemize}

\textbf{Put-опцион} — право \emph{продать} актив по цене $K$.
\begin{itemize}
    \item выгоден, если $S_T < K$;
    \item не используется, если $S_T \ge K$.
\end{itemize}
\end{frame}

\begin{frame}{Лонги и шорты}
    \begin{itemize}
        \item \textbf{Long Call}: вы покупаете право купить. Платите премию.
        \item \textbf{Short Call}: вы продаёте право купить. Получаете премию.
        \item \textbf{Long Put}: вы покупаете право продать. Платите премию.
        \item \textbf{Short Put}: вы продаёте право продать. Получаете премию.
    \end{itemize}
    
    \textcolor{red}{$(!!!)$} Здесь важно не перепутать: long и short в опционах — это не владение активом, а владение \textbf{правом} или \textbf{обязательством}.
\end{frame}

\begin{frame}{Лонги и шорты}
    \begin{itemize}
        \item \textbf{Long Call}: вы покупаете право купить. Платите премию.
        \item \textbf{Short Call}: вы продаёте право купить. Получаете премию.
        \item \textbf{Long Put}: вы покупаете право продать. Платите премию.
        \item \textbf{Short Put}: вы продаёте право продать. Получаете премию.
    \end{itemize}
    
    \textcolor{red}{$(!!!)$} Здесь важно не перепутать: long и short в опционах — это не владение активом, а владение \textbf{правом} или \textbf{обязательством}.
    \begin{itemize}
        \item Покупатель (long) имеет право и ограниченный риск.  \\
        \item Продавец (short) имеет обязательство и потенциально неограниченный риск.
    \end{itemize}
\end{frame}

\begin{frame}
    \centering\huge{Немного формализма...\\ \textit{(выводим формулки наглядно)}}
\end{frame}

\begin{frame}{Call}
    Держатель call-опциона имеет право купить актив по $K$. Рассмотрим возможные действия в зависимости от цены $S_T$:
\[
\begin{cases}
\text{если } S_T \le K,\ \text{право не нужно} \Rightarrow \text{выплата } 0, \\
\text{если } S_T > K,\ \text{покупаем по } K\ \text{и продаём по } S_T \Rightarrow S_T - K.
\end{cases}
\]
Это кусочная функция, которую компактно записывают как
\[
\text{Payoff}_{\text{call}} = \max(S_T - K, 0).
\]
\end{frame}

\begin{frame}{Put}
    Держатель put имеет право продать по $K$:
\[
\begin{cases}
\text{если } S_T \ge K,\ \text{право не нужно} \Rightarrow \text{выплата } 0, \\
\text{если } S_T < K,\ \text{продаём по } K\ \text{то, что стоит } S_T \Rightarrow K - S_T.
\end{cases}
\]
Компактная запись:
\[
\text{Payoff}_{\text{put}} = \max(K - S_T, 0).
\]
\end{frame}

\begin{frame}{Все позиции}
Теперь запишем четыре фундаментальные позиции:
\[
\begin{aligned}
    \begin{cases}
        \text{Payoff}_{\text{Long Call}} &= \max(S_T - K, 0),\\[2pt]
        \text{Payoff}_{\text{Short Call}} &= -\max(S_T - K, 0),\\[2pt]
        \text{Payoff}_{\text{Long Put}} &= \max(K - S_T, 0),\\[2pt]
        \text{Payoff}_{\text{Short Put}} &= -\max(K - S_T, 0).
    \end{cases}
\end{aligned}
\]
\textit{Минус означает обязательство выплатить то, что получает покупатель. }
\end{frame}

\begin{frame}{Payoff - Long Call $\max(S_T - K, 0)$}
    \begin{figure}[h]
        \centering
            \begin{center}
                \includegraphics[width=1\textwidth]{/Users/ito/Downloads/q/long_call.png}
            \end{center}
    \end{figure}
\end{frame}

\begin{frame}{Payoff - Short Call $-\max(S_T - K, 0)$}
    \begin{figure}[h]
        \centering
            \begin{center}
                \includegraphics[width=1\textwidth]{/Users/ito/Downloads/q/short_call.png}
            \end{center}
    \end{figure}
\end{frame}

\begin{frame}{Payoff - Long Put $\max(K - S_T, 0)$}
    \begin{figure}[h]
        \centering
            \begin{center}
                \includegraphics[width=1\textwidth]{/Users/ito/Downloads/q/long_put.png}
            \end{center}
    \end{figure}
\end{frame}

\begin{frame}{Payoff - Short Put $-\max(K - S_T, 0)$}
    \begin{figure}[h]
        \centering
            \begin{center}
                \includegraphics[width=1\textwidth]{/Users/ito/Downloads/q/short_put.png}
            \end{center}
    \end{figure}
\end{frame}

\begin{frame}{Пример}
Пусть базовый актив имеет возможные цены на экспирации:
\[
S_T \in \{80,\ 100,\ 130\}.
\]
Возьмём страйк $K = 100$. Тогда для каждого сценария вычислим выплаты:\\
Теперь для каждого $S_T$ вычислим выплаты по четырём фундаментальным позициям:\\
\centering{Long Call (LC), Short Call (SC), Long Put (LP), Short Put (SP)}.
\end{frame}


\begin{frame}{CALL-OPTION}
1. Long Call по определению:
\[
\text{Payoff}_{LC}(S_T) = \max(S_T - K, 0).
\]
\[
\begin{aligned}
S_T = 80:  &\quad \max(80 - 100, 0) = 0,\\
S_T = 100: &\quad \max(100 - 100, 0) = 0,\\
S_T = 130: &\quad \max(130 - 100, 0) = 30.
\end{aligned}
\]

2. Short Call - продавец call обязуется выплатить покупателю противоположный результат:
\[
\text{Payoff}_{SC}(S_T) = -\max(S_T - K, 0).
\]
\[
\begin{aligned}
S_T = 80:  &\quad 0,\\
S_T = 100: &\quad 0,\\
S_T = 130: &\quad -30.
\end{aligned}
\]  
\end{frame}


\begin{frame}{PUT-OPTION}
    3. Long Put - по определению:
\[
\text{Payoff}_{LP}(S_T) = \max(K - S_T, 0).
\]
\[
\begin{aligned}
S_T = 80:  &\quad \max(100 - 80, 0) = 20,\\
S_T = 100: &\quad \max(100 - 100, 0) = 0,\\
S_T = 130: &\quad \max(100 - 130, 0) = 0.
\end{aligned}
\]

4. Short Put - по оперделению продавец put получает результат со знаком «минус»:
\[
\text{Payoff}_{SP}(S_T) = -\max(K - S_T, 0).
\]
\[
\begin{aligned}
S_T = 80:  &\quad -20,\\
S_T = 100: &\quad 0,\\
S_T = 130: &\quad 0.
\end{aligned}
\]

\end{frame}

\begin{frame}{Итог}
\[
\begin{array}{c|c|c|c|c}
S_T & LC & SC & LP & SP \\ \hline
80  & 0   & 0   & 20  & -20 \\
100 & 0   & 0   & 0   & 0   \\
130 & 30  & -30 & 0   & 0
\end{array}
\]
\begin{itemize}
    \item При низкой цене ($S_T=80$) ценность имеет только put: право продать по 100 приносит 20.  
    \item Вблизи страйка ($S_T=100$) все опционы на границе — payoff равен нулю.  
    \item При высокой цене ($S_T=130$) ценность имеет только call: право купить за 100 и продать за 130 даёт 30.  
\end{itemize}
    
\end{frame}

\begin{frame}
    \centering\huge\textbf{Модели ценообразования\\ Биномиальная модель}
\end{frame}

\begin{frame}
    \centering\huge\textbf{ОФФ-ТОП. Что такое бином Ньютона?}
\end{frame}

\begin{frame}{Оффтоп: что такое бином Ньютона (интуитивно)}

\textbf{Бином Ньютона} — это формула, которая описывает разложение выражения $(a+b)^n$.

\begin{itemize}
    \item при раскрытии скобок мы выбираем $n$ раз — берём $a$ или $b$;
    \item каждая комбинация даёт одно слагаемое вида $a^{k}b^{\,n-k}$;
    \item количество таких комбинаций определяется тем, \emph{сколько способов выбрать} $k$ раз $a$ среди $n$ шагов;
    \item поэтому в формуле появляются биномиальные коэффициенты $\binom{n}{k}$.
\end{itemize}

Главная идея: \textbf{$(a+b)^n$ — это сумма всех возможных комбинаций выбора $a$ и $b$ по $n$ раз}.
\end{frame}

\begin{frame}{Бином Ньютона: формула и связь с вероятностью}
\textbf{Формула бинома Ньютона:}
\[
(a+b)^n = 
\sum_{k=0}^{n} \binom{n}{k}\, a^{\,n-k}\, b^{\,k}.
\]
Биномиальный коэффициент:
\[
\binom{n}{k} = 
\frac{n!}{k!\,(n-k)!}.
\]
\end{frame}

\begin{frame}{Интуиция случайности: монетка}

Самый простой случайный объект — обычная монета.  
Каждый подброс даёт один из двух исходов:
\[
\Omega = \{\text{орёл},\ \text{решка}\}.
\]

Каждый исход сам по себе не особенно интересен.  
Но если подбросить монету много раз, начинаются закономерности.

\begin{itemize}
    \item при $10$ бросках мы ожидаем примерно $5$ орлов;
    \item при $1000$ бросках — около $500$;
    \item но каждый конкретный эксперимент даёт немного другое число.
\end{itemize}
 
\textbf{Отдельный исход непредсказуем, но совокупное поведение имеет структуру}.
\end{frame}

\begin{frame}{Что такое случайная величина (формально)}

Случайная величина — это \textbf{не «случайное число»}.  
Это \emph{функция}, которая каждому исходу ставит в соответствие число:
\[
\xi : \Omega \to \mathbb{R}.
\]
Каждый исход $\omega \in \Omega$ отображается в значение $\xi(\omega)$.
Примеры:
\begin{itemize}
    \item Подбрасываем монету 5 раз: исход — последовательность из H/T.  
    Случайная величина:
    \[
    X(\omega)=\text{число орлов в исходе } \omega.
    \]

    \item В финансах $S_T$ — тоже случайная величина:
    \[
    S_T(\omega) = \text{цена актива в момент } T.
    \]
\end{itemize}

Случайная величина — это правило, которое превращает исходы в числа.
\end{frame}

\begin{frame}{Что такое распределение случайной величины}

Интуитивно: распределение — это закон того, \emph{как часто} случайная величина принимает различные значения.\\
Формально: распределение — это мера на $\mathbb{R}$, индуцированная отображением $\xi$:
\[
P_{\xi}(A) = P\bigl(\xi^{-1}(A)\bigr).
\]

\begin{itemize}
    \item Чтобы узнать вероятность того, что $\xi \in A$,  
    надо взять все исходы $\omega$, которые приводят к этому событию.
    \item Распределение описывает структуру случайной величины то есть какие значения и с какими вероятностями она может принимать.
\end{itemize}

\end{frame}

\begin{frame}{Биномиальное распределение: постановка задачи}

\textbf{Задача.}  
Какова вероятность того, что при $5$ подбрасываниях монеты ровно $3$ раза выпадет орёл?
Определим параметры:
\begin{itemize}
    \item число испытаний: $n = 5$;
    \item число «успехов»: $k = 3$;
    \item вероятность успеха: $p = \frac{1}{2}$;
    \item вероятность неуспеха: $q = 1 - p = \frac{1}{2}$.
\end{itemize}
Общая формула биномиального распределения:
\[
P(X = k) = \binom{n}{k}\, p^{k}\, q^{\,n-k}.
\]
В нашем случае:
\[
P(X=3)=\binom{5}{3}\left(\frac{1}{2}\right)^{3}\left(\frac{1}{2}\right)^{2}.
\]
\end{frame}

\begin{frame}{Биномиальное распределение: решение}
Вычислим биномиальный коэффициент:
\[
\binom{5}{3} = \frac{5!}{3!\,2!}
= \frac{5\cdot 4\cdot 3\cdot 2\cdot 1}{(3\cdot 2\cdot 1)(2\cdot 1)}
= 10.
\]
Тогда вероятность равна:
\[
P(X=3) = 10 \cdot \left(\frac{1}{2}\right)^3 \left(\frac{1}{2}\right)^2
= 10 \cdot \frac{1}{8} \cdot \frac{1}{4}
= \frac{10}{32}
= \frac{5}{16}.
\]
\begin{center}
\Large\textbf{Ответ: } $\displaystyle \frac{5}{16}$.
\end{center}

\end{frame}

\begin{frame}{От монетки к финансовой логике}
Представим теперь не монетку, а цену актива.\\Сегодня цена равна $S_0$. Завтра будет один из двух сценариев:
\[
S_T \text{ чуть выше}, \quad\text{или}\quad S_T \text{ чуть ниже}
\]
\textbf{в каждый маленький промежуток времени цена может сделать один из нескольких типичных шагов}.\\
Монетка — это аналогия:
\begin{itemize}
    \item орёл $\iff$ цена выросла;
    \item решка $\iff$ цена упала.
\end{itemize}
\end{frame}

\begin{frame}{Пример}

Пусть акция стоит $100$ сегодня.

Завтра она:
\begin{itemize}
    \item либо \emph{чуть подрастёт} — например, до $102$;
    \item либо \emph{чуть просядет} — например, до $98$.
\end{itemize}

Мы не знаем, какой сценарий произойдёт.  
Но важно другое:  
\textbf{нам известны возможные варианты, и каждый из них имеет вероятность}.  

Это позволяет:
\begin{itemize}
    \item строить сценарии будущего,
    \item оценивать риск,
    \item моделировать выплаты опционов,
    \item и — главное — строить хедж.
\end{itemize}
\end{frame}

\begin{frame}{Почему двух сценариев достаточно?}

На очень маленьком промежутке времени поведение цены похоже на подброс монетки:

\begin{itemize}
    \item цена делает \textbf{маленький стандартный шаг вверх}  
    (реакция на новости, ликвидность, дисбаланс ордеров);
    \item или \textbf{маленький стандартный шаг вниз}.
\end{itemize}

Это не означает, что цена \emph{реально} может принимать только два значения.  \\
Это означает, что \textbf{для моделирования одного короткого шага двух вариантов достаточно}.

\end{frame}

\begin{frame}{Что такое арбитраж в одном шаге}
Арбитраж — это стратегия без начальных затрат, которая гарантированно приносит прибыль.
В одном шаге модели:
\[
S_0 \to 
\begin{cases}
uS_0 & (\text{рост})\\
dS_0 & (\text{падение})
\end{cases}
\]
Банковский счёт:
\[
B_1 = B_0(1+r\Delta t)
\]
Арбитраж появляется, если можно:
\begin{itemize}
    \item составить портфель из $S$ и $B$,
    \item не вложив ни копейки,
    \item и получить прибыль в обоих состояниях.
\end{itemize}
\end{frame}

\begin{frame}{Когда возникает арбитраж}
Рассмотрим отношение доходности актива и банка.

Если:
\[
u \le 1+r\Delta t
\]
то актив растёт медленнее банка → выгодно занять актив и держать банк.

Если:
\[
d \ge 1+r\Delta t
\]
то актив не падает → выгодно занять деньги под $r$ и купить актив.

В обоих случаях можно получить гарантированную прибыль.
\end{frame}

\begin{frame}{Условие отсутствия арбитража}
Чтобы ни одна арбитражная стратегия не существовала, требуется:

\[
d < 1+r\Delta t < u.
\]

Это означает:
\begin{itemize}
    \item рост актива быстрее роста банка,
    \item падение актива слабее падения банка.
\end{itemize}

Только в этом случае модель консистентна.
\end{frame}

\begin{frame}{Требование отсутствия арбитража}
Чтобы не было арбитража, средняя цена под правильной вероятностью должна равняться росту банка:

\[
\mathbb{E}^{\mathbb{Q}}[S_1]
= S_0(1+r\Delta t).
\]
\[
q = \frac{(1+r\Delta t)-d}{u-d}.
\]

Это единственная вероятность, у которой нет арбитража.
\end{frame}

\begin{frame}{Что такое риск-нейтральная вероятность}
Риск-нейтральная вероятность:
\[
q = \frac{(1+r\Delta t)-d}{u-d}
\]
\begin{itemize}
    \item не реальная вероятность;
    \item не ожидание рынка;
    \item техника для устранения арбитража;
    \item обеспечивает, что дисконтированная цена актива — мартингал.
\end{itemize}
\end{frame}

\begin{frame}{Выплаты по call-опциону в узлах}
\[
C_u = \max(uS_0 - K, 0), \qquad
C_d = \max(dS_0 - K, 0).
\]
\end{frame}

\begin{frame}{Цена call в одном шаге}
\[
C_0 = \frac{1}{1+r\Delta t}
\left(
qC_u + (1-q)C_d
\right).
\]
\end{frame}

\begin{frame}{Схема движения и ценообразования}
\[
S_0
\rightarrow
\begin{cases}
uS_0 & (C_u) \\
dS_0 & (C_d)
\end{cases}
\]
\[
C_0 = \frac{qC_u + (1-q)C_d}{1+r\Delta t}
\]
\end{frame}

\begin{frame}{Идея: портфель, который повторяет опцион}
Предположим, что есть портфель вида:
\[
\Pi = \Delta S_0 - B_0,
\]
который через один шаг даёт выплаты:
\[
\Pi_u = \Delta\,uS_0 - B_0(1+r\Delta t),
\]
\[
\Pi_d = \Delta\,dS_0 - B_0(1+r\Delta t).
\]

Хотим:
\[
\Pi_u = C_u,\qquad \Pi_d = C_d.
\]
Тогда портфель \textbf{реплицирует} опцион.
\end{frame}

\begin{frame}{Решение системы репликации}
Из:
\[
\Delta uS_0 - B_0(1+r\Delta t)=C_u
\]
\[
\Delta dS_0 - B_0(1+r\Delta t)=C_d
\]

вычтем второе из первого:
\[
\Delta S_0(u-d)=C_u - C_d.
\]

Отсюда:
\[
\Delta = \frac{C_u - C_d}{(u-d)S_0}.
\]
\end{frame}

\begin{frame}{Стоимость позиций в долгах}
Подставляем найденную $\Delta$ в уравнение:
\[
\Delta uS_0 - B_0(1+r\Delta t)=C_u.
\]

Тогда:
\[
B_0 = \frac{\Delta uS_0 - C_u}{1+r\Delta t}.
\]
\end{frame}

\begin{frame}{Цена реплицирующего портфеля}
Сегодня стоимость портфеля:
\[
\Pi_0 = \Delta S_0 - B_0.
\]

Так как портфель точно повторяет опцион:
\[
C_0=\Pi_0.
\]
\end{frame}

\begin{frame}{Почему появляется формула с ожиданием}
После пллавных телодвижений получаем:
\[
C_0 = \frac{1}{1+r\Delta t}
\left(
qC_u + (1-q)C_d
\right),
\]
где
\[
q=\frac{(1+r\Delta t)-d}{u-d}.
\]
\end{frame}






\end{document}