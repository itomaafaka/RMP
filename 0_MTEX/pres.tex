\documentclass[aspectratio=169]{beamer}
% Прогресс-бар и нумерация
\usetheme[numbering=fraction,progressbar=frametitle]{metropolis}
\usepackage{graphicx}
\makeatletter
\setbeamertemplate{progress bar in frametitle}{%
  \begin{tikzpicture}[remember picture,overlay]
    % Размер и позиция «капсулы»
    \def\barW{\paperwidth}     % ширина
    \def\barH{0.15cm}          % высота полосы
    \def\barY{-0.32cm}         % смещение вниз внутри area заголовка
    % Прогресс (0..1)
    \pgfmathsetmacro{\p}{\insertframenumber/\inserttotalframenumber}

    % Фон (светлый), рамка и заполнение (оранжевое)
    \path[rounded corners=0.08cm, draw=ClaudeOrange, line width=1pt, fill=ClaudeSand]
      (0,\barY) rectangle (\barW,\barY+\barH);
    \path[rounded corners=0.08cm, fill=ClaudeOrange]
      (0,\barY) rectangle (\p*\barW,\barY+\barH);
  \end{tikzpicture}%
}
\makeatother
\metroset{block=fill}

% --- Язык ---
\usepackage[utf8]{inputenc}
\usepackage[T2A]{fontenc}
\usepackage[russian]{babel}

% --- Шрифты и математика ---
\usepackage[default]{opensans}
\renewcommand{\familydefault}{\sfdefault}
\usefonttheme{professionalfonts} 
\usepackage{amsmath,amssymb}
\usepackage{graphicx}
\usepackage{ragged2e}

% --- Цветовая схема ---
\definecolor{ClaudeOrange1}{HTML}{4967D6}
\definecolor{ClaudeOrange}{HTML}{E39A4C}
\definecolor{ClaudeSand}{HTML}{F6F3EE}
\definecolor{ClaudeDark}{HTML}{3C3328}

\setbeamercolor{background canvas}{bg=ClaudeSand}
\setbeamercolor{normal text}{fg=ClaudeDark}
\setbeamercolor{frametitle}{bg=ClaudeOrange1, fg=ClaudeSand}
\setbeamercolor{structure}{fg=ClaudeOrange}
\setbeamercolor{block title}{fg=white,bg=ClaudeOrange}
\setbeamercolor{block body}{bg=ClaudeSand,fg=ClaudeDark}

% --- tcolorbox для блоков со скруглениями ---
\usepackage{tcolorbox}
\tcbuselibrary{skins,breakable}
\colorlet{BlockBodyBG}{ClaudeSand}
\colorlet{BlockTitleFG}{white}
\colorlet{BlockFrame}{ClaudeOrange}

\setbeamertemplate{block begin}{%
  \begin{tcolorbox}[
      enhanced,
      breakable,
      colback=white,
      colframe=BlockFrame,
      coltitle=BlockTitleFG,
      arc=5pt,                    % радиус скругления (исправлено)
      boxrule=0.1pt,              % толщина рамки
      left=6pt,right=6pt,top=8pt,bottom=8pt, % внутренние отступы
      before skip=8pt,after skip=8pt,        % внешние отступы
      title=\insertblocktitle,
      fonttitle=\bfseries
  ]%
}
\setbeamertemplate{block end}{%
  \end{tcolorbox}%
}

% --- Метаданные презентации ---
\title{\huge{Хэджирование с помощью опционов}}
\author{Выгузов Алексей — ЭКРмд-01-24}
\begin{document}

\begin{frame}[plain]
  \titlepage
\end{frame}

\begin{frame}{План презентации}
    \tableofcontents
\end{frame}

\begin{frame}
    \centering\Large\textbf{0. Банковский счёт, экспонента и интеграл}        
\end{frame}

\begin{frame}{Что такое банковский счёт?}
    \begin{block}{Определение}
        \textbf{Банковский счёт} (безрисковый актив) — это детерминированный процесс $\{B(t)\}_{t\ge 0}$, удовлетворяющий уравнению
        $$
        \frac{dB(t)}{dt} = r(t)\,B(t), \quad r(t) \text{ — известная безрисковая ставка.}
        $$
    \end{block}
Начальное условие принято задавать как $B(0)=1$. Тогда решение имеет вид
\[
B(t) = \exp\!\left(\int_0^t r(s)\,ds\right).
\]
При постоянной ставке $r(t)\equiv r$ получаем классическую формулу:
\[
B(t)=e^{rt}.
\]  
\end{frame}

\begin{frame}{Банковский счёт}
    \textbf{Банковский счёт (безрисковый актив)} определяется как процесс $B(t)$, описывающий эволюцию капитала, размещённого под фиксированную и заранее известную процентную ставку $r$. Он обладает следующими свойствами:
\begin{itemize}
    \item \textbf{Детерминированность} - все значения заранее известны  
    \item \textbf{Отсутствие риска дефолта} - гарантированная выплата
    \item \textbf{Numeraire} — используется для оценки стоимости других активов
    
\end{itemize}

\end{frame}

\begin{frame}{Что означает эта формалу}
Пусть промежуток времени \([0,T]\) разбит на равные отрезки длины \(\Delta t\):
\[
0,\ \Delta t,\ 2\Delta t,\ \dots,\ n\Delta t = T,\qquad 
\Delta t = \frac{T}{n}.
\]
Будем предполагать, что за один малый шаг времени капитал изменяется по правилу
\[
B(t+\Delta t) = B(t)\bigl(1+r\Delta t\bigr).
\]
После \(n\) шагов имеем
\[
B(T) = B(0)\,(1+r\Delta t)^n.
\]
\end{frame}


\begin{frame}{Что означает эта формалу}
    Переходя к пределу при \(\Delta t \to 0\) (то есть при \(n\to\infty\)), получаем
\[
\lim_{\Delta t\to 0} (1+r\Delta t)^{T/\Delta t} = e^{rT}.
\]

Таким образом, в пределе дискретная динамика переходит в дифференциальное уравнение
\[
\frac{dB(t)}{dt}
= \lim_{\Delta t\to 0} 
  \frac{B(t+\Delta t)-B(t)}{\Delta t}
= r B(t).
\]
\end{frame}

\begin{frame}{Пример}
    Пусть на банковском счёте лежит сумма $B(0)=100{\,}000 \text{ рублей},$а годовая ставка равна ключевой ставке ЦБ РФ: $r = 16,5\% = 0.165.$\\
Рассчитаем рост капитала за один год ($T=1$) при разных схемах начисления.
\end{frame}

\begin{frame}{Решение}
1. Простые проценты (линейная модель):
\begin{align*}
B_{1}(1) &= 100{\,}000\left(1+\frac{0.16}{1}\right)^{1} = 100{\,}000 \cdot 1.16 = \bf{116{\,}000} \text{ рублей}.         
\end{align*}
2. Сложные проценты: капитализация раз в год ($n=1$):
\begin{align*}
B_{12}(1) &= 100{\,}000\left(1+\frac{0.16}{12}\right)^{12} \approx 100{\,}000 \cdot 1.17180 \approx \bf{117{\,}180} \text{ рублей}.
\end{align*}
3. Непрерывное начисление ($n\to\infty$):
\[B_{\text{cont}}(1) = 100{\,}000\, e^{0.16} \approx 100{\,}000 \cdot 1.17351 \approx \bf{117{\,}351} \text{ рублей}.\]
\end{frame}

\begin{frame}
    \begin{figure}[h]
    \centering
    \begin{center}
        \includegraphics[width=0.85\textwidth]{bank_exponent.png}
    \end{center}
\end{figure}
\end{frame}

\begin{frame}{Сравнение результатов}
\[
\overbrace{116{\,}000}^{{B_1(1)}}
\;<\;
\overbrace{117{\,}180}^{{B_{12}(1)}}
\;<\;
\overbrace{117{\,}351}^{{B_{\text{cont}}(1)}}
\]

\begin{itemize}
    \item При простой схеме процент не капитализируется — итог меньше всех.
    \item Чем чаще происходит капитализация, тем выше итоговая сумма.
    \item Непрерывный рост даёт теоретический максимум: $e^{r}$.
\end{itemize}
\centering\textbf{Рост ускоряется со временем — \textbf{экспоненциальный эффект}.}
\end{frame}


\end{document}