\documentclass[12pt,a4paper]{article}

\usepackage[utf8]{inputenc}
\usepackage[T2A]{fontenc}
\usepackage[russian]{babel}

% --- Шрифты и математика ---
\usepackage[default]{opensans}
\renewcommand{\familydefault}{\sfdefault}
\usepackage{amsmath,amssymb}
\usepackage{graphicx}
\usepackage{ragged2e}


\usepackage{amsmath,amssymb}

\usepackage{geometry}
\geometry{margin=1.5cm}
\linespread{1}
\usepackage{enumitem}

\begin{document}

\section*{Банковский счёт, экспонента и интеграл}


Банковский счёт (безрисковый актив) — это детерминированный процесс 
$\{B(t)\}_{t\ge 0}$, удовлетворяющий уравнению
\[
\frac{dB(t)}{dt} = r(t)\,B(t),
\]
где $r(t)$ — известная безрисковая ставка. Начальное условие принято задавать
как $B(0)=1$. Тогда решение имеет вид
\[
B(t) = \exp\!\left(\int_0^t r(s)\,ds\right).
\]
При постоянной ставке $r(t)\equiv r$ получаем классическую формулу:
\[
B(t)=e^{rt}.
\]

Банковский счёт — самый простой актив в финансах, который растёт полностью детерминированно: если положить деньги под ставку $r$, мы заранее знаем, сколько получим через любое время $t$.

Именно поэтому банковский счёт используется как «линейка стоимости»:  все рискованные активы сравниваются с ним. Если актив растёт быстрее  $e^{rt}$ — это риск, если медленнее — это невыгодно, а равенство означает полное отсутствие риска. Все модели хеджирования и оценки опционов строятся относительно банковского счёта. Это безрисковый актив, относительно которого мы впоследствии будем измерять стоимость любых  рискованных инструментов: акций, облигаций, фьючерсов и, конечно, опционов.

Предположим, что ставка процента равна $r$. Тогда за малый промежуток времени $\Delta t$ баланс на счёте изменяется по простой формуле:
\[
B(t + \Delta t) = B(t)\,\bigl(1 + r\,\Delta t\bigr).
\]
Пусть промежуток времени \([0,T]\) разбит на равные отрезки длины \(\Delta t\):
\[
0,\ \Delta t,\ 2\Delta t,\ \dots,\ n\Delta t = T,\qquad 
\Delta t = \frac{T}{n}.
\]
Будем предполагать, что за один малый шаг времени капитал изменяется по правилу
\[
B(t+\Delta t) = B(t)\bigl(1+r\Delta t\bigr).
\]
После \(n\) шагов имеем
\[
B(T) = B(0)\,(1+r\Delta t)^n.
\]
Переходя к пределу при \(\Delta t \to 0\) (то есть при \(n\to\infty\)), получаем
\[
\lim_{\Delta t\to 0} (1+r\Delta t)^{T/\Delta t} = e^{rT}.
\]

Таким образом, в пределе дискретная динамика переходит в дифференциальное уравнение
\[
\frac{dB(t)}{dt}
= \lim_{\Delta t\to 0} 
  \frac{B(t+\Delta t)-B(t)}{\Delta t}
= r B(t).
\]

То есть за каждый маленький интервал времени счёт увеличивается на пропорциональную долю от текущего уровня. Но в реальности проценты начисляются непрерывно, а не рывками. Поэтому мы  уменьшаем длину интервала $\Delta t$ всё сильнее, стремим её к нулю и смотрим на поведение процесса в предельном смысле. Это уравнение выражает идею: скорость роста банковского счёта в любой момент времени пропорциональна его текущему значению. 

Пусть $B(0)=100$ и ставка $r = 10\%$ в год.
\[B_{1}(1) = 100{\,}000\left(1+\frac{0.16}{1}\right)^{1} = 100{\,}000 \cdot 1.16 = \bf{116{\,}000} \text{ рублей}. \]
2. Сложные проценты: капитализация раз в год ($n=1$):
\[B_{12}(1) = 100{\,}000\left(1+\frac{0.16}{12}\right)^{12} \approx 100{\,}000 \cdot 1.17180 \approx \bf{117{\,}180} \text{ рублей}.\]
3. Непрерывное начисление ($n\to\infty$):
\[B_{\text{cont}}(1) = 100{\,}000\, e^{0.16} \approx 100{\,}000 \cdot 1.17351 \approx \bf{117{\,}351} \text{ рублей}.\]

Важно: рост ускоряется со временем — именно это и есть \textbf{экспоненциальный эффект}.

Даже в \textbf{самой простой} финансовой конструкции — банковском  счёте — уже присутствуют экспонента и интеграл. Это означает, что без инструментов анализа финансовая математика невозможна в принципе. Почему это так критично? Потому что банковский счёт играет роль \textbf{эталона} для всех остальных активов. Рискованные активы — акции, деривативы, валюты — оцениваются относительно этого детерминированного, предсказуемо растущего инструмента.

И далее вся архитектура финансовой математики строится на следующей идее: существует актив, который растёт как $e^{rt}$, и мы сравниваем с ним поведение всех других инструментов. Это фундамент для понятий \textbf{дисконтирования}, \textbf{отсутствия арбитража} и, в конечном счёте, для механики \textbf{хеджирования опциона}.

\section*{1. Случайность, риск и зачем тут математика}

Представим, что у нас есть акция, которая сегодня стоит 100 рублей.Что будет с этой ценой завтра?\\

На самом деле вариантов бесконечно много: 99, 101, 98.5, 105 — любое число, и заранее мы его не знаем.\\

Важно понять: даже если у нас есть прогнозы, аналитики, графики, новости — рынок всё равно остаётся случайным. Каждое завтрашнее значение цены — это реализация некоторого случайного исхода, который мы можем описать только вероятностно.\\

В финансах мы никогда не спрашиваем:«Какая завтра будет цена?» Это вопрос без смысла.\\

Мы спрашиваем:«Какие ценовые сценарии возможны? Каковы их вероятности?»\\

Потому что именно вероятности определяют: насколько рискован актив, сколько мы можем потерять, как нам строить хеджирующую стратегию.
И вот здесь появляется ключевая идея: цена актива — это случайная величина, а будущее рынка — это множество возможных исходов, которыми мы должны уметь работать.

%%%%%%%%%

В финансах решения всегда принимаются заранее. Если я покупаю актив сегодня — я фиксирую цену сегодня, но результат увижу только завтра или через месяц.

Проблема в том, что будущее неизвестно. Цена может вырасти, может упасть, может остаться прежней.

Поэтому любое решение — купить, продать, держать, хеджировать — всегда делается в условиях неопределённости.

И вот здесь возникает ключевое понятие: риск.

Риск — это не обязательно “потеря денег”.
Риск — это ситуация, в которой результат моего решения зависит от того, какой именно исход реализуется в будущем.

Покупая актив, я фактически делаю ставку на то, что завтра реализуется благоприятный сценарий. Но поскольку существует и неблагоприятный сценарий, я автоматически получаю риск.

Отсюда следует очень практический вывод:

Чтобы принимать осмысленные финансовые решения, нужно понимать распределение будущих исходов.

Без этого невозможно:
\begin{itemize}
    \item оценить убытки,
    \item сравнить инструменты,
    \item построить стратегию,
    \item либо управлять рисками.
\end{itemize}

фундамент мотивации для моделей, которые мы дальше будем строить: биномиальная модель, риск-нейтральная мера, дельта-хедж — всё это способы управлять последствиями будущей неопределённости.

двлее тервер
 
\section*{terver}



\end{document}