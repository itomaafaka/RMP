\documentclass[12pt,a4paper]{article}

\usepackage[utf8]{inputenc}
\usepackage[T2A]{fontenc}
\usepackage[russian]{babel}

% --- Шрифты и математика ---
\usepackage[default]{opensans}
\renewcommand{\familydefault}{\sfdefault}
\usepackage{amsmath,amssymb}
\usepackage{graphicx}
\usepackage{ragged2e}


\usepackage{amsmath,amssymb}

\usepackage{geometry}
\geometry{margin=1.5cm}
\linespread{1}
\usepackage{enumitem}

\begin{document}

\section*{Банковский счёт, экспонента и интеграл}


Банковский счёт (безрисковый актив) — это детерминированный процесс 
$\{B(t)\}_{t\ge 0}$, удовлетворяющий уравнению
\[
\frac{dB(t)}{dt} = r(t)\,B(t),
\]
где $r(t)$ — известная безрисковая ставка. Начальное условие принято задавать
как $B(0)=1$. Тогда решение имеет вид
\[
B(t) = \exp\!\left(\int_0^t r(s)\,ds\right).
\]
При постоянной ставке $r(t)\equiv r$ получаем классическую формулу:
\[
B(t)=e^{rt}.
\]

Банковский счёт — самый простой актив в финансах, который растёт полностью детерминированно: если положить деньги под ставку $r$, мы заранее знаем, сколько получим через любое время $t$.

Именно поэтому банковский счёт используется как «линейка стоимости»:  все рискованные активы сравниваются с ним. Если актив растёт быстрее  $e^{rt}$ — это риск, если медленнее — это невыгодно, а равенство означает полное отсутствие риска. Все модели хеджирования и оценки опционов строятся относительно банковского счёта. Это безрисковый актив, относительно которого мы впоследствии будем измерять стоимость любых  рискованных инструментов: акций, облигаций, фьючерсов и, конечно, опционов.

Предположим, что ставка процента равна $r$. Тогда за малый промежуток времени $\Delta t$ баланс на счёте изменяется по простой формуле:
\[
B(t + \Delta t) = B(t)\,\bigl(1 + r\,\Delta t\bigr).
\]
Пусть промежуток времени \([0,T]\) разбит на равные отрезки длины \(\Delta t\):
\[
0,\ \Delta t,\ 2\Delta t,\ \dots,\ n\Delta t = T,\qquad 
\Delta t = \frac{T}{n}.
\]
Будем предполагать, что за один малый шаг времени капитал изменяется по правилу
\[
B(t+\Delta t) = B(t)\bigl(1+r\Delta t\bigr).
\]
После \(n\) шагов имеем
\[
B(T) = B(0)\,(1+r\Delta t)^n.
\]
Переходя к пределу при \(\Delta t \to 0\) (то есть при \(n\to\infty\)), получаем
\[
\lim_{\Delta t\to 0} (1+r\Delta t)^{T/\Delta t} = e^{rT}.
\]

Таким образом, в пределе дискретная динамика переходит в дифференциальное уравнение
\[
\frac{dB(t)}{dt}
= \lim_{\Delta t\to 0} 
  \frac{B(t+\Delta t)-B(t)}{\Delta t}
= r B(t).
\]

То есть за каждый маленький интервал времени счёт увеличивается на пропорциональную долю от текущего уровня. Но в реальности проценты начисляются непрерывно, а не рывками. Поэтому мы  уменьшаем длину интервала $\Delta t$ всё сильнее, стремим её к нулю и смотрим на поведение процесса в предельном смысле. Это уравнение выражает идею: скорость роста банковского счёта в любой момент времени пропорциональна его текущему значению. 

Пусть $B(0)=100$ и ставка $r = 10\%$ в год.
\[B_{1}(1) = 100{\,}000\left(1+\frac{0.16}{1}\right)^{1} = 100{\,}000 \cdot 1.16 = \bf{116{\,}000} \text{ рублей}. \]
2. Сложные проценты: капитализация раз в год ($n=1$):
\[B_{12}(1) = 100{\,}000\left(1+\frac{0.16}{12}\right)^{12} \approx 100{\,}000 \cdot 1.17180 \approx \bf{117{\,}180} \text{ рублей}.\]
3. Непрерывное начисление ($n\to\infty$):
\[B_{\text{cont}}(1) = 100{\,}000\, e^{0.16} \approx 100{\,}000 \cdot 1.17351 \approx \bf{117{\,}351} \text{ рублей}.\]

Важно: рост ускоряется со временем — именно это и есть \textbf{экспоненциальный эффект}.

Даже в \textbf{самой простой} финансовой конструкции — банковском  счёте — уже присутствуют экспонента и интеграл. Это означает, что без инструментов анализа финансовая математика невозможна в принципе. Почему это так критично? Потому что банковский счёт играет роль \textbf{эталона} для всех остальных активов. Рискованные активы — акции, деривативы, валюты — оцениваются относительно этого детерминированного, предсказуемо растущего инструмента.

И далее вся архитектура финансовой математики строится на следующей идее: существует актив, который растёт как $e^{rt}$, и мы сравниваем с ним поведение всех других инструментов. Это фундамент для понятий \textbf{дисконтирования}, \textbf{отсутствия арбитража} и, в конечном счёте, для механики \textbf{хеджирования опциона}.

\section*{1. Случайность, риск и зачем тут математика}

Представим, что у нас есть акция, которая сегодня стоит 100 рублей.Что будет с этой ценой завтра?\\

На самом деле вариантов бесконечно много: 99, 101, 98.5, 105 — любое число, и заранее мы его не знаем.\\

Важно понять: даже если у нас есть прогнозы, аналитики, графики, новости — рынок всё равно остаётся случайным. Каждое завтрашнее значение цены — это реализация некоторого случайного исхода, который мы можем описать только вероятностно.\\

В финансах мы никогда не спрашиваем:«Какая завтра будет цена?» Это вопрос без смысла.\\

Мы спрашиваем:«Какие ценовые сценарии возможны? Каковы их вероятности?»\\

Потому что именно вероятности определяют: насколько рискован актив, сколько мы можем потерять, как нам строить хеджирующую стратегию.
И вот здесь появляется ключевая идея: цена актива — это случайная величина, а будущее рынка — это множество возможных исходов, которыми мы должны уметь работать.

%%%%%%%%%

В финансах решения всегда принимаются заранее. Если я покупаю актив сегодня — я фиксирую цену сегодня, но результат увижу только завтра или через месяц.

Проблема в том, что будущее неизвестно. Цена может вырасти, может упасть, может остаться прежней.

Поэтому любое решение — купить, продать, держать, хеджировать — всегда делается в условиях неопределённости.

И вот здесь возникает ключевое понятие: риск.

Риск — это не обязательно “потеря денег”.
Риск — это ситуация, в которой результат моего решения зависит от того, какой именно исход реализуется в будущем.

Покупая актив, я фактически делаю ставку на то, что завтра реализуется благоприятный сценарий. Но поскольку существует и неблагоприятный сценарий, я автоматически получаю риск.

Отсюда следует очень практический вывод:

Чтобы принимать осмысленные финансовые решения, нужно понимать распределение будущих исходов.

Без этого невозможно:
\begin{itemize}
    \item оценить убытки,
    \item сравнить инструменты,
    \item построить стратегию,
    \item либо управлять рисками.
\end{itemize}

фундамент мотивации для моделей, которые мы дальше будем строить: биномиальная модель, риск-нейтральная мера, дельта-хедж — всё это способы управлять последствиями будущей неопределённости.

двлее тервер
 
\section*{terver}

Сейчас я покажу вам очень важный момент.
То, что мы называем «случайностью рынка», можно формализовать в три шага.
Настолько просто и быстро, что после этого вы увидите, что вся теория опционов — это на самом деле аккуратная работа с тремя объектами.

Шаг 1. Множество исходов — $\Omega$. Это все возможные сценарии рынка завтра.Это пространство всех будущих «всех миров», которые может выбрать рынок.

Шаг 2. События — $\mathcal{F}$. Это не отдельные исходы, а наборы исходов, которые нас интересуют.

\begin{enumerate}
    \item рынок вырастет → \{\text{up}\};
    \item рынок изменится → \{\text{up},\text{down}\}; 
    \item рынок изменится → \{\text{up},\text{down}\};
\end{enumerate}

Именно такие объединения образуют $\sigma$-алгебру. Почему $\sigma$? Потому что события должны быть замкнуты относительно счетных объединений, пересечений и дополнений. То есть: если мы можем наблюдать события A и B, то мы обязаны иметь возможность наблюдать.

Шаг 3. Вероятность — $P$.
Теперь каждому событию мы назначаем число от $0$ до $1$ — «вес» сценария.
Например:
\[P(\text{up}) = 0.3,\quad
P(\text{down}) = 0.4,\quad
P(\text{flat}) = 0.3.
\]
Это и есть мера вероятности.

\section*{Почему банкам нужно хеджироваться}

Теперь, когда мы увидели, что цены — случайные, а решения принимаются заранее, давайте посмотрим, почему именно банк — не инвестор, не спекулянт — должен хеджировать риски.

Банк работает с большим количеством рыночных факторов:
процентные ставки, валюты, фондовый рынок, сырьевые активы.
Каждый из этих факторов меняется случайно, и каждый напрямую влияет на стоимость позиций банка.

Особенно важен случай с опционами.
Опцион — это инструмент с нелинейным профилем выплаты.
Когда банк продаёт клиенту опцион, он получает премию сегодня, но берёт на себя обязательство, результат которого будет сильно зависеть от будущей цены — да ещё и нелинейно.

Из-за этой нелинейности маленькое движение цены может сильно изменить P\&L банка.
Получается, что даже если базовый актив меняется чуть-чуть, итоговый результат по проданному опциону может меняться очень резко.

Поэтому банку нужно управлять чувствительностью портфеля к движению рынка — дельтой, гаммой, вегой.
И самый базовый инструмент — дельта-хеджирование, о котором мы будем говорить дальше.

\section*{Что значит хеджироваться}

Хеджирование — это не попытка угадать рынок и не способ получить прибыль, а  инструмент, который уменьшает влияние случайности на результат.

Когда у банка есть позиция, которая чувствительна к цене базового актива, банк может открыть корректирующую позицию — такую, которая компенсирует эту чувствительность.

Идея очень простая:
\begin{enumerate}
    \item есть риск
	\item добавляем позицию, которая ведёт себя противоположно
	\item риск уменьшается или исчезает
\end{enumerate}

Цель хеджирования — не максимизация прибыли, а стабилизация итогового результата, чтобы P\&L не прыгал из-за движения рынка.

В контексте опционов это означает: если опцион даёт нелинейный риск, мы должны подобрать комбинацию базовых активов так, чтобы его нейтрализовать. В простейшем случае — это дельта-хедж.


\section*{Опционы: определения, логика, формулы и примеры}
\subsection*{Что такое опцион}

Опцион — это финансовый контракт, который даёт его владельцу \textbf{право}, но не обязанность,совершить определённое действие в будущем:
\begin{itemize}
    \item либо \textbf{купить} базовый актив по фиксированной цене,
    \item либо \textbf{продать} базовый актив по фиксированной цене.
\end{itemize}

Это ключевая характеристика: покупатель опциона принимает решение \emph{после того, как узнает рыночную цену}. Если выгодно — он использует право. Если нет — просто отказывается.

Продавец опциона, наоборот, берёт на себя \textbf{обязательство} выполнить решение покупателя. Поэтому покупатель всегда платит премию, а продавец её получает.

\subsection*{Страйк и дата экспирации}

\textbf{Страйк (K)} — это заранее установленная цена сделки, по которой действует право опциона.
\textbf{Экспирация (T)} — момент времени, когда право может быть использовано (европейский стиль). В момент $T$ определяется конечная выплата опциона.
Таким образом, опцион — это право совершить операцию по цене $K$ в момент времени $T$,
когда уже известно фактическое значение цены $S_T$.

\subsection*{3.3. Два базовых типа: Call и Put}

\textbf{Call-опцион} — право \emph{купить} актив по цене $K$.
\begin{itemize}
    \item выгоден, если $S_T > K$;
    \item не нужен, если $S_T \le K$.
\end{itemize}

\textbf{Put-опцион} — право \emph{продать} актив по цене $K$.
\begin{itemize}
    \item выгоден, если $S_T < K$;
    \item не используется, если $S_T \ge K$.
\end{itemize}

всего лишь фундаментальные конструкции, но вся финансовая инженерия построена на их комбинациях.

\subsection*{Что значит Long и Short в опционах}

Здесь важно не перепутать: long и short в опционах — это не владение активом, а владение \textbf{правом} или \textbf{обязательством}.

\begin{itemize}
    \item \textbf{Long Call}: вы покупаете право купить. Платите премию.
    \item \textbf{Short Call}: вы продаёте право купить. Получаете премию.
    \item \textbf{Long Put}: вы покупаете право продать. Платите премию.
    \item \textbf{Short Put}: вы продаёте право продать. Получаете премию.
\end{itemize}

Покупатель (long) имеет право и ограниченный риск.  
Продавец (short) имеет обязательство и потенциально неограниченный риск.

\subsection*{3.5. Логика вывода формул payoff}

Теперь аккуратно выведём все формулы выплат, не просто запоминая их, а понимая, как они рождаются из логики принятия решений.

\subsubsection*{Call}

Удержатель call-опциона имеет право купить актив по $K$. Рассмотрим возможные действия в зависимости от цены $S_T$:

\[
\begin{cases}
\text{если } S_T \le K,\ \text{право не нужно} \Rightarrow \text{выплата } 0, \\
\text{если } S_T > K,\ \text{покупаем по } K\ \text{и продаём по } S_T \Rightarrow S_T - K.
\end{cases}
\]

Это кусочная функция, которую компактно записывают как
\[
\text{Payoff}_{\text{call}} = \max(S_T - K, 0).
\]

\subsubsection*{Put}
Удержатель put имеет право продать по $K$:
\[
\begin{cases}
\text{если } S_T \ge K,\ \text{право не нужно} \Rightarrow \text{выплата } 0, \\
\text{если } S_T < K,\ \text{продаём по } K\ \text{то, что стоит } S_T \Rightarrow K - S_T.
\end{cases}
\]

Компактная запись:
\[
\text{Payoff}_{\text{put}} = \max(K - S_T, 0).
\]

% ---------------------------------------------------------
\subsection*{Payoff формулы (long/short)}

Теперь запишем четыре фундаментальные позиции:

\[
\begin{aligned}
\text{Payoff}_{\text{Long Call}} &= \max(S_T - K, 0),\\[4pt]
\text{Payoff}_{\text{Short Call}} &= -\max(S_T - K, 0),\\[4pt]
\text{Payoff}_{\text{Long Put}} &= \max(K - S_T, 0),\\[4pt]
\text{Payoff}_{\text{Short Put}} &= -\max(K - S_T, 0).
\end{aligned}
\]
Минус означает обязательство выплатить то, что получает покупатель. 

\subsection*{Расчёт payoff для всех четырёх позиций}

Пусть базовый актив имеет возможные цены на экспирации:
\[
S_T \in \{80,\ 100,\ 130\}.
\]
Возьмём страйк $K = 100$. Тогда для каждого сценария вычислим выплаты:
\[
\begin{array}{c|c|c|c|c}
S_T & \text{Long Call} & \text{Short Call} & \text{Long Put} & \text{Short Put} \\ \hline
80  & 0   & 0   & 20  & -20 \\
100 & 0   & 0   & 0   & 0   \\
130 & 30  & -30 & 0   & 0
\end{array}
\]

\begin{itemize}
    \item При $S_T=80$ call-опционы бесполезны, а put дают выгоду $100-80=20$.
    \item При $S_T=100$ оба типа опционов на границе — payoff равен нулю.
    \item При $S_T=130$ call даёт $130-100=30$, а put бесполезен.
\end{itemize}

Эта таблица показывает:
\begin{itemize}
    \item где появляется выгода,
    \item где появляется обязательство,
    \item почему формулы именно такие.
    \item почему продавец опциона несёт потенциально большой риск,
\end{itemize}

\section*{Зачем? Переход к binomial}

В предыдущем блоке мы увидели, что структура выплат по опционам — это кусочные функции:они резко меняются при переходе цены через страйк.  
Особенно это важно для продавца call и put: его P\&L может быть крайне чувствительным к изменениям цены базового актива.

Возникает естественный вопрос:\\
\textbf{Как вообще меняется цена базового актива и как описать её движение математически?}

Чтобы рассчитывать стоимость опциона или хеджировать проданную позицию, нам нужно понять:
\begin{itemize}
    \item какие движения цены возможны;
    \item насколько сильно они могут отличаться от текущего уровня;
    \item как на эти движения реагирует payoff опциона;
    \item можно ли собрать портфель, который «компенсирует» эти колебания.
\end{itemize}

Полная непрерывная модель (вроде модели Блэка--Шоулза) требует дифференциальных уравнений и стохастического анализа. Но существует простая модель, с которой начинается вся финансовая математика, — \textbf{биномиальная модель движений цены}.  

Она достаточна для:
\begin{itemize}
    \item понимания природы риска;
    \item объяснения хеджирования;
    \item построения портфеля-репликатора;
    \item расчёта цены опциона без арбитража;
    \item объяснения дельты как коэффициента хеджа.
\end{itemize}

И самое главное — биномиальная модель даёт первый пример, где:
\[
\textbf{в теории мы можем (практически) полностью устранить риск продавца опциона.}
\]


ключевая идею: любая модель финансового инструмента должна быть\textbf{арбитражно-свободной}.  
Арбитраж — это стратегия, которая приносит прибыль \textbf{без риска и без вложений}.  Если модель допускает арбитраж, то цена опциона, которую мы получим из неё, будет бессмысленной: в реальном рынке такую возможность мгновенно вынесут.

В одном шаге у нас есть три возможные доходности:
\[
d,\qquad 1+r\Delta t,\qquad u,
\]
которые соответствуют:
\begin{itemize}
    \item доходности актива при движении вниз ($d$);
    \item доходности банковского счёта ($1+r\Delta t$);
    \item доходности актива при движении вверх ($u$).
\end{itemize}
Условие отсутствия арбитража в биномиальной модели формулируется как:
\[
d < 1+r\Delta t < u.
\]
Интуиция:
\begin{itemize}
    \item Если $1+r\Delta t \le d$, то актив всегда растёт медленнее банка.
    Тогда выгодно шортить актив и класть деньги в банк. Получается прибыль без риска.
    \item Если $1+r\Delta t \ge u$, то актив всегда растёт быстрее банка.
    Тогда выгодно занимать под $r$ и покупать актив. Опять бесплатная прибыль.
\end{itemize}

Когда банковская ставка между движениями вверх и вниз, у нас не возникает гарантированного выигрыша. Это и означает, что модель согласована с реальностью и пригодна для дальнейших расчётов.

\subsection*{2. Риск-нейтральная вероятность}
мы НЕ пытаемся угадать реальные рыночные вероятности. Вместо этого мы ищем такую вероятность $q$, при которой \textbf{дисконтированная цена актива становится мартингалом}.

Риск-нейтральная вероятность возникает \textbf{не из статистики}, а из чисто математического требования: не должно быть арбитража
Она имеет форму:
\[
q = \frac{(1+r\Delta t)-d}{u-d}.
\]
$q$ — не прогноз и не вероятность роста по мнению рынка это искусственная техническая вероятность, созданная для удобства, делает модель арбитражно-свободной, она позволяет напрямую вычислять стоимость опциона.

ключевая мысля: в риск-нейтральном мире ожидаемая доходность актива совпадает с безрисковой:
\[
\mathbb{E}^{\mathbb{Q}}[S_1] = S_0(1+r\Delta t).
\]

То есть актив ведёт себя так, будто он растёт как банковский счёт.

\textbf{Строгое определение:}  
Процесс $X_t$ называется мартингалом относительно фильтрации $\mathcal{F}_t$ и вероятности $P$, если:
\[
\mathbb{E}[X_{t+1} \mid \mathcal{F}_t] = X_t.
\]

\textbf{Интуитивно:}  
мартингал — это случайный процесс, у которого отсутствует направленный тренд. Текущее значение — это лучшая возможная оценка будущего.

\textbf{Пример:}
\begin{itemize}
    \item Честная игра: капитал игрока после каждой ставки, если ожидание равно нулю.
    \item Дисконтированная цена актива в риск-нейтральном мире.
\end{itemize}

В биномиальной модели дисконтированная цена актива:
\[
\tilde{S}_t = \frac{S_t}{B_t}
\]
становится мартингалом при вероятности $q$:
\[
\mathbb{E}^{\mathbb{Q}}[\tilde{S}_1 \mid S_0] = \tilde{S}_0.
\]

\subsection*{Цена call в одном шаге}

Теперь, когда у нас есть риск-нейтральная вероятность $q$, мы можем вычислить справедливую цену опциона.

Пусть через один шаг:
\[
C_u = \text{выплата call при движении вверх},\qquad
C_d = \text{выплата call при движении вниз}.
\]

Тогда справедливая цена — это \textbf{дисконтированное риск-нейтральное ожидание}:
\[
C_0 = \frac{1}{1+r\Delta t}\bigl(q C_u + (1-q)C_d\bigr).
\]
\begin{itemize}
    \item это не угадывание цены;
    \item это единственное значение, при котором нельзя построить арбитраж;
    \item формула является дискретным аналогом формулы Блэка--Шоулза;
    \item это фундаментальное равенство всего современного деривативного рынка.
\end{itemize}
тогда логически замкнуть это можно так 
\begin{itemize}
    \item известны $u$ и $d$ (движения цены),
    \item известен банковский счёт,
    \item есть условие ``нет арбитража'',
    \item найдена риск-нейтральная вероятность,
    \item получена цена опциона.
\end{itemize}

\end{document}