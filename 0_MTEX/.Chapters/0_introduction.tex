\begin{center}
    \specialsection{Введение}
\end{center}

{Портфельное инвестирование занимает ключевое место в современной финансовой теории и практике. Оно позволяет решать базовые задачи инвестора — диверсифицировать риски, обеспечивать стабильность доходов и управлять капиталом в условиях неопределённости. В то же время именно риски формируют главную проблему портфельного подхода: изменение рыночных факторов может существенно скорректировать стоимость активов и ожидаемую доходность.

Особый интерес представляет рынок облигаций, играющий центральную роль в финансовой системе России. Государственные облигации (ОФЗ) обеспечивают финансирование бюджетного дефицита, тогда как корпоративные выпуски служат инструментом привлечения капитала компаниями реального сектора. По данным Минфина России, объём внутреннего государственного долга на август 2025 г. составляет десятки триллионов рублей \cite{Minfin2025}. Долговой сегмент отличается высокой ликвидностью и значительными объёмами торгов, что подтверждается отчётами Московской биржи \cite{MOEX2024}.

Важность облигационного рынка для национальной экономики проявляется и в его тесной связи с денежно-кредитной политикой. Резкие изменения ключевой ставки Банка России — до 17\% в декабре 2014 г., до 20\% в феврале 2022 г., кратковременный подъём до 21\% и последующее снижение до 18\% к сентябрю 2025 г. \cite{CBRrate}\cite{CBR2024} — наглядно демонстрируют, насколько стоимость облигаций чувствительна к процентным шокам. Эти эпизоды подтверждают центральное значение процентного риска для динамики портфеля долговых инструментов.

Процентный риск выражает чувствительность стоимости облигаций к изменению ставок, поскольку цена облигации напрямую зависит от выбранной ставки дисконтирования. Таким образом, исследование процентного риска становится фундаментальным элементом анализа рисков портфельного инвестирования.

Объектом исследования в работе выступает облигационный портфель, включающий государственные и корпоративные долговые инструменты Российской Федерации. Предметом исследования являются методы количественной оценки процентного риска, основанные на математическом моделировании и статистическом анализе.

Цель работы заключается в разработке воспроизводимого подхода к измерению и моделированию процентного риска облигационного портфеля на российском рынке, который сочетает классические аналитические показатели и современные методы математической статистики и стохастического моделирования.

Для достижения поставленной цели решаются следующие задачи:
\begin{enumerate}
\vspace{-0.2\baselineskip}
\item выполнить обзор классических методов анализа процентного риска и выявить их ограничения;
\vspace{-0.5\baselineskip}
\item построить теоретическую базу для определения риск-метрик;
\vspace{-0.5\baselineskip}
\item формализовать и рассчитать показатели Value-at-Risk (VaR) и Conditional Value-at-Risk (CVaR);
\vspace{-0.5\baselineskip}
\item разработать и применить статистические модели динамики доходностей и волатильности (ARIMA/ARIMAX, GARCH/EGARCH);
\vspace{-0.5\baselineskip}
\item построить и диагностировать кривую доходности, провести факторизацию динамики методом главных компонент (PCA);
\vspace{-0.5\baselineskip}
\item выполнить стресс-тестирование портфеля с использованием дюрации и выпуклости, факторных сценариев и исторических эпизодов;
\vspace{-0.5\baselineskip}
\item сравнить статистические и стохастические модели процентных ставок и оценить их применимость к российским данным;
\vspace{-0.5\baselineskip}
\item разработать модуль машинного обучения для предсказания нарушений VaR на основе логистической регрессии и сопоставить его эффективность с традиционными методами.
\end{enumerate}

Научная новизна исследования заключается в интеграции различных методов оценки процентного риска в единую систему, сочетающую теоретическую строгость и прикладную направленность. Практическая значимость работы состоит в том, что полученные модели и риск-метрики могут быть непосредственно использованы в практике риск-менеджмента российских финансовых институтов.

}
\clearpage