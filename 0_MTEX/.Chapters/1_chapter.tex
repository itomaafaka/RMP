\section{ Теоретическая часть}
\subsection{Риски портфельного инвестирования}

Портфельное инвестирование всегда сопряжено с неопределённостью, поскольку стоимость активов и их доходность зависят от множества факторов. В финансовой теории выделяют несколько основных категорий рисков.

Рыночный риск связан с изменением цен и доходностей финансовых инструментов под влиянием динамики процентных ставок, валютных курсов и котировок ценных бумаг.

Кредитный риск отражает вероятность дефолта эмитента или ухудшения его кредитного качества, что напрямую влияет на цену корпоративных облигаций и величину спреда к безрисковой ставке.

Риск ликвидности проявляется в ограниченной возможности оперативно реализовать позицию без значительных потерь, особенно на сегментах с невысокой торговой активностью.

Валютный риск возникает при вложениях в инструменты, номинированные в иностранной валюте, и связан с колебаниями курсов.

Операционный риск охватывает ошибки внутренних процессов, технологические сбои и форс-мажорные обстоятельства, влияющие на исполнение сделок.

Каждый из перечисленных рисков имеет значение для управления портфелем, однако в случае облигаций центральное место занимает процентный риск. Его специфика состоит в том, что цена облигации определяется дисконтированием будущих денежных потоков, и любое изменение ставок приводит к изменению текущей стоимости инструмента. Для государственных бумаг процентный риск фактически совпадает с рыночным: при росте ставок цены ОФЗ снижаются. Для корпоративных выпусков процентный риск сочетается с кредитным спредом, но именно чувствительность к динамике кривой доходности остаётся ключевым фактором.

Актуальность процентного риска в России подтверждается эпизодами резких изменений ключевой ставки Банка России: 17\% в декабре 2014 г., 20\% в феврале 2022 г., 21\% в середине 2025 г. и последующее снижение до 18\% к сентябрю того же года. Эти шоки существенно влияли на переоценку портфелей облигаций и определяли поведение инвесторов. Процентный риск можно рассматривать как универсальный элемент анализа для любых долговых инструментов.

В обобщённом виде классификация рисков и соответствующих метрик приведена в таблице~\ref{tab:risk_classification}.
\vspace{-0.5\baselineskip}
\begin{table}[H]
\centering
\caption{Основные виды рисков портфельного инвестирования}
\label{tab:risk_classification}
\vspace{-0.5\baselineskip}
\begin{tabular}{m{3cm}m{12.55cm}}
\toprule
\textbf{Вид риска} & \textbf{Механизм проявления}\\\midrule
Рыночный & Изменение цен и доходностей под воздействием процентных ставок, валютных курсов и котировок финансовых инструментов\\ \hline
Кредитный & Возможность дефолта эмитента или ухудшения его кредитного качества, отражающаяся в цене облигации и величине спреда\\ \hline
Ликвидности & Ограничения на быструю продажу актива без существенной скидки из-за недостаточной глубины рынка\\ \hline
Валютный & Колебания курсов валют, влияющие на стоимость активов, номинированных в иностранной валюте\\ \hline
Операционный & Ошибки внутренних процессов, технологические сбои или внешние форс-мажорные обстоятельства\\ \hline
Процентный & Изменение стоимости облигаций вследствие колебаний процентных ставок и трансформации кривой доходности
\\ \bottomrule\end{tabular}
\end{table}
\vspace{-0.5\baselineskip}
Именно для облигационного портфеля процентный риск можно рассматривать обособленно. В ценообразовании долговых инструментов, помимо того что он играет фундаментальную роль, он ещё имеет практическую значимость для анализа устойчивости портфеля в условиях меняющихся макроэкономических и денежно-кредитных режимов.
\vspace{1.5pt}
\subsection{Классические портфельные теории}
Современная теория портфельного инвестирования берёт своё начало с работ Гарри Марковица, результатом которых стало обоснованное предположение, что риск портфеля зависит не только от рисков отдельных активов, но и от корреляции между ними. Так если $w^{\top}=(w_1,w_2,\dots,w_N)$ – транспонированный вектор весов активов в заданном портфеле, а $\Sigma$ -- ковариационная матрица весов активов, то для $N$ количества активов матрица будет иметь вид: 
\[
\Sigma =
\begin{bmatrix}
\sigma_1^2 & \operatorname{Cov}(r_1,r_2) & \cdots & \operatorname{Cov}(r_1,r_N) \\
\operatorname{Cov}(r_2,r_1) & \sigma_2^2 & \cdots & \operatorname{Cov}(r_2,r_N) \\
\vdots & \vdots & \ddots & \vdots \\
\operatorname{Cov}(r_N,r_1) & \operatorname{Cov}(r_N,r_2) & \cdots & \sigma_N^2
\end{bmatrix}
\]

На главной диагонали расположены индивидуальные риски активов $\sigma_i^2$, а вне диагонали — ковариации $\mathrm{Cov}(r_i,r_j)$, отражающие взаимосвязь доходностей, соответственно дисперсия портфеля будет вычисляться вычисляется как:
\[
\sigma_p^2=[w_1,\dots,w_N]\cdot\Sigma \; \left[
\begin{matrix}
w_1 \\ \vdots \\ w_N
\end{matrix}
\right]=w^{\top}\cdot \Sigma\;w
\]

Риск портфеля определяется не просто суммой индивидуальных рисков, а с учётом их взаимосвязей. Интерпретация полученных значений зависит от поставленных целей: положительная корреляция достигается путём увеличения риска (диспресии, волатильности), отрицательная -- способствует диверсификации.

Ожидаемая доходность портфеля выражается формулой:
\[
\mathbb{E}[R_p]=\sum_{i=1}^n{w_i}{\mathbb{E}[R_i]}
\]

То есть \(\mathbb{E}[R_p]\) задаёт ожидаемую доходность, а \(\sigma^2_p\) -- риск её получения. Эта модель положила основу принципу диверсификации: оптимальный портфель определяется как компромисс между риском и доходностью \cite{markowitz1952}

Развитием идей Г. Марковица связано с моделью оценки капитальных активов (Capital Assets Pricing Model) предложенной Уильямом Шарпом в 1964 году \cite{sharpe2001}. CAPM устанавливает зависимость ожидаемой доходности аткива от систематического риска: 
\[
\mathbb{E}[R_i]=R_{f}+\beta_i(\mathbb{E}[R_{m}]-R_{f}),
\]
где \(R_{f}\) -- безрисковая ставка, \(\mathbb{E}[R_{m}]\) -- доходность рыночного портфеля, а коэффициент \(\beta_i\) определяется как:
\[
\beta_i=\frac{\text{Cov}(R_i-R_{m})}{\sigma_m^2}
\]
Интерпретация модели заключается в следующем: $R_{f}$ задаёт минимальную доходность без риска, $\mathbb{E}[R_m] - R_{f}$ представляет собой рыночную премию за риск, а коэффициент $\beta_i$ характеризует чувствительность доходности актива к колебаниям рынка. При $\beta_i = 1$ актив полностью повторяет динамику рынка; при $\beta_i > 1$ он более волатилен; при $\beta_i < 1$ — менее рискован.

Дальнейшие развитие теория получила в работах Джеймса Тобина, который в 1958 году предложил включать в модель безрисковый актив \cite{tobin1958}. Это привело к формированию концепции линии рынка капитала (Capital Market Line, CML), описывающей линейную зависимость между доходностью и риском:
\[
\mathbb{E}[R_{p}]=R_{m} + \frac{\mathbb{E}[R_{m}]-R_{f}}{\sigma_{m}}\cdot \sigma_{p}
\]
Прямая CML проходит через точку с координатами \((0,R_{f})\) и касается эффективной границы в точке рыночного портфеля. Любой портфель может быть представлен как комбинация рыночного портфеля и безрискового актива: 
\[
w\;\cdot\;(\text{Market Portfolio})\;\cdot\;(1-w)\cdot(\text{Risk-free Assets})
\]
а при \(w>1\) изспользуется заём (плечо), который увеличивает долю рискованного актива.

Несмотря на фундаментальное значение данных моделей для современной финансовой науки, их применимость к облигациям ограничена. Коэффициент $\beta$ и рыночный риск плохо отражают специфику долговых инструментов. Реакция стоимости облигаций на изменение ставок носит нелинейный характер и описывается через дюрацию и выпуклость, а не через линейную зависимость, предполагаемую CAPM. Использование стандартного отклонения доходностей как меры риска (в модели CML) не учитывает структуру кривой доходности и особенности переоценки долговых обязательств.То есть, классические модели формируют общий фундамент, но для анализа облигационного портфеля они оказываются недостаточными. Это и обуславливает необходимость применения специализированных методов оценки процентного риска.

\subsection{Классические методы процентного риска}
В отличие от акций, доходность облигаций предопределена условиями выпуска, а цена зависит от уровня процентных ставок. Поэтому классические методы анализа процентного риска основывается на аппроксимации переоценки стоимости долгового инструмента при изменении ставок.

Прежде чем переходить к этим показателям, зафиксируем базовую зависимость между ценой облигации и её доходностью к погашению. Цена облигации определяется как сумма дисконтированных денежных потоков:
\begin{equation} \label{eq:bond_price}
P = \sum_{t=1}^{T} \frac{CF_t}{(1+y)^t},
\end{equation}
где $CF_t$ — денежный поток (купон или номинал в момент времени $t$), $y$ — доходность к погашению, $T$ — срок до погашения.

Доходность к погашению играет фундаментальную роль, так как именно через неё устанавливается связь между ценой и структурой будущих выплат.

На основе \eqref{eq:bond_price} вводится показатель дюрации (duration) как средневзвешенный срок поступления денежных потоков, отражающий чувствительность цены облигации к изменению ставки доходности:
\begin{equation} \label{eq:duration}
D = \frac{\sum_{t=1}^{T} t \cdot \frac{CF_t}{(1+y)^t}}{P}.
\end{equation}

Модель показывает, что каждый поток учитывается пропорционально своей приведённой стоимости в цене облигации. Фактически дюрация представляет собой первую производную функции цены по доходности и отражает локальный наклон зависимости «цена–доходность».

Для малых изменений доходности связь между ценой и ставкой может быть аппроксимирована линейно:
\begin{equation} \label{eq:duration_approx}
\frac{\Delta P}{P} \approx -D \cdot \Delta y, \tag{2.1}
\end{equation}
где $\Delta P / P$ — относительное изменение цены, а $\Delta y$ — изменение доходности (обычно выражаемое в процентных пунктах).

Однако зависимость цены от доходности нелинейна, и при значительных изменениях ставок линейного приближения оказывается недостаточно. Для учёта кривизны вводится показатель выпуклости (convexity), основанный на второй производной цены по доходности:
\begin{equation} \label{eq:convexity}
C = \frac{1}{P} \sum_{t=1}^{T} \frac{CF_t \cdot t(t+1)}{(1+y)^{t+2}}.
\end{equation}

Совместное приближение второго порядка имеет вид:
\begin{equation} \label{eq:duration_convexity}
\frac{\Delta P}{P} \approx -D \cdot \Delta y + \tfrac{1}{2} C \cdot (\Delta y)^2.\tag{3.1}
\end{equation}

Первый член аппроксимации отражает линейную чувствительность, а второй — кривизну зависимости. Учёт выпуклости особенно важен при стрессовых изменениях ставок: если доходность увеличивается на несколько процентных пунктов, модель только с дюрацией систематически недооценивает потери.

Для количественной оценки возможных потерь также широко применяется показатель Value-at-Risk (VaR) -- граница убытков, которую с вероятностью $\alpha$ портфель не превысит за заданный горизонт:
\begin{equation} \label{eq:var}
\text{VaR}_\alpha(P) = \inf\{\ell \in \mathbb{R} : \mathbb{P}(L \leq \ell) \geq \alpha\},
\end{equation}
где $L$ — случайная величина убытка.

Однако VaR не содержит информации о величине потерь «за пределами квантиля» и может недооценивать риск при тяжёлых хвостах распределения. Для преодоления этих ограничений применяется Conditional Value-at-Risk (CVaR), или Expected Shortfall (ES), отражающий средний убыток в наихудших $(1-\alpha)$ процентах случаев:
\begin{equation} \label{eq:cvar}
\text{CVaR}_\alpha(P) = \mathbb{E}[\,L \mid L \geq \text{VaR}_\alpha(P)\,].
\end{equation}

Показатели дюрации и выпуклости предоставляют локальную аппроксимацию чувствительности цены к изменениям ставок, тогда как VaR и CVaR дают вероятностную оценку потерь. Вместе они формируют базовый инструментарий для анализа процентного риска. Однако корректная оценка в условиях высокой волатильности требует учёта динамики условной дисперсии и автокорреляции доходностей. Строгие постановки задач, формальные определения и современные модели будут изложены в Главе 2.
\clearpage
