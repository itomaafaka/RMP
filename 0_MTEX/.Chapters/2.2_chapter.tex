\subsection{Статистические методы анализа процентного риска}
\subsubsection{Виды доходностей облигаций}
В 1 главе была определена доходность облигации через цену и ставку к погашению, а в разделе 2.1 -- была абстрактно введена случайная величина $\xi$ для которой можно вычислять различные показатели. Теперь необходимо объединить эти подходы. В анализе долговых ценных бумаг можно выделить следующие виды доходностей для облигаций:

Простая (фактическая) доходность за период $[t-1,t]$ определяется как
\[

R_t = \frac{P_t + CF_t - P_{t-1}}{P_{t-1}},
\]
где $P_t$ --- рыночная цена облигации в момент $t$, $CF_t$ --- денежный поток (купон или амортизация) между $t-1$ и $t$.  
При этом цена $P_t$ может трактоваться как чистая (без учёта накопленного купонного дохода, без НКД) или как <<грязная>> (с учётом НКД).  Для корректной оценки доходностей предпочтительнее использовать именно <<грязные>> цены, поскольку они отражают реальный поток, получаемый инвестором.

Для статистического анализа также можно использовать логарфмическую доходность. 
\[
\label{eq:rtln}
r_t = \ln \frac{P_t + CF_t}{P_{t-1}}. \tag{13}
\]
Её преимущество заключается в аддитивности (суммарная доходность за каждый период равна сумме доходностей за каждый )
\[
\label{eq:lnadd}
\ln \frac{P_t}{P_{t-h}} = \sum_{i=1}^h r_{t-i}. \tag{14}
\]

Для долгосрочного анализа и построения кривой доходности важна доходность к погашению (YTM). Она определяется как ставка $y_t$, которая при дисконтировании будущих потоков восстанавливает цену облигации:
\[
\label{eq:ytmdis}
P_t = \sum_{k=1}^T \frac{C}{(1+y_t)^k} + \frac{N}{(1+y_t)^T} \tag{15}
\]
где $C$ --- купон, $N$ --- номинал, $T$ --- число оставшихся периодов. 

Чувствительность цены к изменению ставок, как было показано в главе 1, измеряется дюрацией \eqref{eq:duration} и выпуклостью \eqref{eq:convexity}. Эти показатели будут использоваться далее в стресс-тестах.

\subsubsection{ Модели условного среднего: ARIMA и ARIMAX}
После обоснованного введения доходностей облигаций как случайных величин возникает задача описать их динамику во времени. Временные ряды доходностей часто обладают автокорреляцией, когда значения текущего периода зависят от прошлых. 
Например, рост ставок сегодня может увеличить вероятность их роста завтра. Для формализации этой зависимости используют авторегрессионные модели.

В моделях авторегресии (Autoregression, AR) значение ряда зависит от $p$ предыдущих значений:
\[
\label{eq:AR}
r_t = c\;+\;\sum^p_{i=1}\phi_i \cdot r_{t-i} + \varepsilon_t \tag{16}
\]
здесь $c$ -- константа, $\phi_i$ -- коэффициент авторегрессии, $\varepsilon_t$ -- белый шум или ошибка прогноза (шок) в момент времени $t$, то есть независимая случайная величина с нулевым средним и постоянной дисперсией -- некая часть доходности, которая не может быть предсказана на основе прошлых данных.

В моделях скользящей средней (Moving Average, MA) значение зависит от прошлых ошибок
\[
\label{eq:MA}
r_t=\varepsilon_t+\sum^{q}_{j=1} \theta_j\cdot \varepsilon_{t-j} \tag{17}
\]
где $\theta_j$ -- влияние прошлых шоков

Интерпретация коэффициентов проста: $\phi_i$ измеряют вклад прошлых доходностей $r_{t-i}$ в текущее значение $r_t$, 
а $\theta_j$ отражают влияние прошлых ошибок прогноза $\varepsilon_{t-j}$. 
Так, $\phi_1$ показывает, насколько сильно сегодняшняя доходность зависит от вчерашней, 
а $\theta_1$ --- насколько сильно на неё повлиял вчерашний непредсказанный шок.

Объединяя \eqref{eq:AR} и \eqref{eq:MA}, получаем модель авторегрессии -- скользящей средней (Autoregressive moving average, ARMA) с параметрами ($p,q$):
\[
\label{eq:ARMA}
r_t = c + \sum_{i=1}^{p}\phi_i r_{t-i} + \varepsilon_t + \sum_{j=1}^{q}\theta_j \varepsilon_{t-j} \tag{18}
\]

Важным замечанием является то, что при построении моделей ключевую роль играет стационарность -- свойство временного ряда, у которого среднее, дисперсия и ковариации не зависят от времени. Финансовые цены и ставки обычно нестационарны, тогда как доходности после преобразования чаще можно считать стационарными.  

Если же стационарность отсутствует, применяют оператор конечных разностей $(1-L)^d$, где $L$ — лаг-оператор. Так из \eqref{eq:ARMA} получается интегрированная модель авторегрессии -- скользящего среднего, или \text{ARIMA} с параметрами ($p,d,q$), де $d$ показывает, сколько раз нужно продифференцировать ряд, чтобы сделать его стационарным:
\[
\label{eq:ARIMA}
\phi(L)(1-L)^d r_t = \theta(L)\varepsilon_t, \quad\Rightarrow\quad
r_t - (1+\phi) r_{t-1} + \phi r_{t-2} = c + \varepsilon_t + \theta \varepsilon_{t-1}. \tag{19}
\]

Также для учёта внешних факторов применяется расширение ARIMAX (Autoregressive Integrated Moving Average with eXogenous inputs):
\[
\label{eq:ARIMAX}
\phi(L)(1-L)^d r_t = \beta^\top x_t + \theta(L)\varepsilon_t, \tag{20}
\]
где $x_t$ — вектор экзогенных переменных (ключевая ставка, инфляция и т.д), а $\beta$ — коэффициенты их влияния.

Важное замечание, что коэффициенты моделей ($\phi_i, \theta_j, \beta, c$) не рассчитываются вручную. Для AR-моделей можно использовать уравнения Юла–Уокера, связывающие коэффициенты с автокорреляционной функцией, однако в общем случае ARIMA/ARIMAX применяют метод максимального правдоподобия (MLE). Этот метод подбирает параметры так, чтобы вероятность наблюдаемых данных была максимальной. 
На практике вычисления выполняются специализированными пакетами (например, statsmodels в Python), которые автоматически оценивают параметры и строят прогнозы.

Таким образом, ARIMA \eqref{eq:ARIMA} моделирует внутреннюю динамику доходностей через их прошлые значения и ошибки прогноза, а ARIMAX \eqref{eq:ARIMAX} позволяет дополнительно учитывать макроэкономические факторы. 
Такие модели используются для прогноза распределения доходностей облигаций, 
а затем на их основе рассчитываются риск-метрики, такие как Value-at-Risk и Conditional Value-at-Risk.