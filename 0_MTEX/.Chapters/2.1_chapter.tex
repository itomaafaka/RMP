\section{ Математический аппарат моделирования процентного риска}
\subsection{ Вероятностно-мерный фундамент}

Прежде чем переходить к прикладным моделям, необходимо зафиксировать строгую основу. Мы будем работать в вероятностном пространстве, где каждый рыночный сценарий заранее зашит в пространство элементарных исходов. Формальные определения здесь важны не сами по себе, а потому что именно на них опираются все последующие риск-метрики.

Пусть $\Omega$ -- пространство элементарных исходов, которое можно интерпретировать как множество всех возможных сценариев рынка. Например, одно $\omega \in \Omega$ соответствует конкретному пути динамики процентных ставок за день.
Далее появляется необходимость свободно совершать над событиями логические операции (не/или/и) и оставаться внутри однго и того же класса  событий. Ведь было бы странно, например, если бы мы складывали доходности, а в ответе получили нечто за пределами финансового пространства, нам бы просто было неудобно работать с такой структурой. Поэтому возникает необходимость ввести понятие \emph{алгебры}.

\emph{Алгеброй множеств} \(\mathcal{A}\) -- называется семейство событий, для которых выполняются три простых условия:
\vspace{-0.5\baselineskip}
\begin{enumerate}
    \item \(\varnothing\in\mathcal{A}\) -- наличие пустого множества;
    \vspace{-0.5\baselineskip}
    \item \(\forall A\in \mathcal{A}:A^C\in\mathcal{A}\) -- замкнутость дополнений;
    \vspace{-0.5\baselineskip}
    \item \(\forall A, B \in\mathcal{A}:A\cup B \in \mathcal{A}\) -- замкнутость  конечных объединений;
\end{enumerate}
\vspace{-0.5\baselineskip}
Это позволит нам формально описывать некоторые события: <<рост был хотя бы в одном из заданного количества дней>> или <<за заданный промежуток не произошло падений>>

Однако может возникнуть ряд событий, когда алгебры \(\mathcal{A}\) может быть недостаточно. В финансах мы постоянно сталкиваемся с событиями, которые описываются через пределы или счётные конструкции. Такие конструкции требуют замкнутости не только относительно конечных, но и относительно счётных операций. Поэтому вводится более сильное понятие: \(\sigma\)-алгеброй событий \(\mathcal{F}\) называется семейство подмножеств \(\Omega\), удовлетворяющих условиям:
\vspace{-0.5\baselineskip}
\begin{enumerate}
    \item \(\varnothing\in\mathcal{F}\) -- наличие пустого множества;
    \vspace{-0.5\baselineskip}
    \item \(\forall A\in \mathcal{F}:A^C\in\mathcal{F}\) -- замкнутость дополнений;
    \vspace{-0.5\baselineskip}
    \item \(\forall A_1, A_2\dots A_n \in\mathcal{F}:\bigcup_{n=1}^\infty A_n \in \mathcal{A}\) -- \(\sigma\)-аддитивность или счётная-аддитивность;
\end{enumerate}
\vspace{-0.5\baselineskip}

Но, даже $\sigma$-алгебра $\mathcal{F}$ всех подмножеств $\Omega$ оказывается слишком широкой: среди них существуют множества, которым невозможно согласованно приписать вероятность. Чтобы избежать подобных парадоксов, на вещественной прямой $\mathbb{R}$ используют борелевскую $\sigma$-алгебру $\mathfrak{B}(\mathbb{R})$ -- наименьшая $\sigma$-алгебра, содержащая все открытые множества (а значит, интервалы, полуинтервалы и замкнутые множества). Именно на ней строятся распределения случайных величин и, как следствие, все риск-метрики, используемые далее.

Теперь, имея пространство сценариев $\Omega$ и класс событий $\mathcal{F}$, можно сказать: пара \((\Omega, \mathcal{F})\) -- измеримое пространство. Для измерения вероятности вводится понятие мера -- функция \(\mu : \mathcal{F}\to[0;+\infty)\), которая соответствует правилам, мера пустого множества -- <<мера нуль>> \(\mu(\varnothing)=0\) и \(\sigma\)-аддитивность:\;\(\forall A_i\cap A_j = \varnothing \rightarrow\mu\left(\bigcup^\infty_{n=1}A_n\right)=\sum_{n=1}^\infty\mu(A_n)\)

Важным примером меры является --- \emph{мера Лебега} $\lambda$ на $\mathbb R$, которую обычно обозначают $dx$. 
Интуитивно она соответствует привычной длине: например, $\lambda([a,b])=b-a$. 
Отдельные точки при этом имеют меру ноль: $\lambda(\{a\})=0$. 
Таким образом, мера Лебега задаёт естественный способ «измерять» подмножества прямой и именно она используется, когда мы записываем привычные интегралы $\int f(x)\,dx$. 

В случае финансовых моделей нас интересует уже не длина интервала, а вероятность наступления рыночного сценария, откуда и появляется специальный класс мер — вероятностные меры, которые удовлетворяют аксиомам Колмогорова и мы получаем следующее:

Вероятностное пространство -- тройка \((\Omega, \mathcal{F}, \mathbb{P})\), где \(\mathbb{P}\) - вероятностная мера на \(\mathcal{F}\), то есть \(\sigma\)-аддитивная функция \(\mathbb{P}:\mathcal{F}\to[0,1]\), которая удовлетворяет трём аксиомам Колмогорова: 
\vspace{-0.5\baselineskip}
\begin{enumerate}
    \item \(\mathbb{P}(\Omega)=1\) -- условия нормировки;
    \vspace{-0.5\baselineskip}
    \item \(\mathbb{P}(A) \ge 0 : \forall A \in \mathcal{F}\) -- условия неотрицательности;
    \vspace{-0.5\baselineskip}
    \item \(\mathbb{P}(\bigsqcup_{n=1}^\infty A_n)=\sum_{n=1}^\infty\mathbb{P}(A_n)\) -- условие \(\sigma\)-аддитивности.
\end{enumerate}
\vspace{-0.5\baselineskip}

Но для анализа рисков нас интересуют не просто события вида «произошло / не произошло», а числовые характеристики сценариев: убыток портфеля, доходность облигации, изменение ставки и так далее. Поэтому вводится следующее понятие

Случайная величина \(\xi\) на вреоятностном пространстве \((\Omega, \mathcal{F}, \mathbb{P})\) называется измеримым отображением:
$$
\xi : (\Omega,\mathcal F) \to (\mathbb R,\mathfrak B(\mathbb R)),
$$
где \(\mathfrak B(\mathbb R)\) -- борелевская $\sigma$-алгебра. которое каждой траектории рынка $\omega \in \Omega$ ставит в соответствие число $\xi(\omega)$. То есть формализует рыночные сценарии через изменение числовых показателей, с которыми возможно проводить дальнейшие расчёты.

Распределение случайной величины будет \(\xi\) убдет задаваться как образ вероятностной меры: 
$$
\mathbb P_\xi(B)=\mathbb P\{\xi\in B\},\quad B\in\mathfrak B(\mathbb R).
$$

Чтобы было удобнее работать с распределением случайной величины, вводят понятие функции распределение (cumulative distribution function, CDF):
Функция распределения случайной величины $\xi$ определяется как:
\begin{equation}
\label{eq:cdf}
F_\xi(x)=\mathbb{P}(\xi\leq x), \quad x \in \mathbb{R}     
\tag{4}
\end{equation}
и обладает рядом свойств: 
\begin{enumerate}
\vspace{-0.5\baselineskip}
    \item Монотонно неубывающая $x_1\leq x_2 \rightarrow f(x_1)\leq f(x_2)$
    \item Ограниченная $\quad\lim_{x \to -\infty} F_\xi(x) = 0, \qquad \lim_{x \to +\infty} F_\xi(x) = 1$
    \item Правосторонняя непрерывность $F_{\xi}(x)=\lim_{x\to\ x_0^+}F_{\xi}(x_0)$
    \item $\forall\;\;a < b \rightarrow \mathbb{P}(a <\xi\leq b)=F_{\xi}(a)-F_{\xi}(b)$
\end{enumerate}
Вместе с CDF \eqref{eq:cdf} справедливо ввести определение \emph{квантиля} -- значение, которое заданное случайная величина $\xi$ не превышает с фиксированной вероятностью. Иначе говоря, квантиль  уровня $\alpha\in(0,1)$ задаётся формулой: 
\begin{equation}
\label{eq:quantile}
    q_\xi(\alpha) = \inf \{x \in \mathbb R : F_\xi(x) \ge \alpha\} \tag{5}
\end{equation}
Немало важно заметить, что функция распределения не всегда имеют одинаковую природу. Их можно разделить на два вида -- дискретная функция распределения (<<лестница>>) и непрервная функция распределения (<<гладкая кривая>>). На рисунке ~\ref{fig:cdf} представлены абстрактные примеры наглядно иллюстрирующие виды распределений 
\begin{figure}[H]
    \centering
    \includegraphics[width=1\linewidth]{graph/DC_cdf.pdf}
    \caption{Дискретная и непрерывная функция распределения}
    \label{fig:cdf}
\end{figure}

Это различие можно формализовать языком теории меры. В дискретном случае вероятность задаётся «атомами» -- отдельными точками, которым приписан положительный вес, вероятность, в данном случае, накапливается скачками, а в непрерывном случае вероятность «распределена» по оси $\mathbb R$, и функция распределения $F_\xi(x)$ оказывается абсолютно непрерывной. Тогда у неё существует производная почти всюду (almost everywhere, a.e.), которая и называется \emph{функцией плотности распределения} $f_\xi(x)$. Формально переход от одной меры к другой описывает фундаментальный результат -- теорема Радона--Никодима (или производная Радона-Никодима).
\begin{theorem}[Теорема Радона--Никодима]
Если $\nu \ll \mu$\footnote{\;\;Читается как: мера $\nu$ абсолютно непрерывно относительно меры $\mu$} -- $\sigma$-конечные меры, то существует измеримая функция $p=\tfrac{d\nu}{d\mu}$ такая, что
\[
\nu(A) = \int_A p \, d\mu, \qquad A \in \mathcal F.
\]
\end{theorem}
В нашем случае: если распределение $\mathbb P_\xi$ абсолютно непрерывно относительно меры Лебега $dx$, 
то существует функция плотности $f_\xi(x)$, для которой
\[
\label{eq:density}
F_\xi(x) = \int_{-\infty}^x f_\xi(t)\,dt \tag{6}
\quad \Rightarrow \quad 
\overbrace{f_\xi(x) = \frac{d}{dx}F_\xi(x)}^{a.e.}\quad
\Leftrightarrow \quad \mathbb{P}_\xi=\int_A f_\xi(t)dt 
\]

Чтобы не работать с полным распределениями часто используют их числовыми характеристиками, которые описывают "среднее положение" случайной величины и разброс её значений. 

Математическое ожидание $\mathbb{E}[\xi]$ -- это средняя доходность или убыток по портфелю:
\[
\label{eq:mean}
\mathbb{E}[\xi]=\int_{\mathbb{R}}x\,d\mathbb{P}_{\xi}(x) \tag{7}
\]

Диспресия (Variance, Var, $\sigma_\xi^2$) и стандартное отклонение ($\sigma_\xi$) -- степень разброса исхода событий
\[
\label{eq:variance}
\text{Var}(\xi):=\mathbb{E}\left[(\xi -\mathbb{E}[\xi])^2\right]:=\sigma_\xi^2 \tag{8}
\]
\[
\label{eq:std_sqrt}
\sigma_\xi:=\sqrt{\text{Var}(\xi)} \tag{9}
\]
В случае, когда распределение $\mathbb P_\xi$ абсолютно непрерывно относительно меры Лебега $dx$, то тогда и только тогда у него будет существовать плотность $f_\xi(x)$, а математическое ожидание и дисперсия будут иметь такой вид:
$$
\mathbb{E}[\xi]=\int_{\mathbb{R}}xf_\xi(x)dx
\qquad
{Var}(\xi):=\int_\mathbb{R}(x-\mathbb{E}(\xi))^2f_\xi(x)dx
$$

В дальнейшем нам понадобится уметь сравнивать несколько случайных величин, и для их анализа важны будут не только их собственные разбросы, но и то, как они связаны друг с другом. Именно это объясняют следующие показатели:

Ковариация -- индикатор, который показывает направление линейной зависимости между двумя случайными величинами, то есть если она положительна величины имеют общую тенденцию к росту или падению, если отрицательна, то тенденция <<разнонаправленности>>, одно растёт, другое падает, в случае когда ковариация равна нулю -- это отсутсвие линейной зависимости, но не обязательно полной. 
\[
\label{eq:covar}
\text{Cov}(\xi,\eta) = \mathbb{E}[(\xi-\mathbb{E}[\xi])\cdot(\eta-\mathbb{E}[\eta])]\qquad\tag{10}
\]

Корреляция -- мера, которая определяет силу направления линейной зависимости, логика сохраняется такая же, ближе к 1/-1 -- сильная положительная/отрицательная связь соответственно, а 0 -- слабая связь или полное её отстутсвие
\[
\label{eq:correl}
\rho(\xi, \eta)=\frac{\text{Cov}(\xi,\eta)}{\sigma_\xi\cdot\sigma_\eta}\qquad\tag{11}
\]
\clearpage
\subsection{Статистические методы анализа процентного риска}
\subsubsection{Виды доходностей облигаций}
В 1 главе была определена доходность облигации через цену и ставку к погашению, а в разделе 2.1 -- была абстрактно введена случайная величина $\xi$ для которой можно вычислять различные показатели. Теперь необходимо объединить эти подходы. В анализе долговых ценных бумаг можно выделить следующие виды доходностей для облигаций:

Простая (фактическая) доходность за период $[t-1,t]$ определяется как
\[

R_t = \frac{P_t + CF_t - P_{t-1}}{P_{t-1}},
\]
где $P_t$ --- рыночная цена облигации в момент $t$, $CF_t$ --- денежный поток (купон или амортизация) между $t-1$ и $t$.  
При этом цена $P_t$ может трактоваться как чистая (без учёта накопленного купонного дохода, без НКД) или как <<грязная>> (с учётом НКД).  Для корректной оценки доходностей предпочтительнее использовать именно <<грязные>> цены, поскольку они отражают реальный поток, получаемый инвестором.

Для статистического анализа также можно использовать логарфмическую доходность. 
\[
\label{eq:rtln}
r_t = \ln \frac{P_t + CF_t}{P_{t-1}}. \tag{13}
\]
Её преимущество заключается в аддитивности (суммарная доходность за каждый период равна сумме доходностей за каждый )
\[
\label{eq:lnadd}
\ln \frac{P_t}{P_{t-h}} = \sum_{i=1}^h r_{t-i}. \tag{14}
\]

Для долгосрочного анализа и построения кривой доходности важна доходность к погашению (YTM). Она определяется как ставка $y_t$, которая при дисконтировании будущих потоков восстанавливает цену облигации:
\[
\label{eq:ytmdis}
P_t = \sum_{k=1}^T \frac{C}{(1+y_t)^k} + \frac{N}{(1+y_t)^T} \tag{15}
\]
где $C$ --- купон, $N$ --- номинал, $T$ --- число оставшихся периодов. 

Чувствительность цены к изменению ставок, как было показано в главе 1, измеряется дюрацией \eqref{eq:duration} и выпуклостью \eqref{eq:convexity}. Эти показатели будут использоваться далее в стресс-тестах.

\subsubsection{ Модели условного среднего: ARIMA и ARIMAX}
После обоснованного введения доходностей облигаций как случайных величин возникает задача описать их динамику во времени. Временные ряды доходностей часто обладают автокорреляцией, когда значения текущего периода зависят от прошлых. 
Например, рост ставок сегодня может увеличить вероятность их роста завтра. Для формализации этой зависимости используют авторегрессионные модели.

В моделях авторегресии (Autoregression, AR) значение ряда зависит от $p$ предыдущих значений:
\[
\label{eq:AR}
r_t = c\;+\;\sum^p_{i=1}\phi_i \cdot r_{t-i} + \varepsilon_t \tag{16}
\]
здесь $c$ -- константа, $\phi_i$ -- коэффициент авторегрессии, $\varepsilon_t$ -- белый шум или ошибка прогноза (шок) в момент времени $t$, то есть независимая случайная величина с нулевым средним и постоянной дисперсией -- некая часть доходности, которая не может быть предсказана на основе прошлых данных.

В моделях скользящей средней (Moving Average, MA) значение зависит от прошлых ошибок
\[
\label{eq:MA}
r_t=\varepsilon_t+\sum^{q}_{j=1} \theta_j\cdot \varepsilon_{t-j} \tag{17}
\]
где $\theta_j$ -- влияние прошлых шоков

Интерпретация коэффициентов проста: $\phi_i$ измеряют вклад прошлых доходностей $r_{t-i}$ в текущее значение $r_t$, 
а $\theta_j$ отражают влияние прошлых ошибок прогноза $\varepsilon_{t-j}$. 
Так, $\phi_1$ показывает, насколько сильно сегодняшняя доходность зависит от вчерашней, 
а $\theta_1$ --- насколько сильно на неё повлиял вчерашний непредсказанный шок.

Объединяя \eqref{eq:AR} и \eqref{eq:MA}, получаем модель авторегрессии -- скользящей средней (Autoregressive moving average, ARMA) с параметрами ($p,q$):
\[
\label{eq:ARMA}
r_t = c + \sum_{i=1}^{p}\phi_i r_{t-i} + \varepsilon_t + \sum_{j=1}^{q}\theta_j \varepsilon_{t-j} \tag{18}
\]

Важным замечанием является то, что при построении моделей ключевую роль играет стационарность -- свойство временного ряда, у которого среднее, дисперсия и ковариации не зависят от времени. Финансовые цены и ставки обычно нестационарны, тогда как доходности после преобразования чаще можно считать стационарными.  

Если же стационарность отсутствует, применяют оператор конечных разностей $(1-L)^d$, где $L$ — лаг-оператор. Так из \eqref{eq:ARMA} получается интегрированная модель авторегрессии -- скользящего среднего, или \text{ARIMA} с параметрами ($p,d,q$), де $d$ показывает, сколько раз нужно продифференцировать ряд, чтобы сделать его стационарным:
\[
\label{eq:ARIMA}
\phi(L)(1-L)^d r_t = \theta(L)\varepsilon_t, \quad\Rightarrow\quad
r_t - (1+\phi) r_{t-1} + \phi r_{t-2} = c + \varepsilon_t + \theta \varepsilon_{t-1}. \tag{19}
\]

Также для учёта внешних факторов применяется расширение ARIMAX (Autoregressive Integrated Moving Average with eXogenous inputs):
\[
\label{eq:ARIMAX}
\phi(L)(1-L)^d r_t = \beta^\top x_t + \theta(L)\varepsilon_t, \tag{20}
\]
где $x_t$ — вектор экзогенных переменных (ключевая ставка, инфляция и т.д), а $\beta$ — коэффициенты их влияния.

Важное замечание, что коэффициенты моделей ($\phi_i, \theta_j, \beta, c$) не рассчитываются вручную. Для AR-моделей можно использовать уравнения Юла–Уокера, связывающие коэффициенты с автокорреляционной функцией, однако в общем случае ARIMA/ARIMAX применяют метод максимального правдоподобия (MLE). Этот метод подбирает параметры так, чтобы вероятность наблюдаемых данных была максимальной. 
На практике вычисления выполняются специализированными пакетами (например, statsmodels в Python), которые автоматически оценивают параметры и строят прогнозы.

Таким образом, ARIMA \eqref{eq:ARIMA} моделирует внутреннюю динамику доходностей через их прошлые значения и ошибки прогноза, а ARIMAX \eqref{eq:ARIMAX} позволяет дополнительно учитывать макроэкономические факторы. 
Такие модели используются для прогноза распределения доходностей облигаций, 
а затем на их основе рассчитываются риск-метрики, такие как Value-at-Risk и Conditional Value-at-Risk.